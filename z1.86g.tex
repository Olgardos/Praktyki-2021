\documentclass[12pt, a4paper]{article}
\usepackage[utf8]{inputenc}
\usepackage{polski}
\usepackage{amsthm}  %pakiet do tworzenia twierdzeń itp.
\usepackage{amsmath} %pakiet do niektórych symboli matematycznych
\usepackage{amssymb} %pakiet do symboli mat., np. \nsubseteq
\usepackage{amsfonts}
\usepackage{graphicx} %obsługa plików graficznych z rozszerzeniem png, jpg
\theoremstyle{definition} %styl dla definicji
\newtheorem{zad}{} 
\title{Multizestaw zadań}
\author{Radosław Grzyb}
%\date{\today}
\date{}
\newcounter{liczniksekcji}
\newcommand{\kategoria}[1]{\section{#1}} %olreślamy nazwę kateforii zadań
\newcommand{\zadStart}[1]{\begin{zad}#1\newline} %oznaczenie początku zadania
\newcommand{\zadStop}{\end{zad}}   %oznaczenie końca zadania
%Makra opcjonarne (nie muszą występować):
\newcommand{\rozwStart}[2]{\noindent \textbf{Rozwiązanie (autor #1 , recenzent #2): }\newline} %oznaczenie początku rozwiązania, opcjonarnie można wprowadzić informację o autorze rozwiązania zadania i recenzencie poprawności wykonania rozwiązania zadania
\newcommand{\rozwStop}{\newline}                                            %oznaczenie końca rozwiązania
\newcommand{\odpStart}{\noindent \textbf{Odpowiedź:}\newline}    %oznaczenie początku odpowiedzi końcowej (wypisanie wyniku)
\newcommand{\odpStop}{\newline}                                             %oznaczenie końca odpowiedzi końcowej (wypisanie wyniku)
\newcommand{\testStart}{\noindent \textbf{Test:}\newline} %ewentualne możliwe opcje odpowiedzi testowej: A. ? B. ? C. ? D. ? itd.
\newcommand{\testStop}{\newline} %koniec wprowadzania odpowiedzi testowych
\newcommand{\kluczStart}{\noindent \textbf{Test poprawna odpowiedź:}\newline} %klucz, poprawna odpowiedź pytania testowego (jedna literka): A lub B lub C lub D itd.
\newcommand{\kluczStop}{\newline} %koniec poprawnej odpowiedzi pytania testowego 
\newcommand{\wstawGrafike}[2]{\begin{figure}[h] \includegraphics[scale=#2] {#1} \end{figure}} %gdyby była potrzeba wstawienia obrazka, parametry: nazwa pliku, skala (jak nie wiesz co wpisać, to wpisz 1)
\begin{document}
\maketitle
\kategoria{Wikieł/Z1.86g}
\zadStart{Zadanie z Wikieł Z 1.86g moja wersja nr [nrWersji]}
%[p1]:[2,4,5,7,8,10,11,13,14,16,17,19,20,22,25]
Rozwiązać nierówność:
$$2^{[p1]x+3}-5^{[p1]x}<7\cdot2^{[p1]x-2}-3\cdot5^{[p1]x-1}$$
\zadStop
\rozwStart{Radosław Grzyb}{}
Przekształcamy lekko naszą nierówność:
$$8\cdot2^{[p1]x}-5^{[p1]x}<\frac{7}{4}\cdot2^{[p1]x}-\frac{3}{5}\cdot5^{[p1]x}$$
Mnożymy obie strony przez $20$:
$$160\cdot2^{[p1]x}-20\cdot5^{[p1]x}<35\cdot2^{[p1]x}-12\cdot5^{[p1]x}$$
$$160\cdot2^{[p1]x}-35\cdot2^{[p1]x}<20\cdot5^{[p1]x}-12\cdot5^{[p1]x}$$
$$125\cdot2^{[p1]x}<8\cdot5^{[p1]x}$$
Dzielimy obie strony przez $5^{[p1]x}$ oraz $125$:
$$\left(\frac{2}{5}\right)^{[p1]x}<\frac{8}{125}$$ 
$$\left(\frac{2}{5}\right)^{[p1]x}<\left(\frac{2}{5}\right)^3$$
Logarytmując obie strony otrzymujemy:
$$[p1]x>3\implies x>\frac{3}{[p1]}$$
\rozwStop
\odpStart
\odpStop
\testStart
A.$$x>[p1]$$
B.$$x<\frac{3}{[p1]}$$
C.$$x>\frac{1}{[p1]}$$
D.$$x>\frac{3}{[p1]}$$
\testStop
\kluczStart
D
\kluczStop
\end{document}