\documentclass[12pt, a4paper]{article}
\usepackage[utf8]{inputenc}
\usepackage{polski}

\usepackage{amsthm}  %pakiet do tworzenia twierdzeń itp.
\usepackage{amsmath} %pakiet do niektórych symboli matematycznych
\usepackage{amssymb} %pakiet do symboli mat., np. \nsubseteq
\usepackage{amsfonts}
\usepackage{graphicx} %obsługa plików graficznych z rozszerzeniem png, jpg
\theoremstyle{definition} %styl dla definicji
\newtheorem{zad}{} 
\title{Multizestaw zadań}
\author{Mirella Narewska}
%\date{\today}
\date{}
\newcounter{liczniksekcji}
\newcommand{\kategoria}[1]{\section{#1}} %olreślamy nazwę kateforii zadań
\newcommand{\zadStart}[1]{\begin{zad}#1\newline} %oznaczenie początku zadania
\newcommand{\zadStop}{\end{zad}}   %oznaczenie końca zadania
%Makra opcjonarne (nie muszą występować):
\newcommand{\rozwStart}[2]{\noindent \textbf{Rozwiązanie (autor #1 , recenzent #2): }\newline} %oznaczenie początku rozwiązania, opcjonarnie można wprowadzić informację o autorze rozwiązania zadania i recenzencie poprawności wykonania rozwiązania zadania
\newcommand{\rozwStop}{\newline}                                            %oznaczenie końca rozwiązania
\newcommand{\odpStart}{\noindent \textbf{Odpowiedź:}\newline}    %oznaczenie początku odpowiedzi końcowej (wypisanie wyniku)
\newcommand{\odpStop}{\newline}                                             %oznaczenie końca odpowiedzi końcowej (wypisanie wyniku)
\newcommand{\testStart}{\noindent \textbf{Test:}\newline} %ewentualne możliwe opcje odpowiedzi testowej: A. ? B. ? C. ? D. ? itd.
\newcommand{\testStop}{\newline} %koniec wprowadzania odpowiedzi testowych
\newcommand{\kluczStart}{\noindent \textbf{Test poprawna odpowiedź:}\newline} %klucz, poprawna odpowiedź pytania testowego (jedna literka): A lub B lub C lub D itd.
\newcommand{\kluczStop}{\newline} %koniec poprawnej odpowiedzi pytania testowego 
\newcommand{\wstawGrafike}[2]{\begin{figure}[h] \includegraphics[scale=#2] {#1} \end{figure}} %gdyby była potrzeba wstawienia obrazka, parametry: nazwa pliku, skala (jak nie wiesz co wpisać, to wpisz 1)

\begin{document}
\maketitle


\kategoria{Dymkowska,Beger/C1.7i}
\zadStart{Zadanie z Dymkowska,Beger C 1.7 i) moja wersja nr [nrWersji]}
%[a]:[2,3,4,5,6,7,8]
%[b]:[2,3,4,5,6,7,8]
%[a]=random.randint(2,10)
%[b]=random.randint(2,20) 
%[b1]=[b]-16
%[pierw]=pow([b1],1/2)
%math.gcd([a],[b1])==1 and [pierw].is_integer()==False and [b1]>16
Obliczyć całki ułamków prostych.$$\int\frac{[a]}{x^{2}+16x+[b]}dx$$
\zadStop
\rozwStart{Mirella Narewska}{}
$$\int\frac{[a]}{x^{2}+16x+[b]}dx$$
$$[a]\int\frac{1}{x^{2}+16x+[b]}dx$$
$$[a]\int\frac{1}{(x+4)^{2}+[b1]}dx$$
$$\text{Podstawiamy: }t=x+4\Rightarrow dt=dx$$
$$[a]\int\frac{1}{(t)^{2}+[b1]}dt$$
$$[a]\int\frac{1}{(t)^{2}+(\sqrt{[b1]})^{2}}dt$$
$$\text{Korzystamy ze wzoru: }\int\frac{dx}{x^{2}+a^{2}}=\frac{1}{a}arctg(\frac{x}{a})+C$$
$$\frac{[a]}{\sqrt{[b1]}}arctg\bigg(\frac{t}{\sqrt{[b1]}}\bigg)+C=\frac{[a]}{\sqrt{[b1]}}arctg\bigg(\frac{x+4}{\sqrt{[b1]}}\bigg)+C$$
\rozwStop
\odpStart
$$\frac{[a]}{\sqrt{[b1]}}arctg\bigg(\frac{x+4}{\sqrt{[b1]}}\bigg)+C$$
\odpStop
\testStart
A.$\frac{[a]}{\sqrt{[b1]}}arctg\big(\frac{x+4}{\sqrt{[b1]}}\big)+C$
\\B.$\frac{[a]}{[b1]}tg\big(\frac{x+4}{\sqrt{[b1]}}\big)+C$
\\C.$tg\big(\frac{x-[a]}{\sqrt{[b1]}}\big)+C$
\\D.$\frac{[a]}{\sqrt{[b1]}}arctg\big(\frac{x-[b]}{\sqrt{[b1]}}\big)+C$
\testStop
\kluczStart
A
\kluczStop



\end{document}