\documentclass[12pt, a4paper]{article}
\usepackage[utf8]{inputenc}
\usepackage{polski}
\usepackage{amsthm}  %pakiet do tworzenia twierdzeń itp.
\usepackage{amsmath} %pakiet do niektórych symboli matematycznych
\usepackage{amssymb} %pakiet do symboli mat., np. \nsubseteq
\usepackage{amsfonts}
\usepackage{graphicx} %obsługa plików graficznych z rozszerzeniem png, jpg
\theoremstyle{definition} %styl dla definicji
\newtheorem{zad}{} 
\title{Multizestaw zadań}
\author{Radosław Grzyb}
%\date{\today}
\date{}
\newcounter{liczniksekcji}
\newcommand{\kategoria}[1]{\section{#1}} %olreślamy nazwę kateforii zadań
\newcommand{\zadStart}[1]{\begin{zad}#1\newline} %oznaczenie początku zadania
\newcommand{\zadStop}{\end{zad}}   %oznaczenie końca zadania
%Makra opcjonarne (nie muszą występować):
\newcommand{\rozwStart}[2]{\noindent \textbf{Rozwiązanie (autor #1 , recenzent #2): }\newline} %oznaczenie początku rozwiązania, opcjonarnie można wprowadzić informację o autorze rozwiązania zadania i recenzencie poprawności wykonania rozwiązania zadania
\newcommand{\rozwStop}{\newline}                                            %oznaczenie końca rozwiązania
\newcommand{\odpStart}{\noindent \textbf{Odpowiedź:}\newline}    %oznaczenie początku odpowiedzi końcowej (wypisanie wyniku)
\newcommand{\odpStop}{\newline}                                             %oznaczenie końca odpowiedzi końcowej (wypisanie wyniku)
\newcommand{\testStart}{\noindent \textbf{Test:}\newline} %ewentualne możliwe opcje odpowiedzi testowej: A. ? B. ? C. ? D. ? itd.
\newcommand{\testStop}{\newline} %koniec wprowadzania odpowiedzi testowych
\newcommand{\kluczStart}{\noindent \textbf{Test poprawna odpowiedź:}\newline} %klucz, poprawna odpowiedź pytania testowego (jedna literka): A lub B lub C lub D itd.
\newcommand{\kluczStop}{\newline} %koniec poprawnej odpowiedzi pytania testowego 
\newcommand{\wstawGrafike}[2]{\begin{figure}[h] \includegraphics[scale=#2] {#1} \end{figure}} %gdyby była potrzeba wstawienia obrazka, parametry: nazwa pliku, skala (jak nie wiesz co wpisać, to wpisz 1)
\begin{document}
\maketitle
\kategoria{Beger/c1.15d}
\zadStart{Zadanie z Beger C 1.15d moja wersja nr [nrWersji]}
%[p1]:[3,5,7,9,11,13,15,17,19,21]
%[dd]=[p1]-2
%[ww]=[p1]-1
Stosując odpowiednie metody całkowania obliczyć całkę:
$$\int \frac{e^{\sqrt[[p1]]{x}}}{\sqrt[[p1]]{x^{[dd]}}}\,dx$$
\zadStop
\rozwStart{Radosław Grzyb}{}
Dokonujemy podstawienia:
$$u=\sqrt[[p1]]{x}\implies du=\frac{1}{[p1]x^{\frac{[ww]}{[p1]}}}dx\implies dx=[p1]x^{\frac{[ww]}{[p1]}}du$$
Otrzymujemy:
$$\int \frac{e^{x^{\frac{1}{[p1]}}}}{x^{\frac{[dd]}{[p1]}}}\cdot[p1]x^{\frac{[ww]}{[p1]}}\,du=[p1]\int\frac{e^u\cdot u^{[ww]}}{u^{[dd]}}du=[p1]\int ue^u \,du$$
Otrzymaną całkę policzymy przez części:
$$f=u \implies f'=1$$
$$g'=e^u \implies g=e^u$$
Podstawiając do $fg'=fg-\int f'g$ otrzymujemy praktycznie finalny wynik: 
$$[p1]ue^u-[p1]\int1\cdot e^u du=[p1]ue^u-[p1]e^u+C=[p1]\sqrt[[p1]]{x}e^{\sqrt[[p1]]{x}}-[p1]e^{\sqrt[[p1]]{x}}+C$$
\rozwStop
\odpStart
$$[p1]\sqrt[[p1]]{x}e^{\sqrt[[p1]]{x}}-[p1]e^{\sqrt[[p1]]{x}}+C$$
\odpStop
\testStart
A.$$\sqrt[[p1]]{x}e^{\sqrt[[p1]]{x}}-e^{\sqrt[[p1]]{x}}+C$$
B.$$[p1]\sqrt[[p1]]{x}e^{\sqrt[[p1]]{x}}-[p1]e^{\sqrt[[p1]]{x}}+C$$
C.$$e^{\sqrt[[p1]]{x}}-[p1]e^{\sqrt[[p1]]{x}}+C$$
D.$$e^{\sqrt[[p1]]{x}}+C$$
\testStop
\kluczStart
B
\kluczStop
\end{document}