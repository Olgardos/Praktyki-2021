\documentclass[12pt, a4paper]{article}
\usepackage[utf8]{inputenc}
\usepackage{polski}

\usepackage{amsthm}  %pakiet do tworzenia twierdzeń itp.
\usepackage{amsmath} %pakiet do niektórych symboli matematycznych
\usepackage{amssymb} %pakiet do symboli mat., np. \nsubseteq
\usepackage{amsfonts}
\usepackage{graphicx} %obsługa plików graficznych z rozszerzeniem png, jpg
\theoremstyle{definition} %styl dla definicji
\newtheorem{zad}{} 
\title{Multizestaw zadań}
\author{Robert Fidytek}
%\date{\today}
\date{}
\newcounter{liczniksekcji}
\newcommand{\kategoria}[1]{\section{#1}} %olreślamy nazwę kateforii zadań
\newcommand{\zadStart}[1]{\begin{zad}#1\newline} %oznaczenie początku zadania
\newcommand{\zadStop}{\end{zad}}   %oznaczenie końca zadania
%Makra opcjonarne (nie muszą występować):
\newcommand{\rozwStart}[2]{\noindent \textbf{Rozwiązanie (autor #1 , recenzent #2): }\newline} %oznaczenie początku rozwiązania, opcjonarnie można wprowadzić informację o autorze rozwiązania zadania i recenzencie poprawności wykonania rozwiązania zadania
\newcommand{\rozwStop}{\newline}                                            %oznaczenie końca rozwiązania
\newcommand{\odpStart}{\noindent \textbf{Odpowiedź:}\newline}    %oznaczenie początku odpowiedzi końcowej (wypisanie wyniku)
\newcommand{\odpStop}{\newline}                                             %oznaczenie końca odpowiedzi końcowej (wypisanie wyniku)
\newcommand{\testStart}{\noindent \textbf{Test:}\newline} %ewentualne możliwe opcje odpowiedzi testowej: A. ? B. ? C. ? D. ? itd.
\newcommand{\testStop}{\newline} %koniec wprowadzania odpowiedzi testowych
\newcommand{\kluczStart}{\noindent \textbf{Test poprawna odpowiedź:}\newline} %klucz, poprawna odpowiedź pytania testowego (jedna literka): A lub B lub C lub D itd.
\newcommand{\kluczStop}{\newline} %koniec poprawnej odpowiedzi pytania testowego 
\newcommand{\wstawGrafike}[2]{\begin{figure}[h] \includegraphics[scale=#2] {#1} \end{figure}} %gdyby była potrzeba wstawienia obrazka, parametry: nazwa pliku, skala (jak nie wiesz co wpisać, to wpisz 1)

\begin{document}
\maketitle



\kategoria{Dymkowska,Beger/C1.15e}
\zadStart{Zadanie z Dymkowska,Beger C 1.15 e) moja wersja nr [nrWersji]}
%[a]:[2,3,4,5,6,7,8,9]
Stosując odpowiednie metody całkowania, obliczyć całkę $\displaystyle \int \frac{dx}{[a]+e^x} \ dx$
\zadStop
\rozwStart{Mirella Narewska}{}
$$\int \frac{dx}{[a]+e^x} \ dx$$
$$\text{Całkujemy przez podstawienie:} u=e^x\Rightarrow du=e^x \ dx$$
$$=\int \frac{1}{u(u+[a])} \ du= \int \left( \frac{1}{[a]u} -\frac{1}{[a](u+[a])} \right) \ du=\frac{1}{a}\int \frac{1}{u} \ du-\frac{1}{[a]}\int \frac{1}{u+[a]} \ du$$
$$=\frac{1}{[a]}\ln u -\frac{1}{[a]}\ln (u+[a]) +C=\frac{1}{[a]}\left(x-\ln (e^x+[a])\right)+C$$
\odpStart
$$\frac{1}{[a]}\left(x-\ln (e^x+[a])\right)+C$$
\odpStop
\testStart
A.$\frac{1}{[a]}\left(x-\ln (e^x+[a])\right)+C$
\\
B.$(x-\ln (e^x+[a]))+C$
\\
C.$e^{\sqrt{x}}(\sqrt{x}-1)+C$
\\
D.$e^{[a]x+[b]}+C$
\testStop
\kluczStart
A
\kluczStop


\end{document}