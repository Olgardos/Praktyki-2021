\documentclass[12pt, a4paper]{article}
\usepackage[utf8]{inputenc}
\usepackage{polski}
\usepackage{amsthm}  %pakiet do tworzenia twierdzeń itp.
\usepackage{amsmath} %pakiet do niektórych symboli matematycznych
\usepackage{amssymb} %pakiet do symboli mat., np. \nsubseteq
\usepackage{amsfonts}
\usepackage{graphicx} %obsługa plików graficznych z rozszerzeniem png, jpg
\theoremstyle{definition} %styl dla definicji
\newtheorem{zad}{} 
\title{Multizestaw zadań}
\author{Radosław Grzyb}
%\date{\today}
\date{}
\newcounter{liczniksekcji}
\newcommand{\kategoria}[1]{\section{#1}} %olreślamy nazwę kateforii zadań
\newcommand{\zadStart}[1]{\begin{zad}#1\newline} %oznaczenie początku zadania
\newcommand{\zadStop}{\end{zad}}   %oznaczenie końca zadania
%Makra opcjonarne (nie muszą występować):
\newcommand{\rozwStart}[2]{\noindent \textbf{Rozwiązanie (autor #1 , recenzent #2): }\newline} %oznaczenie początku rozwiązania, opcjonarnie można wprowadzić informację o autorze rozwiązania zadania i recenzencie poprawności wykonania rozwiązania zadania
\newcommand{\rozwStop}{\newline}                                            %oznaczenie końca rozwiązania
\newcommand{\odpStart}{\noindent \textbf{Odpowiedź:}\newline}    %oznaczenie początku odpowiedzi końcowej (wypisanie wyniku)
\newcommand{\odpStop}{\newline}                                             %oznaczenie końca odpowiedzi końcowej (wypisanie wyniku)
\newcommand{\testStart}{\noindent \textbf{Test:}\newline} %ewentualne możliwe opcje odpowiedzi testowej: A. ? B. ? C. ? D. ? itd.
\newcommand{\testStop}{\newline} %koniec wprowadzania odpowiedzi testowych
\newcommand{\kluczStart}{\noindent \textbf{Test poprawna odpowiedź:}\newline} %klucz, poprawna odpowiedź pytania testowego (jedna literka): A lub B lub C lub D itd.
\newcommand{\kluczStop}{\newline} %koniec poprawnej odpowiedzi pytania testowego 
\newcommand{\wstawGrafike}[2]{\begin{figure}[h] \includegraphics[scale=#2] {#1} \end{figure}} %gdyby była potrzeba wstawienia obrazka, parametry: nazwa pliku, skala (jak nie wiesz co wpisać, to wpisz 1)
\begin{document}
\maketitle
\kategoria{Beger/c1.12d}
\zadStart{Zadanie z Beger C 1.12d moja wersja nr [nrWersji]}
%[p1]:[6,8,10,12,14]
%[p3]:[2,3,4,5]
%[dp1]=int([p1]/2)
%[dp2]=[p1]-1
%[s1]=2*[p3]
%[s2]=[p3]**2
%[s3]=[p1]*[s1]
%[s4]=[p1]*[s2]
Obliczyć całkę funkcji niewymiernej:
$$\int \frac{\sqrt[[dp1]]{x}}{x+[p3]\sqrt[[p1]]{x^{[dp2]}}}\,dx$$
\zadStop
\rozwStart{Radosław Grzyb}{}
Najpierw przekształćmy nieco naszą funkcję podcałkową:
$$\int \frac{\sqrt[[dp1]]{x}}{x+[p3]\sqrt[[p1]]{x^{[dp2]}}}\,dx=\int \frac{x^{\frac{1}{[dp1]}}}{x^{\frac{[p1]}{[p1]}}+[p3]x^{\frac{[dp2]}{[p1]}}}\,dx=\int \frac{(x^{\frac{1}{[p1]}})^{2}}{x^{\frac{[dp2]}{[p1]}}(x^{\frac{1}{[p1]}}+[p3])}\,dx$$
Możemy teraz dokonać pierwszego podstawienia:
$$u=x^{\frac{1}{[p1]}}\implies du=\frac{1}{[p1]x^{\frac{[dp2]}{[p1]}}}dx\implies dx=[p1]x^{\frac{[dp2]}{[p1]}}du$$
A więc otrzymujemy:
$$\int \frac{(x^{\frac{1}{[p1]}})^{2}}{x^{\frac{[dp2]}{[p1]}}(x^{\frac{1}{[p1]}}+[p3])}\cdot[p1]x^{\frac{[dp2]}{[p1]}}\,du=
[p1]\int\frac{u^2}{u+[p3]} \,du$$
Dokonujemy kolejnego podstawienia:
$$t=u+[p3]\implies dt=du \implies u=t-[p3]$$
Otrzymujemy wówczas:
$$[p1]\int\frac{(t-[p3])^2}{t} \,dt=[p1]\int\frac{t^{2}-[s1]t+[s2]}{t} \,dt=\int[p1]t-[s3]+\frac{[s4]}{t} \,dt$$
Obliczamy tę całkę i otrzymujemy wynik:
$$\int[p1]t-[s3]+\frac{[s4]}{t} \,dt=\frac{[p1]}{2}t^{2}-[s3]t+[s4]\ln|t|+C=$$
$$=[dp1](u+[p3])^{2}-[s3](u+[p3])+[s4]\ln|(u+[p3])|+C=$$
$$=[dp1](x^{\frac{1}{[p1]}}+[p3])^{2}-[s3](x^{\frac{1}{[p1]}}+[p3])+[s4]\ln|(x^{\frac{1}{[p1]}}+[p3])|+C$$
\rozwStop
\odpStart
$$=[dp1](x^{\frac{1}{[p1]}}+[p3])^{2}-[s3](x^{\frac{1}{[p1]}}+[p3])+[s4]\ln|(x^{\frac{1}{[p1]}}+[p3])|+C$$
\odpStop
\testStart
A.$$-\frac{1}{[p3]\arctan^{[s1]}(x)}+C$$
B.$$=[dp1](x^{\frac{1}{[p1]}}+[p3])^{2}-[s3](x^{\frac{1}{[p1]}}+[p3])+[s4]\ln|(x^{\frac{1}{[p1]}}+[p3])|+C$$
C.$$=[dp1](x^{\frac{1}{[p1]}}+[p3])^{3}-[s3](x^{\frac{1}{[p1]}}+[p3])+[s4]\ln|(x^{\frac{1}{[p1]}}+[p3])|+C$$
D.$$\frac{1}{[s1]}\cot|\frac{x}{[p3]}|+C$$
\testStop
\kluczStart
B
\kluczStop
\end{document}