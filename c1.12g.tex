\documentclass[12pt, a4paper]{article}
\usepackage[utf8]{inputenc}
\usepackage{polski}
\usepackage{amsthm}  %pakiet do tworzenia twierdzeń itp.
\usepackage{amsmath} %pakiet do niektórych symboli matematycznych
\usepackage{amssymb} %pakiet do symboli mat., np. \nsubseteq
\usepackage{amsfonts}
\usepackage{graphicx} %obsługa plików graficznych z rozszerzeniem png, jpg
\theoremstyle{definition} %styl dla definicji
\newtheorem{zad}{} 
\title{Multizestaw zadań}
\author{Radosław Grzyb}
%\date{\today}
\date{}
\newcounter{liczniksekcji}
\newcommand{\kategoria}[1]{\section{#1}} %olreślamy nazwę kateforii zadań
\newcommand{\zadStart}[1]{\begin{zad}#1\newline} %oznaczenie początku zadania
\newcommand{\zadStop}{\end{zad}}   %oznaczenie końca zadania
%Makra opcjonarne (nie muszą występować):
\newcommand{\rozwStart}[2]{\noindent \textbf{Rozwiązanie (autor #1 , recenzent #2): }\newline} %oznaczenie początku rozwiązania, opcjonarnie można wprowadzić informację o autorze rozwiązania zadania i recenzencie poprawności wykonania rozwiązania zadania
\newcommand{\rozwStop}{\newline}                                            %oznaczenie końca rozwiązania
\newcommand{\odpStart}{\noindent \textbf{Odpowiedź:}\newline}    %oznaczenie początku odpowiedzi końcowej (wypisanie wyniku)
\newcommand{\odpStop}{\newline}                                             %oznaczenie końca odpowiedzi końcowej (wypisanie wyniku)
\newcommand{\testStart}{\noindent \textbf{Test:}\newline} %ewentualne możliwe opcje odpowiedzi testowej: A. ? B. ? C. ? D. ? itd.
\newcommand{\testStop}{\newline} %koniec wprowadzania odpowiedzi testowych
\newcommand{\kluczStart}{\noindent \textbf{Test poprawna odpowiedź:}\newline} %klucz, poprawna odpowiedź pytania testowego (jedna literka): A lub B lub C lub D itd.
\newcommand{\kluczStop}{\newline} %koniec poprawnej odpowiedzi pytania testowego 
\newcommand{\wstawGrafike}[2]{\begin{figure}[h] \includegraphics[scale=#2] {#1} \end{figure}} %gdyby była potrzeba wstawienia obrazka, parametry: nazwa pliku, skala (jak nie wiesz co wpisać, to wpisz 1)
\begin{document}
\maketitle
\kategoria{Beger/c1.12g}
\zadStart{Zadanie z Beger C 1.12g moja wersja nr [nrWersji]}
%[p1]:[1,2,3,4,5,6,7,8,9]
%[p2]:[2,3,4,5,6,7]
%[ab1]=2*[p2]
%[ab2]=[p2]**2
%[aab1]=4*[ab1]
%[aab2]=4*[ab2]
Obliczyć całkę funkcji niewymiernej:
$$\int \frac{1}{\sqrt{x-[p1]}+[p2]\sqrt[4]{x-[p1]}}\,dx$$
\zadStop
\rozwStart{Radosław Grzyb}{}
Dokonujemy podstawienia:
$$u=x-[p1]\implies du=dx$$
Otrzymujemy:
$$\int \frac{1}{\sqrt{u}+[p2]\sqrt[4]{u}}\,du$$
Dokonujemy kolejnego podstawienia:
$$t=\sqrt[4]{u} \implies dt=\frac{1}{4}u^{-\frac{3}{4}} du\implies du=4u^{\frac{3}{4}}dt$$
Otrzymujemy: 
$$\int \frac{1}{\sqrt{u}+[p2]\sqrt[4]{u}}\cdot4u^{\frac{3}{4}}\,dt=4\int \frac{u^{\frac{3}{4}}}{u^{\frac{2}{4}}+[p2]u^{\frac{1}{4}}}\,dt=4\int\frac{t^3}{t^2+[p2]t}\,dt=4\int\frac{t^3}{t(t+[p2])}\,dt=4\int\frac{t^2}{t+[p2]}\,dt$$
Ponownie dokonamy podstawienia!:
$$w=t+[p2] \implies dw=dt$$
$$t=w-[p2]$$
Otrzymujemy do policzenia całkę:
$$4\int\frac{t^2}{t+[p2]}\,dt=4\int\frac{(w-[p2])^2}{w}\,dw=4\int\frac{w^2-[ab1]w+[ab2]}{w}\,dw=$$
$$=4\int w\,dw-4\int [ab1]\,dw +4\int\frac{[ab2]}{w}\,dw=4\int w\,dw-\int [aab1]\,dw +[aab2]\int\frac{1}{w}\,dw=$$
$$=2w^{2}-[aab1]w+[aab2]\ln|w|+C=2(t+[p2])^{2}-[aab1](t+[p2])+[aab2]\ln|t+[p2]|+C=$$
$$=2(\sqrt[4]{u}+[p2])^{2}-[aab1](\sqrt[4]{u}+[p2])+[aab2]\ln|\sqrt[4]{u}+[p2]|+C$$\\
Po finalnym podstawieniu otrzymujemy odpowiedź:
$$2(\sqrt[4]{x-[p1]}+[p2])^{2}-[aab1](\sqrt[4]{x-[p1]}+[p2])+[aab2]\ln|\sqrt[4]{x-[p1]}+[p2]|+C$$
\rozwStop
\odpStart
$$2(\sqrt[4]{x-[p1]}+[p2])^{2}-[aab1](\sqrt[4]{x-[p1]}+[p2])+[aab2]\ln|\sqrt[4]{x-[p1]}+[p2]|+C$$
\odpStop
\testStart
A.$$2(\sqrt[4]{x-[p1]}+[p2])^{2}-[aab1](\sqrt[4]{x-[p1]}+[p2])+[aab2]\ln|x-[p1]|+C$$
B.$$2(\sqrt[3]{x-[p1]}+[p2])^{2}-[aab1](\sqrt[4]{x-[p1]}+[p2])+[aab2]\ln|\sqrt[3]{x-[p1]}+[p2]|+C$$
C.$$2(\sqrt[4]{x-[p1]}+[p2])^{2}-[aab1](\sqrt[4]{x-[p1]}+[p2])+[aab2]\ln|\sqrt[4]{x-[p1]}+[p2]|+C$$
D.$$2(\sqrt[4]{x-[p1]}+[p2])^{3}+C$$
\testStop
\kluczStart
C
\kluczStop
\end{document}