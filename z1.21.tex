\documentclass[12pt, a4paper]{article}
\usepackage[utf8]{inputenc}
\usepackage{polski}

\usepackage{amsthm}  %pakiet do tworzenia twierdzeń itp.
\usepackage{amsmath} %pakiet do niektórych symboli matematycznych
\usepackage{amssymb} %pakiet do symboli mat., np. \nsubseteq
\usepackage{amsfonts}
\usepackage{graphicx} %obsługa plików graficznych z rozszerzeniem png, jpg
\theoremstyle{definition} %styl dla definicji
\newtheorem{zad}{} 
\title{Multizestaw zadań}
\author{Robert Fidytek}
%\date{\today}
\date{}
\newcounter{liczniksekcji}
\newcommand{\kategoria}[1]{\section{#1}} %olreślamy nazwę kateforii zadań
\newcommand{\zadStart}[1]{\begin{zad}#1\newline} %oznaczenie początku zadania
\newcommand{\zadStop}{\end{zad}}   %oznaczenie końca zadania
%Makra opcjonarne (nie muszą występować):
\newcommand{\rozwStart}[2]{\noindent \textbf{Rozwiązanie (autor #1 , recenzent #2): }\newline} %oznaczenie początku rozwiązania, opcjonarnie można wprowadzić informację o autorze rozwiązania zadania i recenzencie poprawności wykonania rozwiązania zadania
\newcommand{\rozwStop}{\newline}                                            %oznaczenie końca rozwiązania
\newcommand{\odpStart}{\noindent \textbf{Odpowiedź:}\newline}    %oznaczenie początku odpowiedzi końcowej (wypisanie wyniku)
\newcommand{\odpStop}{\newline}                                             %oznaczenie końca odpowiedzi końcowej (wypisanie wyniku)
\newcommand{\testStart}{\noindent \textbf{Test:}\newline} %ewentualne możliwe opcje odpowiedzi testowej: A. ? B. ? C. ? D. ? itd.
\newcommand{\testStop}{\newline} %koniec wprowadzania odpowiedzi testowych
\newcommand{\kluczStart}{\noindent \textbf{Test poprawna odpowiedź:}\newline} %klucz, poprawna odpowiedź pytania testowego (jedna literka): A lub B lub C lub D itd.
\newcommand{\kluczStop}{\newline} %koniec poprawnej odpowiedzi pytania testowego 
\newcommand{\wstawGrafike}[2]{\begin{figure}[h] \includegraphics[scale=#2] {#1} \end{figure}} %gdyby była potrzeba wstawienia obrazka, parametry: nazwa pliku, skala (jak nie wiesz co wpisać, to wpisz 1)

\begin{document}
\maketitle


\kategoria{Wikieł/Z1.21}
\zadStart{Zadanie z Wikieł Z 1.21  moja wersja nr [nrWersji]}
%[a]:[2,3,4,5,6,7,8]
%[b]:[1,2,3]
%[n]:[8,10,12,14]
%[dn]=int((math.perm([n],5)/math.perm(5,4)))
%[ap]=[a]*([n]-5)
%[b5]=(-[b])**(5)
%[dnb5]=[dn]*[b5]
%[absdnb5]=abs([dnb5])
%[dnblad]=int((math.perm([n],6)/math.perm(6,5)))
%[apblad]=[a]*([n]-6)
%[dnb5blad]=[dnblad]*[b5]
%[absdnb5blad]=abs([dnb5blad])
Obliczyć wyraz szósty rozwinięcia dwumianu $(x^{[a]}-[b])^{[n]}$.
\zadStop
\rozwStart{Wojciech Przybylski}{Maja Szabłowska}
Wzór na k-ty wyraz:
$$A_{k}={n\choose k-1}\cdot a^{n-k+1}\cdot b^{k-1}$$
$$A_{6}={[n]\choose 6}\cdot \big(x^{[a]}\big)^{[n]-6+1}\cdot (-[b])^{6-1}=[dn]\cdot x^{[ap]}\cdot ([b5])^{5}=$$
$$=[dnb5]\cdot x^{[ap]}$$
\rozwStop
\odpStart
$[dnb5]\cdot x^{[ap]}$.
\odpStop
\testStart
A. $[dnb5]\cdot x^{[ap]}$
B. $[absdnb5]\cdot x^{[ap]}$
C. $[absdnb5blad]\cdot x^{[apblad]}$
D. $[dnb5blad]\cdot x^{[apblad]}$
E. $0$
F. $1$
\testStop
\kluczStart
A
\kluczStop



\end{document}