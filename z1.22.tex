\documentclass[12pt, a4paper]{article}
\usepackage[utf8]{inputenc}
\usepackage{polski}

\usepackage{amsthm}  %pakiet do tworzenia twierdzeń itp.
\usepackage{amsmath} %pakiet do niektórych symboli matematycznych
\usepackage{amssymb} %pakiet do symboli mat., np. \nsubseteq
\usepackage{amsfonts}
\usepackage{graphicx} %obsługa plików graficznych z rozszerzeniem png, jpg
\theoremstyle{definition} %styl dla definicji
\newtheorem{zad}{} 
\title{Multizestaw zadań}
\author{Robert Fidytek}
%\date{\today}
\date{}
\newcounter{liczniksekcji}
\newcommand{\kategoria}[1]{\section{#1}} %olreślamy nazwę kateforii zadań
\newcommand{\zadStart}[1]{\begin{zad}#1\newline} %oznaczenie początku zadania
\newcommand{\zadStop}{\end{zad}}   %oznaczenie końca zadania
%Makra opcjonarne (nie muszą występować):
\newcommand{\rozwStart}[2]{\noindent \textbf{Rozwiązanie (autor #1 , recenzent #2): }\newline} %oznaczenie początku rozwiązania, opcjonarnie można wprowadzić informację o autorze rozwiązania zadania i recenzencie poprawności wykonania rozwiązania zadania
\newcommand{\rozwStop}{\newline}                                            %oznaczenie końca rozwiązania
\newcommand{\odpStart}{\noindent \textbf{Odpowiedź:}\newline}    %oznaczenie początku odpowiedzi końcowej (wypisanie wyniku)
\newcommand{\odpStop}{\newline}                                             %oznaczenie końca odpowiedzi końcowej (wypisanie wyniku)
\newcommand{\testStart}{\noindent \textbf{Test:}\newline} %ewentualne możliwe opcje odpowiedzi testowej: A. ? B. ? C. ? D. ? itd.
\newcommand{\testStop}{\newline} %koniec wprowadzania odpowiedzi testowych
\newcommand{\kluczStart}{\noindent \textbf{Test poprawna odpowiedź:}\newline} %klucz, poprawna odpowiedź pytania testowego (jedna literka): A lub B lub C lub D itd.
\newcommand{\kluczStop}{\newline} %koniec poprawnej odpowiedzi pytania testowego 
\newcommand{\wstawGrafike}[2]{\begin{figure}[h] \includegraphics[scale=#2] {#1} \end{figure}} %gdyby była potrzeba wstawienia obrazka, parametry: nazwa pliku, skala (jak nie wiesz co wpisać, to wpisz 1)

\begin{document}
\maketitle


\kategoria{Wikieł/Z1.22}
\zadStart{Zadanie z Wikieł Z 1.22  moja wersja nr [nrWersji]}
%[a]:[444,646,686,868,888]
%[b]:[2,3,5,7,11,13]
%[k1]=int([a]/2)
%[k]=int([a]/2+1)
%[ak1]=[a]-[k]+1
%[c]=int(([k1]/2+[ak1])/3)
Obliczyć środkowy wyraz rozwinięcia dwumianu $\bigg(\sqrt[3]{[b]}+\sqrt[6]{[b]}\bigg)^{[a]}$.
\zadStop
\rozwStart{Wojciech Przybylski}{Maja Szabłowska}
Wzór na k-ty wyraz:
$$A_{k}={n\choose k-1}\cdot a^{n-k+1}\cdot b^{k-1}$$
$$A_{[k]}={[a]\choose [k1]}\cdot \big([b]^{\frac{1}{3}}\big)^{[a]-[k]+1}\cdot \big([b]^{\frac{1}{6}}\big)^{[k]-1}=$$
$$={[a]\choose [k1]}\cdot [b]^{[c]}$$
\rozwStop
\odpStart
${[a]\choose [k1]}\cdot [b]^{[c]}$
\odpStop
\testStart
A. ${[a]\choose [k1]}\cdot [b]^{[c]}$
B. ${[a]\choose [k]}\cdot [b]^{[c]}$
C. ${[a]\choose [k]}\cdot [b]^{[ak1]}$
D. ${[a]\choose [k1]}\cdot [b]^{[ak1]}$
E. $0$
F. $1$
\testStop
\kluczStart
A
\kluczStop



\end{document}