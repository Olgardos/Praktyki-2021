\documentclass[12pt, a4paper]{article}
\usepackage[utf8]{inputenc}
\usepackage{polski}

\usepackage{amsthm}  %pakiet do tworzenia twierdzeń itp.
\usepackage{amsmath} %pakiet do niektórych symboli matematycznych
\usepackage{amssymb} %pakiet do symboli mat., np. \nsubseteq
\usepackage{amsfonts}
\usepackage{graphicx} %obsługa plików graficznych z rozszerzeniem png, jpg
\theoremstyle{definition} %styl dla definicji
\newtheorem{zad}{} 
\title{Multizestaw zadań}
\author{Robert Fidytek}
%\date{\today}
\date{}
\newcounter{liczniksekcji}
\newcommand{\kategoria}[1]{\section{#1}} %olreślamy nazwę kateforii zadań
\newcommand{\zadStart}[1]{\begin{zad}#1\newline} %oznaczenie początku zadania
\newcommand{\zadStop}{\end{zad}}   %oznaczenie końca zadania
%Makra opcjonarne (nie muszą występować):
\newcommand{\rozwStart}[2]{\noindent \textbf{Rozwiązanie (autor #1 , recenzent #2): }\newline} %oznaczenie początku rozwiązania, opcjonarnie można wprowadzić informację o autorze rozwiązania zadania i recenzencie poprawności wykonania rozwiązania zadania
\newcommand{\rozwStop}{\newline}                                            %oznaczenie końca rozwiązania
\newcommand{\odpStart}{\noindent \textbf{Odpowiedź:}\newline}    %oznaczenie początku odpowiedzi końcowej (wypisanie wyniku)
\newcommand{\odpStop}{\newline}                                             %oznaczenie końca odpowiedzi końcowej (wypisanie wyniku)
\newcommand{\testStart}{\noindent \textbf{Test:}\newline} %ewentualne możliwe opcje odpowiedzi testowej: A. ? B. ? C. ? D. ? itd.
\newcommand{\testStop}{\newline} %koniec wprowadzania odpowiedzi testowych
\newcommand{\kluczStart}{\noindent \textbf{Test poprawna odpowiedź:}\newline} %klucz, poprawna odpowiedź pytania testowego (jedna literka): A lub B lub C lub D itd.
\newcommand{\kluczStop}{\newline} %koniec poprawnej odpowiedzi pytania testowego 
\newcommand{\wstawGrafike}[2]{\begin{figure}[h] \includegraphics[scale=#2] {#1} \end{figure}} %gdyby była potrzeba wstawienia obrazka, parametry: nazwa pliku, skala (jak nie wiesz co wpisać, to wpisz 1)

\begin{document}
\maketitle



\kategoria{Wikieł/P3.28c}
\zadStart{Zadanie z Wikieł P 3.28 c) moja wersja nr [nrWersji]}
%[a]:[2,3,4,5,6,7,8,9,10,11,12,13,14,15,16,17,18,19]
%[b]:[2,3,4,5,6,7,8,9,10,11,12,13,14,15,16,17,18,19]
%[b1]=[b]-1
%[aa]=[a]*[a]
%[bb]=[b]*[b]
%[ab2]=[a]*[b]*2
%[a]!=[b] 
Obliczyć granicę ciągu $a_{n}=\frac{1}{([a]n+[b])^3}+\frac{2}{([a]n+[b])^3}+\frac{3}{([a]n+[b])^3}+\cdots+\frac{[a]n+[b1]}{([a]n+[b])^3}$.
\zadStop
\rozwStart{Robert Fidytek}{}
$$\lim\limits_{n\to\infty}\left(\frac{1}{([a]n+[b])^3}+\frac{2}{([a]n+[b])^3}+\frac{3}{([a]n+[b])^3}+\cdots+\frac{[a]n+[b1]}{([a]n+[b])^3}\right)=$$ 
$$=\lim\limits_{n\to\infty}\left(\frac{1+2+3+\cdots+([a]n+[b1])}{([a]n+[b])^3}\right)=$$ 
$$=\lim\limits_{n\to\infty}\left(\frac{\frac{(1+([a]n+[b1]))([a]n+[b1])}{2}}{([a]n+[b])^3}\right)=$$ 
$$=\lim\limits_{n\to\infty}\frac{([a]n+[b])([a]n+[b1])}{2([a]n+[b])([a]n+[b])^2}=$$ 
$$=\lim\limits_{n\to\infty}\frac{[a]n+[b1]}{2([a]n+[b])^2}=$$ 
$$=\lim\limits_{n\to\infty}\frac{[a]n+[b1]}{2\left(([a]n)^2+2\cdot[a]\cdot[b]n+[b]^2\right)}=$$ 
$$=\lim\limits_{n\to\infty}\frac{[a]n+[b1]}{2\left([aa]n^2+[ab2]n+[bb]\right)}=$$ 
$$=\lim\limits_{n\to\infty}\frac{n\left([a]+\frac{[b1]}{n}\right)}{2n^2\left([aa]+\frac{[ab2]}{n}+\frac{[bb]}{n^2}\right)}=$$ 
$$=\lim\limits_{n\to\infty}\frac{[a]+\frac{[b1]}{n}}{2n\left([aa]+\frac{[ab2]}{n}+\frac{[bb]}{n^2}\right)}=$$
$$=\left[\frac{[a]}{2\cdot\infty\cdot [aa]}\right]=0$$ 
\rozwStop
\odpStart
$0$
\odpStop
\testStart
A.$0$
B.$-\infty$
C.$\infty$
D.$\frac{[a]}{[b]}$
E.$1$
F.$[b]$
G.$e^{[a]}$
H.$\frac{[b]}{[a]}$
I.$[a]$
\testStop
\kluczStart
A
\kluczStop


\end{document}