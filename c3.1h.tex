\documentclass[12pt, a4paper]{article}
\usepackage[utf8]{inputenc}
\usepackage{polski}

\usepackage{amsthm}  %pakiet do tworzenia twierdzeń itp.
\usepackage{amsmath} %pakiet do niektórych symboli matematycznych
\usepackage{amssymb} %pakiet do symboli mat., np. \nsubseteq
\usepackage{amsfonts}
\usepackage{graphicx} %obsługa plików graficznych z rozszerzeniem png, jpg
\theoremstyle{definition} %styl dla definicji
\newtheorem{zad}{} 
\title{Multizestaw zadań}
\author{Robert Fidytek}
%\date{\today}
\date{}
\newcounter{liczniksekcji}
\newcommand{\kategoria}[1]{\section{#1}} %olreślamy nazwę kateforii zadań
\newcommand{\zadStart}[1]{\begin{zad}#1\newline} %oznaczenie początku zadania
\newcommand{\zadStop}{\end{zad}}   %oznaczenie końca zadania
%Makra opcjonarne (nie muszą występować):
\newcommand{\rozwStart}[2]{\noindent \textbf{Rozwiązanie (autor #1 , recenzent #2): }\newline} %oznaczenie początku rozwiązania, opcjonarnie można wprowadzić informację o autorze rozwiązania zadania i recenzencie poprawności wykonania rozwiązania zadania
\newcommand{\rozwStop}{\newline}                                            %oznaczenie końca rozwiązania
\newcommand{\odpStart}{\noindent \textbf{Odpowiedź:}\newline}    %oznaczenie początku odpowiedzi końcowej (wypisanie wyniku)
\newcommand{\odpStop}{\newline}                                             %oznaczenie końca odpowiedzi końcowej (wypisanie wyniku)
\newcommand{\testStart}{\noindent \textbf{Test:}\newline} %ewentualne możliwe opcje odpowiedzi testowej: A. ? B. ? C. ? D. ? itd.
\newcommand{\testStop}{\newline} %koniec wprowadzania odpowiedzi testowych
\newcommand{\kluczStart}{\noindent \textbf{Test poprawna odpowiedź:}\newline} %klucz, poprawna odpowiedź pytania testowego (jedna literka): A lub B lub C lub D itd.
\newcommand{\kluczStop}{\newline} %koniec poprawnej odpowiedzi pytania testowego 
\newcommand{\wstawGrafike}[2]{\begin{figure}[h] \includegraphics[scale=#2] {#1} \end{figure}} %gdyby była potrzeba wstawienia obrazka, parametry: nazwa pliku, skala (jak nie wiesz co wpisać, to wpisz 1)

\begin{document}
\maketitle


\kategoria{Dymkowska, Beger/c3.1h}
\zadStart{Zadanie z Dymkowskiej, Beger C 3.1h) moja wersja nr [nrWersji]}
%[p1]:[2,3,4,5,6,7,8,9,10]
%[p2]:[2,3,4,5,6,7,8,9,10]
%[y1]=[p2]+1
%[d1]=2*([p1]+[y1])
%[d2]=2*[y1]
%[d3]=2*([p1]+1)
%[d4]=2
%[up]=[y1]*([p1]+1)-([p1]+[y1])*([p1]+1)-([p1]+[y1])*[y1]+([p1]+[y1])*[y1]*([p1]+1)
%[down]=([p1]+[y1])*[y1]*([p1]+1)*2
%[g]=math.gcd(abs([up]),[down])
%[wyn1]=int([up]/[g])
%[wyn2]=int([down]/[g])
%[wyn2]>1
Obliczyć całkę podwójną po prostokącie P $$\iint_P \frac{dxdy}{(x+y+1)^3}, P: 0\leq x \leq [p1], 0 \leq y \leq [p2]$$
\zadStop
\rozwStart{Jakub Janik}{}
$$\iint_P \frac{dxdy}{(x+y+1)^3}=\int_0^{[p1]}dx\int_0^{[p2]} \frac{1}{(x+y+1)^3}\ dy=$$
$$=\int_0^{[p1]} (-\frac{1}{2(x+y+1)^2}\Big|_0^{[p2]})\ dx=\int_0^{[p1]}-\frac{1}{2(x+[y1])^2}+\frac{1}{2(x+1)^2}\ dx=$$
$$=\frac{1}{2(x+[y1])}-\frac{1}{2(x+1)}\Big|_0^{[p1]}=\frac{1}{[d1]}-\frac{1}{[d2]}-\frac{1}{[d3]}+\frac{1}{[d4]}=\frac{[wyn1]}{[wyn2]}$$
\rozwStop
\odpStart
$\frac{[wyn1]}{[wyn2]}$
\odpStop
\testStart
A.$\frac{[wyn1]}{[wyn2]}$
B.$0$
C.$-\frac{[wyn1]}{[wyn2]}$
D.$\infty$
\testStop
\kluczStart
A
\kluczStop



\end{document}