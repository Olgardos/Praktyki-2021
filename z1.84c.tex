\documentclass[12pt, a4paper]{article}
\usepackage[utf8]{inputenc}
\usepackage{polski}
\usepackage{amsthm}  %pakiet do tworzenia twierdzeń itp.
\usepackage{amsmath} %pakiet do niektórych symboli matematycznych
\usepackage{amssymb} %pakiet do symboli mat., np. \nsubseteq
\usepackage{amsfonts}
\usepackage{graphicx} %obsługa plików graficznych z rozszerzeniem png, jpg
\theoremstyle{definition} %styl dla definicji
\newtheorem{zad}{} 
\title{Multizestaw zadań}
\author{Radosław Grzyb}
%\date{\today}
\date{}
\newcounter{liczniksekcji}
\newcommand{\kategoria}[1]{\section{#1}} %olreślamy nazwę kateforii zadań
\newcommand{\zadStart}[1]{\begin{zad}#1\newline} %oznaczenie początku zadania
\newcommand{\zadStop}{\end{zad}}   %oznaczenie końca zadania
%Makra opcjonarne (nie muszą występować):
\newcommand{\rozwStart}[2]{\noindent \textbf{Rozwiązanie (autor #1 , recenzent #2): }\newline} %oznaczenie początku rozwiązania, opcjonarnie można wprowadzić informację o autorze rozwiązania zadania i recenzencie poprawności wykonania rozwiązania zadania
\newcommand{\rozwStop}{\newline}                                            %oznaczenie końca rozwiązania
\newcommand{\odpStart}{\noindent \textbf{Odpowiedź:}\newline}    %oznaczenie początku odpowiedzi końcowej (wypisanie wyniku)
\newcommand{\odpStop}{\newline}                                             %oznaczenie końca odpowiedzi końcowej (wypisanie wyniku)
\newcommand{\testStart}{\noindent \textbf{Test:}\newline} %ewentualne możliwe opcje odpowiedzi testowej: A. ? B. ? C. ? D. ? itd.
\newcommand{\testStop}{\newline} %koniec wprowadzania odpowiedzi testowych
\newcommand{\kluczStart}{\noindent \textbf{Test poprawna odpowiedź:}\newline} %klucz, poprawna odpowiedź pytania testowego (jedna literka): A lub B lub C lub D itd.
\newcommand{\kluczStop}{\newline} %koniec poprawnej odpowiedzi pytania testowego 
\newcommand{\wstawGrafike}[2]{\begin{figure}[h] \includegraphics[scale=#2] {#1} \end{figure}} %gdyby była potrzeba wstawienia obrazka, parametry: nazwa pliku, skala (jak nie wiesz co wpisać, to wpisz 1)
\begin{document}
\maketitle
\kategoria{Wikieł/Z1.84c}
\zadStart{Zadanie z Wikieł Z 1.84c moja wersja nr [nrWersji]}
%[p1]:[1,2]
%[p2]:[2,4,5,7,8,10,11,13,14]
%[c1]=4*[p2]
%[c2]=3-[c1]
%[c3]=[c2]/10
%[wynik]=2/[p1]
%[wynikzly1]=int(1-[wynik])
%[wynikzly2]=int(2*[wynik]-3)
%[wynikzly3]=int(5*[wynik]-12)
%[lynik]=int([wynik])
%([wynik]).is_integer() is True
Rozwiązać równanie:
$$\frac{3}{10}\cdot\left(\frac{3}{[p2]}\right)^{[p1]x-2}=\frac{6}{5}\cdot\left(\frac{3}{[p2]}\right)^{[p1]x-3}[c3]$$.
\zadStop
\rozwStart{Radosław Grzyb}{}
Mnożymy obie strony równania przez $10$ i otrzymujemy:\\
$$3\cdot\left(\frac{3}{[p2]}\right)^{[p1]x-2}=12\cdot\left(\frac{3}{[p2]}\right)^{[p1]x-3}[c2]$$
$$3\cdot\left(\frac{3}{[p2]}\right)^{[p1]x-2}-12\cdot\left(\frac{3}{[p2]}\right)^{[p1]x-3}=[c2]$$
$$3\cdot\left(\frac{3}{[p2]}\right)^{[p1]x-2}-12\cdot\left(\frac{3}{[p2]}\right)^{[p1]x-2}\cdot\frac{[p2]}{3}=[c2]$$
$$3\cdot\left(\frac{3}{[p2]}\right)^{[p1]x-2}-[c1]\cdot\left(\frac{3}{[p2]}\right)^{[p1]x-2}=[c2]$$
$$[c2]\cdot\left(\frac{3}{[p2]}\right)^{[p1]x-2}=[c2]$$.
$$\left(\frac{3}{[p2]}\right)^{[p1]x-2}=1$$.
Żeby nasza funkcja była równa 1, jej wykładnik musi być równy 0, a więc:\\
$$[p1]x-2=0\implies [p1]x=2\implies x=[lynik]$$
\rozwStop
\odpStart
$$[lynik]$$
\odpStop
\testStart
A.$$[wynikzly1]$$
B.$$[wynikzly2]$$
C.$$[lynik]$$
D.$$[wynikzly3]$$
\testStop
\kluczStart
C
\kluczStop
\end{document}