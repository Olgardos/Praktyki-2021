\documentclass[12pt, a4paper]{article}
\usepackage[utf8]{inputenc}
\usepackage{polski}

\usepackage{amsthm}  %pakiet do tworzenia twierdzeń itp.
\usepackage{amsmath} %pakiet do niektórych symboli matematycznych
\usepackage{amssymb} %pakiet do symboli mat., np. \nsubseteq
\usepackage{amsfonts}
\usepackage{graphicx} %obsługa plików graficznych z rozszerzeniem png, jpg
\theoremstyle{definition} %styl dla definicji
\newtheorem{zad}{} 
\title{Multizestaw zadań}
\author{Robert Fidytek}
%\date{\today}
\date{}
\newcounter{liczniksekcji}
\newcommand{\kategoria}[1]{\section{#1}} %olreślamy nazwę kateforii zadań
\newcommand{\zadStart}[1]{\begin{zad}#1\newline} %oznaczenie początku zadania
\newcommand{\zadStop}{\end{zad}}   %oznaczenie końca zadania
%Makra opcjonarne (nie muszą występować):
\newcommand{\rozwStart}[2]{\noindent \textbf{Rozwiązanie (autor #1 , recenzent #2): }\newline} %oznaczenie początku rozwiązania, opcjonarnie można wprowadzić informację o autorze rozwiązania zadania i recenzencie poprawności wykonania rozwiązania zadania
\newcommand{\rozwStop}{\newline}                                            %oznaczenie końca rozwiązania
\newcommand{\odpStart}{\noindent \textbf{Odpowiedź:}\newline}    %oznaczenie początku odpowiedzi końcowej (wypisanie wyniku)
\newcommand{\odpStop}{\newline}                                             %oznaczenie końca odpowiedzi końcowej (wypisanie wyniku)
\newcommand{\testStart}{\noindent \textbf{Test:}\newline} %ewentualne możliwe opcje odpowiedzi testowej: A. ? B. ? C. ? D. ? itd.
\newcommand{\testStop}{\newline} %koniec wprowadzania odpowiedzi testowych
\newcommand{\kluczStart}{\noindent \textbf{Test poprawna odpowiedź:}\newline} %klucz, poprawna odpowiedź pytania testowego (jedna literka): A lub B lub C lub D itd.
\newcommand{\kluczStop}{\newline} %koniec poprawnej odpowiedzi pytania testowego 
\newcommand{\wstawGrafike}[2]{\begin{figure}[h] \includegraphics[scale=#2] {#1} \end{figure}} %gdyby była potrzeba wstawienia obrazka, parametry: nazwa pliku, skala (jak nie wiesz co wpisać, to wpisz 1)

\begin{document}
\maketitle


\kategoria{Dymkowska, Beger/c3.9a}
\zadStart{Zadanie z Dymkowskiej, Beger C 3.9a) moja wersja nr [nrWersji]}
%[p1]:[2,4,6,8,10]
%[p2]=[p1]*[p1]
%[p3]=int([p2]*[p2]/4)
%[wyn]=2*[p3]
Wprowadzając współrzędne biegunowe, obliczyć całkę podwójną po obszarze $$\iint_D x^2+y^2 dxdy, D: x^2+y^2 \leq [p2]$$
\zadStop
\rozwStart{Jakub Janik}{}
Wprowadzamy współrzędne biegunowe, czyli $$x(r,\phi)=r\cdot\cos{(\phi)}$$
$$y(r,\phi)=r\cdot\sin{(\phi)}$$
Wtedy obszar D wyraża się następująco $$D: 0 \leq r \leq [p1], 0 \leq \phi \leq 2\pi$$
Przechodzimy do obliczenia całki
$$\iint_D r^3\ drd\phi = \int_0^{2\pi}d\phi\int_0^{[p1]}r^3\ dr=\int_0^{2\pi}(\frac{1}{4}r^4\Big|_0^{[p1]})d\phi$$
$$=\int_0^{2\pi}[p3]\ d\phi=[p3]\phi\Big|_0^{2\pi}=[wyn]\pi$$
\rozwStop
\odpStart
$[wyn]\pi$
\odpStop
\testStart
A.$[wyn]\pi$
B.$0$
C.$-[wyn]\pi$
D.$\infty$
\testStop
\kluczStart
A
\kluczStop



\end{document}