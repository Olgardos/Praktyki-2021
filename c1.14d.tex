\documentclass[7pt, a4paper]{article}
\usepackage[utf8]{inputenc}
\usepackage{polski}
\usepackage{amsthm}  %pakiet do tworzenia twierdzeń itp.
\usepackage{amsmath} %pakiet do niektórych symboli matematycznych
\usepackage{amssymb} %pakiet do symboli mat., np. \nsubseteq
\usepackage{amsfonts}
\usepackage{graphicx} %obsługa plików graficznych z rozszerzeniem png, jpg
\theoremstyle{definition} %styl dla definicji
\newtheorem{zad}{} 
\title{Multizestaw zadań}
\author{Radosław Grzyb}
%\date{\today}
\date{}
\newcounter{liczniksekcji}
\newcommand{\kategoria}[1]{\section{#1}} %olreślamy nazwę kateforii zadań
\newcommand{\zadStart}[1]{\begin{zad}#1\newline} %oznaczenie początku zadania
\newcommand{\zadStop}{\end{zad}}   %oznaczenie końca zadania
%Makra opcjonarne (nie muszą występować):
\newcommand{\rozwStart}[2]{\noindent \textbf{Rozwiązanie (autor #1 , recenzent #2): }\newline} %oznaczenie początku rozwiązania, opcjonarnie można wprowadzić informację o autorze rozwiązania zadania i recenzencie poprawności wykonania rozwiązania zadania
\newcommand{\rozwStop}{\newline}                                            %oznaczenie końca rozwiązania
\newcommand{\odpStart}{\noindent \textbf{Odpowiedź:}\newline}    %oznaczenie początku odpowiedzi końcowej (wypisanie wyniku)
\newcommand{\odpStop}{\newline}                                             %oznaczenie końca odpowiedzi końcowej (wypisanie wyniku)
\newcommand{\testStart}{\noindent \textbf{Test:}\newline} %ewentualne możliwe opcje odpowiedzi testowej: A. ? B. ? C. ? D. ? itd.
\newcommand{\testStop}{\newline} %koniec wprowadzania odpowiedzi testowych
\newcommand{\kluczStart}{\noindent \textbf{Test poprawna odpowiedź:}\newline} %klucz, poprawna odpowiedź pytania testowego (jedna literka): A lub B lub C lub D itd.
\newcommand{\kluczStop}{\newline} %koniec poprawnej odpowiedzi pytania testowego 
\newcommand{\wstawGrafike}[2]{\begin{figure}[h] \includegraphics[scale=#2] {#1} \end{figure}} %gdyby była potrzeba wstawienia obrazka, parametry: nazwa pliku, skala (jak nie wiesz co wpisać, to wpisz 1)
\begin{document}
\maketitle
\kategoria{Beger/c1.14d}
\zadStart{Zadanie z Beger C 1.14d moja wersja nr [nrWersji]}
%[p1]:[2,4,6,8,10]
%[p2]:[2,3,5,6,7,11,10,13]
%[p3]:[1,2,3,4,5,6,7]
%[g1]=2*[p2]
%[h1]=int([p1]/2)
%[k1]=-2*[p2]
%[A]=2/[k1]
%[k2]=[p1]+[h1]
%[b1]=[k2]*[A]
%[B]=[b1]/[p2]
%[c1]=[p3]*[A]+[h1]*[B]
%[lambda]=-[c1]
%[Ac]=10*[A]
%[Acc]=100*[A]
%[Accc]=1000*[A]
%[Bc]=10*[B]
%[Bcc]=100*[B]
%[Bccc]=1000*[B]
%[Lc]=10*[lambda]
%[Lcc]=100*[lambda]
%[Lccc]=1000*[lambda]
%[Delta]=[p1]**2+4*[p2]*[p3]
%[minia]=2*[p1]
%[4a]=4*[p2]**2
%[o1]=[p2]*[Delta]
%[gcd]=math.gcd([p1],[minia])
%[gcd2]=math.gcd([Delta],[4a])
%[gp1]=int([p1]/[gcd])
%[gp2]=int([minia]/[gcd])
%[gp3]=int([Delta]/[gcd2])
%[gp4]=int([4a]/[gcd2])
%[gp4i]=math.sqrt([gp4])
%[gp4ii]=int([gp4i])
%(([Ac]).is_integer() is True or ([Acc]).is_integer() is True or ([Accc]).is_integer() is True) and (([Bc]).is_integer() is True or ([Bcc]).is_integer() is True or ([Bccc]).is_integer() is True) and (([Lc]).is_integer() is True or ([Lcc]).is_integer() is True or ([Lccc]).is_integer() is True) and ([gp4i]).is_integer() is True and [gp4]>1
Metodą współczynników nieoznaczonych obliczyć całki funkcji niewymiernej.
$$\int \frac{2x^{2}}{\sqrt{[p1]x-[p2]x^2+[p3]}} \,dx$$
\zadStop
\rozwStart{Radosław Grzyb}{}
Do obliczenia całki wykorzystamy wzór:
$$\int \frac{W_n(x)}{\sqrt{ax^2+bx+c}} \,dx=W_{n-1}(x)\sqrt{ax^2+bx+c}+\lambda\int \frac{1}{\sqrt{ax^2+bx+c}} \,dx$$
Gdzie $W_n(x)$ oznacza wielomian stopnia $n$.\\
Podstawiając do wzoru otrzymujemy:
$$\int \frac{2x^{2}}{\sqrt{-[p2]x^2+[p1]x+[p3]}} \,dx=(Ax+B)\sqrt{-[p2]x^2+[p1]x+[p3]}+\lambda\int \frac{1}{\sqrt{-[p2]x^2+[p1]x+[p3]}} \,dx$$
Następnie różniczkujemy obie strony równania otrzymując:
$$\frac{2x^{2}}{\sqrt{-[p2]x^2+[p1]x+[p3]}}=A\sqrt{-[p2]x^2+[p1]x+[p3]}+\frac{(Ax+B)(-[g1]x+[p1])}{2\sqrt{-[p2]x^2+[p1]x+[p3]}}+ \frac{\lambda}{\sqrt{-[p2]x^2+[p1]x+[p3]}}$$
$$\frac{2x^{2}}{\sqrt{-[p2]x^2+[p1]x+[p3]}}=A\sqrt{-[p2]x^2+[p1]x+[p3]}+\frac{(Ax+B)(-[p2]x+[h1])}{\sqrt{-[p2]x^2+[p1]x+[p3]}}+ \frac{\lambda}{\sqrt{-[p2]x^2+[p1]x+[p3]}}$$
Mnożymy obie strony równania przez $\sqrt{-[p2]x^2+[p1]x+[p3]}$:
$$2x^{2}=A(-[p2]x^2+[p1]x+[p3])+(Ax+B)(-[p2]x+[h1])+\lambda$$
$$2x^{2}=-[p2]Ax^2+[p1]Ax+[p3]A-[p2]Ax^2+[h1]Ax-[p2]Bx+[h1]B+\lambda$$
Przyrównując do siebie odpowiednie współczynniki otrzymujemy do rozwiązania prosty układ równań:
$$\begin{cases} 2=[k1]A \implies A=[A] \\ 0=[k2]A-[p2]B \implies 0=[b1]-[p2]B \implies B=[B]  \\ 0=[p3]A+[h1]B+\lambda \implies 0=[c1]+\lambda \implies \lambda=[lambda] \end{cases}$$
Podstawmy otrzymane wartości do naszego wzoru:
$$\int \frac{2x^{2}}{\sqrt{-[p2]x^2+[p1]x+[p3]}} \,dx=([A]x[B])\sqrt{-[p2]x^2+[p1]x+[p3]}+[lambda]\int \frac{1}{\sqrt{-[p2]x^2+[p1]x+[p3]}} \,dx$$
Super! Ale to jeszcze nie koniec... Pozostaje nam obliczyć całkę $\int \frac{1}{\sqrt{-[p2]x^2+[p1]x+[p3]}} \,dx$. Do tego celu wykorzystamy pomocninczy wzór: $$ax^{2}+bx+c = a[(x+\frac{b}{2a})^{2}-\frac{\Delta}{4a^{2}}]$$.\\
Obliczmy najpierw deltę i $4a^2$:
$$\Delta=[p1]^2-4\cdot(-[p2])\cdot[p3]=[Delta]$$
$$4a^2=4(-[p2])^2=[4a]$$
A teraz podstawmy:
$$-[p2][(x+\frac{[p1]}{[minia]})^{2}-\frac{[Delta]}{[4a]}]=[p2][-(x+\frac{[p1]}{[minia]})^{2}+\frac{[Delta]}{[4a]}]=[p2][-(x+\frac{[gp1]}{[gp2]})^{2}+\frac{[gp3]}{[gp4]}]$$
A więc:
$$\int \frac{1}{\sqrt{[p2][-(x+\frac{[gp1]}{[gp2]})^{2}+\frac{[gp3]}{[gp4]}]}} \,dx=\frac{1}{\sqrt{[p2]}}\int \frac{1}{\sqrt{\frac{[gp3]}{[gp4]}-(x+\frac{[gp1]}{[gp2]})^{2}}}\,dx$$
Podstawmy teraz $t=x+\frac{[gp1]}{[gp2]}$, a więc $dt=dx$
$$\frac{1}{\sqrt{[p2]}}\int \frac{1}{\sqrt{\frac{[gp3]}{[gp4]}-t^{2}}}\,dt=\frac{1}{\sqrt{[p2]}}\int \frac{1}{\sqrt{(\frac{\sqrt{[gp3]}}{[gp4ii]})^2-t^{2}}}\,dt$$
Skorzystamy teraz z gotowego wzoru na całkę: $\int \frac{1}{\sqrt{a^{2}-x^{2}}} \,dx = \arcsin(\frac{x}{a})+C$\\
Po podstawieniu otrzymujemy:
$$\frac{1}{\sqrt{[p2]}}\arcsin(\frac{x}{\frac{\sqrt{[gp3]}}{[gp4ii]}})+C=\frac{1}{\sqrt{[p2]}}\arcsin(\frac{[gp4ii]x}{\sqrt{[gp3]}})+C$$
A zatem finalny wynik naszej całki to:
$$\int \frac{2x^{2}}{\sqrt{-[p2]x^2+[p1]x+[p3]}} \,dx=([A]x[B])\sqrt{-[p2]x^2+[p1]x+[p3]}+\frac{[lambda]}{\sqrt{[p2]}}\arcsin(\frac{[gp4ii]x}{\sqrt{[gp3]}})+C$$
\rozwStop
\odpStart
$$([A]x[B])\sqrt{-[p2]x^2+[p1]x+[p3]}+\frac{[lambda]}{\sqrt{[p2]}}\arcsin(\frac{[gp4ii]x}{\sqrt{[gp3]}})+C$$
\odpStop
\testStart
A.$$([A]x^{2}[B]+[p3])\sqrt{-[p2]x^2+[p1]x+[p3]}+\frac{[lambda]}{\sqrt{[p2]}}\arcsin(\frac{[gp4ii]x}{\sqrt{[gp3]}})+C$$
B.$$([A]x[B])\sqrt{-[p2]x^2+[p1]x+[p3]}+\frac{[lambda]}{\sqrt{[p2]}}\arcsin(\frac{[gp4ii]x}{\sqrt{[gp3]}})+C$$
C.$$([A]x[B])\sqrt{-[p2]x^2+[p1]x+[p3]}+\frac{[lambda]}{\sqrt{[p2]}}\arccos(\frac{[gp4ii]x}{\sqrt{[gp3]}})+C$$
D.$$([A]x^{3}[B])\sqrt{-[p2]x^2+[p1]x+[p3]}+\frac{[lambda]}{\sqrt{[p2]}}\arcsin(\frac{[gp4ii]x}{\sqrt{[gp3]}})+C$$
\testStop
\kluczStart
B
\kluczStop
\end{document}