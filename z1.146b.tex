\documentclass[12pt, a4paper]{article}
\usepackage[utf8]{inputenc}
\usepackage{polski}

\usepackage{amsthm}  %pakiet do tworzenia twierdzeń itp.
\usepackage{amsmath} %pakiet do niektórych symboli matematycznych
\usepackage{amssymb} %pakiet do symboli mat., np. \nsubseteq
\usepackage{amsfonts}
\usepackage{graphicx} %obsługa plików graficznych z rozszerzeniem png, jpg
\theoremstyle{definition} %styl dla definicji
\newtheorem{zad}{} 
\title{Multizestaw zadań}
\author{Robert Fidytek}
%\date{\today}
\date{}
\newcounter{liczniksekcji}
\newcommand{\kategoria}[1]{\section{#1}} %olreślamy nazwę kateforii zadań
\newcommand{\zadStart}[1]{\begin{zad}#1\newline} %oznaczenie początku zadania
\newcommand{\zadStop}{\end{zad}}   %oznaczenie końca zadania
%Makra opcjonarne (nie muszą występować):
\newcommand{\rozwStart}[2]{\noindent \textbf{Rozwiązanie (autor #1 , recenzent #2): }\newline} %oznaczenie początku rozwiązania, opcjonarnie można wprowadzić informację o autorze rozwiązania zadania i recenzencie poprawności wykonania rozwiązania zadania
\newcommand{\rozwStop}{\newline}                                            %oznaczenie końca rozwiązania
\newcommand{\odpStart}{\noindent \textbf{Odpowiedź:}\newline}    %oznaczenie początku odpowiedzi końcowej (wypisanie wyniku)
\newcommand{\odpStop}{\newline}                                             %oznaczenie końca odpowiedzi końcowej (wypisanie wyniku)
\newcommand{\testStart}{\noindent \textbf{Test:}\newline} %ewentualne możliwe opcje odpowiedzi testowej: A. ? B. ? C. ? D. ? itd.
\newcommand{\testStop}{\newline} %koniec wprowadzania odpowiedzi testowych
\newcommand{\kluczStart}{\noindent \textbf{Test poprawna odpowiedź:}\newline} %klucz, poprawna odpowiedź pytania testowego (jedna literka): A lub B lub C lub D itd.
\newcommand{\kluczStop}{\newline} %koniec poprawnej odpowiedzi pytania testowego 
\newcommand{\wstawGrafike}[2]{\begin{figure}[h] \includegraphics[scale=#2] {#1} \end{figure}} %gdyby była potrzeba wstawienia obrazka, parametry: nazwa pliku, skala (jak nie wiesz co wpisać, to wpisz 1)

\begin{document}
\maketitle


\kategoria{Wikieł/Z1.146b}
\zadStart{Zadanie z Wikieł Z 1.146 b) moja wersja nr [nrWersji]}
%[p1]:[1,2,3,4,5,6]
%[p2]:[2,3,5,6,7,8,10]
%[a]=random.randint(2,10)
%[b]=random.randint(2,10)
%[c]=random.randint(2,10)
%[d]=random.randint(1,10)
%[a3]=[p1]**(3)
%[p21]=[p2]+16
%[3a2b]=3*[p1]*[p1]
%[3ab2]=3*[p1]*[p2]
%[b2]=[p2]
%[a2]=[p1]*[p1]
%[2ab]=2*[p1]
%[cp1]=[c]*[p1]
%[ar3]=([a3]+[3ab2])*[a]
%[ar3p]=([3a2b]+[b2])*[a]
%[br2]=([a2]+[b2])*[b]
%[br2p]=[b]*[2ab]
%[cp1d]=-[d]+[cp1]
%[bezpierw]=[ar3]-[br2]+[cp1d]
%[zpierw]=[c]+[ar3p]-[br2p]
%[abszpierw]=abs([zpierw])
%[cp1d]>0 and [bezpierw]>0 and [zpierw]>0 and [zpierw]!=1 
Obliczyć wartość wielomianu:\\ $W(x)=[a]x^{3}-[b]x^{2}+[c]x-[d]$ dla $x=[p1]+\sqrt{[p2]}$.
\zadStop
\rozwStart{Wojciech Przybylski}{Maja Szabłowska}
$$W([p1]+\sqrt{[p2]})=[a]([p1]+\sqrt{[p2]})^{3}-[b]([p1]+\sqrt{[p2]})^{2}+[c]([p1]+\sqrt{[p2]})-[d]=$$
$$=[a]([a3]+[3a2b]\sqrt{[p2]}+[3ab2]+[b2]\sqrt{[p2]})+$$
$$-[b]([a2]+[2ab]\sqrt{[p2]}+[b2])+[cp1]+[c]\sqrt{[p2]}-[d]=$$
$$=[ar3]+[ar3p]\sqrt{[p2]}-[br2]-[br2p]\sqrt{[p2]}+[cp1d]+[c]\sqrt{[p2]}=$$
$$=[bezpierw]+[abszpierw]\sqrt{[p2]}$$
\rozwStop
\odpStart
$[bezpierw]+[abszpierw]\sqrt{[p2]}$
\odpStop
\testStart
A. $[bezpierw]+[abszpierw]\sqrt{[p2]}$\\
B. $[bezpierw]-[abszpierw]\sqrt{[p2]}$\\
C. $[bezpierw]$\\
D. $[abszpierw]\sqrt{[p2]}$\\
E. $[zpierw]-[bezpierw]\sqrt{[p21]}$\\
F. $-[abszpierw]\sqrt{[p2]}$
\testStop
\kluczStart
A
\kluczStop



\end{document}