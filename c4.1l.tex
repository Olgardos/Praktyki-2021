\documentclass[12pt, a4paper]{article}
\usepackage[utf8]{inputenc}
\usepackage{polski}

\usepackage{amsthm}  %pakiet do tworzenia twierdzeń itp.
\usepackage{amsmath} %pakiet do niektórych symboli matematycznych
\usepackage{amssymb} %pakiet do symboli mat., np. \nsubseteq
\usepackage{amsfonts}
\usepackage{graphicx} %obsługa plików graficznych z rozszerzeniem png, jpg
\theoremstyle{definition} %styl dla definicji
\newtheorem{zad}{} 
\title{Multizestaw zadań}
\author{Robert Fidytek}
%\date{\today}
\date{}
\newcounter{liczniksekcji}
\newcommand{\kategoria}[1]{\section{#1}} %olreślamy nazwę kateforii zadań
\newcommand{\zadStart}[1]{\begin{zad}#1\newline} %oznaczenie początku zadania
\newcommand{\zadStop}{\end{zad}}   %oznaczenie końca zadania
%Makra opcjonarne (nie muszą występować):
\newcommand{\rozwStart}[2]{\noindent \textbf{Rozwiązanie (autor #1 , recenzent #2): }\newline} %oznaczenie początku rozwiązania, opcjonarnie można wprowadzić informację o autorze rozwiązania zadania i recenzencie poprawności wykonania rozwiązania zadania
\newcommand{\rozwStop}{\newline}                                            %oznaczenie końca rozwiązania
\newcommand{\odpStart}{\noindent \textbf{Odpowiedź:}\newline}    %oznaczenie początku odpowiedzi końcowej (wypisanie wyniku)
\newcommand{\odpStop}{\newline}                                             %oznaczenie końca odpowiedzi końcowej (wypisanie wyniku)
\newcommand{\testStart}{\noindent \textbf{Test:}\newline} %ewentualne możliwe opcje odpowiedzi testowej: A. ? B. ? C. ? D. ? itd.
\newcommand{\testStop}{\newline} %koniec wprowadzania odpowiedzi testowych
\newcommand{\kluczStart}{\noindent \textbf{Test poprawna odpowiedź:}\newline} %klucz, poprawna odpowiedź pytania testowego (jedna literka): A lub B lub C lub D itd.
\newcommand{\kluczStop}{\newline} %koniec poprawnej odpowiedzi pytania testowego 
\newcommand{\wstawGrafike}[2]{\begin{figure}[h] \includegraphics[scale=#2] {#1} \end{figure}} %gdyby była potrzeba wstawienia obrazka, parametry: nazwa pliku, skala (jak nie wiesz co wpisać, to wpisz 1)

\begin{document}
\maketitle


\kategoria{Dymkowska, Beger/c4.1l}
\zadStart{Zadanie z Dymkowskiej, Beger C 4.1l) moja wersja nr [nrWersji]}
%[p1]:[2,3,4,5,6,7,8,9,10]
%[p2]:[2,3,4,5,6,7,8,9,10]
Obliczyć całkę potrójną po prostopadłościanie P $$\iiint_P 2xe^{x^2+y+z}\ dxdydz, P: -1 \leq x \leq 0, -[p1] \leq y \leq 0, -[p2] \leq z \leq 0$$
\zadStop
\rozwStart{Jakub Janik}{}
$$\iiint_P 2xe^{x^2+y+z}\ dxdydz=\int_{-1}^0 2xe^{x^2}\ dx\cdot\int_{-[p1]}^0e^y\ dy\cdot\int_{-[p2]}e^z\ dz$$
Skorzystamy z podstawienia$$x^2=t$$ 
$$ 2x\ dr=dt$$
\begin{displaymath}
\left.\begin{array}{c|c|c}
x & -1 & 0 \\ \hline
t & 1 & 0
\end{array}\right.
\end{displaymath}
Dostajemy
$$\int_{-1}^0 e^{t}\ dt\cdot\int_{-[p1]}^0e^y\ dy\cdot\int_{-[p2]}e^z\ dz=$$
$$e^t\Big|_1^0\cdot e^y\Big|_{-[p1]}^0\cdot e^z\Big|_{-[p2]}^0=(1-e)(1-\frac{1}{e^{[p1]}})(1-\frac{1}{e^{[p2]}})$$
\rozwStop
\odpStart
$(1-e)(1-\frac{1}{e^{[p1]}})(1-\frac{1}{e^{[p2]}})$
\odpStop
\testStart
A.$(1-e)(1-\frac{1}{e^{[p1]}})(1-\frac{1}{e^{[p2]}})$
B.$0$
C.$-(1-e)(1-\frac{1}{e^{[p1]}})(1-\frac{1}{e^{[p2]}})$
D.$\infty$
\testStop
\kluczStart
A
\kluczStop



\end{document}