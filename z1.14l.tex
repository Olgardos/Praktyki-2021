\documentclass[12pt, a4paper]{article}
\usepackage[utf8]{inputenc}
\usepackage{polski}

\usepackage{amsthm}  %pakiet do tworzenia twierdzeń itp.
\usepackage{amsmath} %pakiet do niektórych symboli matematycznych
\usepackage{amssymb} %pakiet do symboli mat., np. \nsubseteq
\usepackage{amsfonts}
\usepackage{graphicx} %obsługa plików graficznych z rozszerzeniem png, jpg
\theoremstyle{definition} %styl dla definicji
\newtheorem{zad}{} 
\title{Multizestaw zadań}
\author{Robert Fidytek}
%\date{\today}
\date{}
\newcounter{liczniksekcji}
\newcommand{\kategoria}[1]{\section{#1}} %olreślamy nazwę kateforii zadań
\newcommand{\zadStart}[1]{\begin{zad}#1\newline} %oznaczenie początku zadania
\newcommand{\zadStop}{\end{zad}}   %oznaczenie końca zadania
%Makra opcjonarne (nie muszą występować):
\newcommand{\rozwStart}[2]{\noindent \textbf{Rozwiązanie (autor #1 , recenzent #2): }\newline} %oznaczenie początku rozwiązania, opcjonarnie można wprowadzić informację o autorze rozwiązania zadania i recenzencie poprawności wykonania rozwiązania zadania
\newcommand{\rozwStop}{\newline}                                            %oznaczenie końca rozwiązania
\newcommand{\odpStart}{\noindent \textbf{Odpowiedź:}\newline}    %oznaczenie początku odpowiedzi końcowej (wypisanie wyniku)
\newcommand{\odpStop}{\newline}                                             %oznaczenie końca odpowiedzi końcowej (wypisanie wyniku)
\newcommand{\testStart}{\noindent \textbf{Test:}\newline} %ewentualne możliwe opcje odpowiedzi testowej: A. ? B. ? C. ? D. ? itd.
\newcommand{\testStop}{\newline} %koniec wprowadzania odpowiedzi testowych
\newcommand{\kluczStart}{\noindent \textbf{Test poprawna odpowiedź:}\newline} %klucz, poprawna odpowiedź pytania testowego (jedna literka): A lub B lub C lub D itd.
\newcommand{\kluczStop}{\newline} %koniec poprawnej odpowiedzi pytania testowego 
\newcommand{\wstawGrafike}[2]{\begin{figure}[h] \includegraphics[scale=#2] {#1} \end{figure}} %gdyby była potrzeba wstawienia obrazka, parametry: nazwa pliku, skala (jak nie wiesz co wpisać, to wpisz 1)

\begin{document}
\maketitle


\kategoria{Wikieł/Z1.14l}
\zadStart{Zadanie z Wikieł Z 1.14 l) moja wersja nr [nrWersji]}
%[a]:[1,2,3,4,5,6]
%[b]:[1,2,3,4,5,6,7,8]
%[c]:[1,2,3,4,5,6]
%[d]:[2,3,4,5]
%[e]=[d]*[c]
%[f]:[1,2,3,4,5,6,7,8]
%[-a]=-[a]
%[g]=[d]-2
%[h]=[f]+[a]-[b]+[e]
%[i]=[f]-[a]-[b]+[e]
%[j]=2+[d]
%[k]=[f]-[a]+[b]+[e]
%[l]=2-[d]
%[m]=[f]-[a]+[b]-[e]
%[-a]<[b]<[c]<[d] and math.gcd([h],[g])==1 and math.gcd([i],[d])==1 and math.gcd([j],[k])==1 and math.gcd([l],[m])==1 and [g]!=1 and [k]!=1 and [l]!=1 and [l]!=0 and -[a]*[g]<[h] and [i]>[b]*[d] and [b]*[j]<[k]<[c]*[j] and [c]<([m]/[l])
Rozwiązać równanie $|x+[a]|+|x-[b]|-|[d]x-[e]|=[f]$.
\zadStop
\rozwStart{Klaudia Klejdysz}{}
$$|x+[a]|+|x-[b]|-|[d]x-[e]|=[f]$$
\indent Przypadek 1: $x\in(-\infty,-[a])$
$$-(x+[a])-(x-[b])+([d]x-[e])=[f]$$
$$-x-[a]-x+[b]+[d]x-[e]=[f]$$
$$(-1-1+[d])x=[f]+[a]-[b]+[e]$$
$$x=\frac{[h]}{[g]}$$
Ostatecznie otrzymujemy:
$$x\in(-\infty,-[a])\qquad\land\qquad x=\frac{[h]}{[g]}$$
Czyli z przypadku 1 otrzymujemy sprzeczność, a więc: $x\in\emptyset$.\\


Przypadek 2: $x\in[-[a],[b]]$ $$(x+[a])-(x-[b])+([d]x-[e])=[f]$$
$$x+[a]-x+[b]+[d]x-[e]=[f]$$
$$[d]x=[f]-[a]-[b]+[e]$$
$$x=\frac{[i]}{[d]}$$
Ostatecznie otrzymujemy:
$$x\in[-[a],[b]]\qquad\land\qquad x=\frac{[i]}{[d]}$$
Czyli z przypadku 2 otrzymujemy sprzeczność, a więc: $x\in\emptyset$.\\


Przypadek 3: $x\in([b],[c])$ $$(x+[a])+(x-[b])+([d]x-[e])=[f]$$
$$x+[a]+x-[b]+[d]x-[e]=[f]$$
$$(1+1+[d])x=[f]-[a]+[b]+[e]$$
$$x=\frac{[k]}{[j]}$$
Ostatecznie otrzymujemy:
$$x\in([b],[c])\qquad\land\qquad x=\frac{[k]}{[j]}$$
Czyli z przypadku 3 mamy: $x=\frac{[k]}{[j]}$.\\


Przypadek 4: $x\in[[c],\infty)$ $$(x+[a])+(x-[b])-([d]x-[e])=[f]$$
$$x+[a]+x-[b]-[d]x+[e]=[f]$$
$$(1+1-[d])x=[f]-[a]+[b]-[e]$$
$$x=\frac{[m]}{[l]}$$
Ostatecznie otrzymujemy:
$$x\in[[c],\infty)\qquad\land\qquad x=\frac{[m]}{[l]}$$
Czyli z przypadku 4 mamy: $x=\frac{[m]}{[l]}$.\\
Ostatecznym rozwiązaniem wyjściowego równania jest:
$$x\in\bigg\{\frac{[k]}{[j]},\frac{[m]}{[l]}\bigg\}$$
\rozwStop
\odpStart
$x\in\bigg\{\frac{[k]}{[j]},\frac{[m]}{[l]}\bigg\}$
\odpStop
\testStart
A.$x\in\bigg\{\frac{[k]}{[j]},\frac{[m]}{[l]}\bigg\}$\\
B.$x\in\mathbb{R}$\\
C.$x\in\emptyset$\\
D.$x\in\bigg\{\frac{[k]}{[j]}\bigg\}$
\testStop
\kluczStart
A
\kluczStop



\end{document}

