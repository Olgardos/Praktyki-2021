\documentclass[12pt, a4paper]{article}
\usepackage[utf8]{inputenc}
\usepackage{polski}

\usepackage{amsthm}  %pakiet do tworzenia twierdzeń itp.
\usepackage{amsmath} %pakiet do niektórych symboli matematycznych
\usepackage{amssymb} %pakiet do symboli mat., np. \nsubseteq
\usepackage{amsfonts}
\usepackage{graphicx} %obsługa plików graficznych z rozszerzeniem png, jpg
\theoremstyle{definition} %styl dla definicji
\newtheorem{zad}{} 
\title{Multizestaw zadań}
\author{Robert Fidytek}
%\date{\today}
\date{}
\newcounter{liczniksekcji}
\newcommand{\kategoria}[1]{\section{#1}} %olreślamy nazwę kateforii zadań
\newcommand{\zadStart}[1]{\begin{zad}#1\newline} %oznaczenie początku zadania
\newcommand{\zadStop}{\end{zad}}   %oznaczenie końca zadania
%Makra opcjonarne (nie muszą występować):
\newcommand{\rozwStart}[2]{\noindent \textbf{Rozwiązanie (autor #1 , recenzent #2): }\newline} %oznaczenie początku rozwiązania, opcjonarnie można wprowadzić informację o autorze rozwiązania zadania i recenzencie poprawności wykonania rozwiązania zadania
\newcommand{\rozwStop}{\newline}                                            %oznaczenie końca rozwiązania
\newcommand{\odpStart}{\noindent \textbf{Odpowiedź:}\newline}    %oznaczenie początku odpowiedzi końcowej (wypisanie wyniku)
\newcommand{\odpStop}{\newline}                                             %oznaczenie końca odpowiedzi końcowej (wypisanie wyniku)
\newcommand{\testStart}{\noindent \textbf{Test:}\newline} %ewentualne możliwe opcje odpowiedzi testowej: A. ? B. ? C. ? D. ? itd.
\newcommand{\testStop}{\newline} %koniec wprowadzania odpowiedzi testowych
\newcommand{\kluczStart}{\noindent \textbf{Test poprawna odpowiedź:}\newline} %klucz, poprawna odpowiedź pytania testowego (jedna literka): A lub B lub C lub D itd.
\newcommand{\kluczStop}{\newline} %koniec poprawnej odpowiedzi pytania testowego 
\newcommand{\wstawGrafike}[2]{\begin{figure}[h] \includegraphics[scale=#2] {#1} \end{figure}} %gdyby była potrzeba wstawienia obrazka, parametry: nazwa pliku, skala (jak nie wiesz co wpisać, to wpisz 1)

\begin{document}
\maketitle


\kategoria{Dymkowska, Beger/c4.7e}
\zadStart{Zadanie z Dymkowskiej, Beger C 4.7e) moja wersja nr [nrWersji]}
%[p1]:[2,3,4,5,6,7,8,9,10]
%[p2]:[2,3,4,5,6,7,8,9,10]
%[p3]=[p1]*[p2]
%[p4]=[p1]*[p1]
%[a]=[p4]*[p1]
%[b]=[p2]*[p2]
%[c]=[b]*[p2]*3
%[a1]=[a]*2
%[b1]=[b]*3
%[c1]=4*[c]
%[wyn]=[a1]*[p2]*4-[a]*[p2]*4+[a]*[p2]
%[g]=math.gcd([wyn],12)
%[up]=int([wyn]/[g])
%[down]=int(12/[g])
%[down]>1
Za pomocą całki potrójnej obliczyć objętość bryły ograniczonej powierzchniami
$$x=0, y=0, [p1]x+[p2]y=[p3], z=0, z=[p4]-y^2$$
\zadStop
\rozwStart{Jakub Janik}{}
Obszar V wyraża się $$0 \leq x \leq [p2], 0 \leq y \leq [p1]-\frac{[p1]}{[p2]}x, 0 \leq z \leq [p4]-y^2$$
Możemy przejść do obliczenia objętości
$$\iiint_V dxdydz=\int_0^{[p2]}dx\int_0^{[p1]-\frac{[p1]}{[p2]}x}dy\int_0^{[p4]-y^2}dz=$$
$$=\int_0^{[p2]}dx\int_0^{[p1]-\frac{[p1]}{[p2]}x}[p4]-y^2\ dy=\int_0^{[p2]}([p4]y-\frac{1}{3}y^3\Big|_0^{[p1]-\frac{[p1]}{[p2]}x})dx=$$
$$=\int_0^{[p2]}[a]-\frac{[a]}{[p2]}x-\frac{[a]}{3}+\frac{[a]}{[p2]}x-\frac{[a]}{[b]}x^2+\frac{[a]}{[c]}x^3\ dx=$$
$$=\frac{[a1]}{3}x-\frac{[a]}{[b1]}x^3+\frac{[a]}{[c1]}x^4\Big|_0^{[p2]}=\frac{[up]}{[down]}$$
\rozwStop
\odpStart
$\frac{[up]}{[down]}$
\odpStop
\testStart
A.$\frac{[up]}{[down]}$
B.$0$
C.$-\frac{[up]}{[down]}$
D.$\infty$
\testStop
\kluczStart
A
\kluczStop



\end{document}