\documentclass[12pt, a4paper]{article}
\usepackage[utf8]{inputenc}
\usepackage{polski}

\usepackage{amsthm}  %pakiet do tworzenia twierdzeń itp.
\usepackage{amsmath} %pakiet do niektórych symboli matematycznych
\usepackage{amssymb} %pakiet do symboli mat., np. \nsubseteq
\usepackage{amsfonts}
\usepackage{graphicx} %obsługa plików graficznych z rozszerzeniem png, jpg
\theoremstyle{definition} %styl dla definicji
\newtheorem{zad}{} 
\title{Multizestaw zadań}
\author{Robert Fidytek}
%\date{\today}
\date{}
\newcounter{liczniksekcji}
\newcommand{\kategoria}[1]{\section{#1}} %olreślamy nazwę kateforii zadań
\newcommand{\zadStart}[1]{\begin{zad}#1\newline} %oznaczenie początku zadania
\newcommand{\zadStop}{\end{zad}}   %oznaczenie końca zadania
%Makra opcjonarne (nie muszą występować):
\newcommand{\rozwStart}[2]{\noindent \textbf{Rozwiązanie (autor #1 , recenzent #2): }\newline} %oznaczenie początku rozwiązania, opcjonarnie można wprowadzić informację o autorze rozwiązania zadania i recenzencie poprawności wykonania rozwiązania zadania
\newcommand{\rozwStop}{\newline}                                            %oznaczenie końca rozwiązania
\newcommand{\odpStart}{\noindent \textbf{Odpowiedź:}\newline}    %oznaczenie początku odpowiedzi końcowej (wypisanie wyniku)
\newcommand{\odpStop}{\newline}                                             %oznaczenie końca odpowiedzi końcowej (wypisanie wyniku)
\newcommand{\testStart}{\noindent \textbf{Test:}\newline} %ewentualne możliwe opcje odpowiedzi testowej: A. ? B. ? C. ? D. ? itd.
\newcommand{\testStop}{\newline} %koniec wprowadzania odpowiedzi testowych
\newcommand{\kluczStart}{\noindent \textbf{Test poprawna odpowiedź:}\newline} %klucz, poprawna odpowiedź pytania testowego (jedna literka): A lub B lub C lub D itd.
\newcommand{\kluczStop}{\newline} %koniec poprawnej odpowiedzi pytania testowego 
\newcommand{\wstawGrafike}[2]{\begin{figure}[h] \includegraphics[scale=#2] {#1} \end{figure}} %gdyby była potrzeba wstawienia obrazka, parametry: nazwa pliku, skala (jak nie wiesz co wpisać, to wpisz 1)

\begin{document}
\maketitle


\kategoria{Dymkowska, Beger/c4.7f}
\zadStart{Zadanie z Dymkowskiej, Beger C 4.7f) moja wersja nr [nrWersji]}
%[p1]:[2,3,4,5,6,7,8,9,10]
%[p13]=pow([p1],3)
%[a]=[p13]+[p1]*3
%[up1]=4*[p13]+4*[p1]-[a]
%[down1]=pow([p1],4)*12
%[g]=math.gcd([up1],[down1])
%[up]=int([up1]/[g])
%[down]=int([down1]/[g])
Za pomocą całki potrójnej obliczyć objętość bryły ograniczonej powierzchniami
$$x=0, y=1, y=[p1]x, z=0, z=-x^2-y^2$$
\zadStop
\rozwStart{Jakub Janik}{}
Obszar V wyraża się $$0 \leq x \leq \frac{1}{[p1]}, [p1]x \leq y \leq 1, -x^2-y^2 \leq z \leq 0$$
Możemy przejść do obliczenia objętości
$$\iiint_V dxdydz=\int_0^{\frac{1}{[p1]}}dx\int_{[p1]x}^1dy\int_{-x^2-y^2}^0dz=$$
$$=\int_0^{\frac{1}{[p1]}}dx\int_{[p1]x}^1x^2+y^2\ dy=\int_0^{\frac{1}{[p1]}}x^2y+\frac{1}{3}y^3\Big|_{[p1]x}^1dx=$$
$$=\int_0^{\frac{1}{[p1]}}x^2-[p1]x^3+\frac{1}{3}-\frac{[p13]}{3}x^3\ dx=$$
$$=\frac{1}{3}x+\frac{1}{3}x^3-\frac{[a]}{12}x^4\Big|_0^{\frac{1}{[p1]}}=\frac{[up]}{[down]}$$
\rozwStop
\odpStart
$\frac{[up]}{[down]}$
\odpStop
\testStart
A.$\frac{[up]}{[down]}$
B.$0$
C.$-\frac{[up]}{[down]}$
D.$\infty$
\testStop
\kluczStart
A
\kluczStop



\end{document}