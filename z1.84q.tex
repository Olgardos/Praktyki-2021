\documentclass[12pt, a4paper]{article}
\usepackage[utf8]{inputenc}
\usepackage{polski}
\usepackage{amsthm}  %pakiet do tworzenia twierdzeń itp.
\usepackage{amsmath} %pakiet do niektórych symboli matematycznych
\usepackage{amssymb} %pakiet do symboli mat., np. \nsubseteq
\usepackage{amsfonts}
\usepackage{graphicx} %obsługa plików graficznych z rozszerzeniem png, jpg
\theoremstyle{definition} %styl dla definicji
\newtheorem{zad}{} 
\title{Multizestaw zadań}
\author{Radosław Grzyb}
%\date{\today}
\date{}
\newcounter{liczniksekcji}
\newcommand{\kategoria}[1]{\section{#1}} %olreślamy nazwę kateforii zadań
\newcommand{\zadStart}[1]{\begin{zad}#1\newline} %oznaczenie początku zadania
\newcommand{\zadStop}{\end{zad}}   %oznaczenie końca zadania
%Makra opcjonarne (nie muszą występować):
\newcommand{\rozwStart}[2]{\noindent \textbf{Rozwiązanie (autor #1 , recenzent #2): }\newline} %oznaczenie początku rozwiązania, opcjonarnie można wprowadzić informację o autorze rozwiązania zadania i recenzencie poprawności wykonania rozwiązania zadania
\newcommand{\rozwStop}{\newline}                                            %oznaczenie końca rozwiązania
\newcommand{\odpStart}{\noindent \textbf{Odpowiedź:}\newline}    %oznaczenie początku odpowiedzi końcowej (wypisanie wyniku)
\newcommand{\odpStop}{\newline}                                             %oznaczenie końca odpowiedzi końcowej (wypisanie wyniku)
\newcommand{\testStart}{\noindent \textbf{Test:}\newline} %ewentualne możliwe opcje odpowiedzi testowej: A. ? B. ? C. ? D. ? itd.
\newcommand{\testStop}{\newline} %koniec wprowadzania odpowiedzi testowych
\newcommand{\kluczStart}{\noindent \textbf{Test poprawna odpowiedź:}\newline} %klucz, poprawna odpowiedź pytania testowego (jedna literka): A lub B lub C lub D itd.
\newcommand{\kluczStop}{\newline} %koniec poprawnej odpowiedzi pytania testowego 
\newcommand{\wstawGrafike}[2]{\begin{figure}[h] \includegraphics[scale=#2] {#1} \end{figure}} %gdyby była potrzeba wstawienia obrazka, parametry: nazwa pliku, skala (jak nie wiesz co wpisać, to wpisz 1)
\begin{document}
\maketitle
\kategoria{Wikieł/Z1.84q}
\zadStart{Zadanie z Wikieł Z 1.84q moja wersja nr [nrWersji]}
%[p1]:[1,2,3,4,5,6,7,8,10]
%[p2]:[2,3,5]
%[p3]:[1,2,3,4,5,6,7,8,9,11,12,13]
%[r]=[p2]**[p1]
%[pp]=[p2]**2
%[gcd]=math.gcd([p1],[pp])
%[gcd1]=int([p1]/[gcd])
%[gcd2]=int([pp]/[gcd])
%[f]=[p3]*[gcd2]
%[Delta]=[p1]**2+4*[f]*[gcd2]
%[sDelta]=math.sqrt([Delta])
%[tDelta]=int([sDelta])
%[t1]=int((-[p1]-[tDelta])/2)
%[t2]=int((-[p1]+[tDelta])/2)
%[Delta]>0 and ([sDelta]).is_integer() is True
Rozwiązać równanie:
$$[pp]^x+[p1]\cdot[p2]^{x-\frac{1}{2}}=[p3]$$
\zadStop
\rozwStart{Radosław Grzyb}{}
Przekształcamy nasze równanie:
$$[p2]^{2x}+[p1]\cdot\frac{1}{[pp]}\cdot[p2]^{x}=[p3]$$
$$[p2]^{2x}+\frac{[gcd1]}{[gcd2]}\cdot[p2]^{x}=[p3]$$
$$[gcd2]\cdot[p2]^{2x}+[gcd1]\cdot[p2]^{x}-[f]=0$$
Podstawiając $t=[p2]^{x}$ otrzymamy do rozwiązania równanie kwadratowe:
$$[gcd2]t^{2}+[p1]t-[f]=0$$
$$\Delta_{t}=[p1]^2-4\cdot[gcd2]\cdot(-[f])=[Delta]\implies \sqrt{\Delta_{t}}=[tDelta]$$\\
Czas znaleźć miejsca zerowe:
$$t_{1}=\frac{-[p1]-[tDelta]}{2\cdot[gcd2]}=[t1]$$
$$t_{2}=\frac{-[p1]+[tDelta]}{2\cdot[gcd2]}=[t2]$$
Współczynnik $a$ naszej funkcji kwadratowej jest dodatni oraz odrzucamy ujemne rozwiązania ponieważ $[p2]^x\geq0$. Zatem naszym rozwiązaniem jest:
$$t=[t2]$$
$$[p2]^x=[t2] \implies x=\frac{\ln[t2]}{\ln[p2]}$$
\rozwStop
\odpStart
$$x=\frac{\ln[t2]}{\ln[p2]}$$
\odpStop
\testStart
A.$$x=\frac{\ln[t2]}{\sin[p2]}$$
B.$$x=\frac{\arcsin[t2]}{\ln[p2]}$$
C.$$x=\frac{\tan[t2]}{\ln[p2]}$$
D.$$x=\frac{\ln[t2]}{\ln[p2]}$$
\testStop
\kluczStart
D
\kluczStop
\end{document}