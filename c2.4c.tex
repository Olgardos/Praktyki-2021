\documentclass[12pt, a4paper]{article}
\usepackage[utf8]{inputenc}
\usepackage{polski}

\usepackage{amsthm}  %pakiet do tworzenia twierdzeń itp.
\usepackage{amsmath} %pakiet do niektórych symboli matematycznych
\usepackage{amssymb} %pakiet do symboli mat., np. \nsubseteq
\usepackage{amsfonts}
\usepackage{graphicx} %obsługa plików graficznych z rozszerzeniem png, jpg
\theoremstyle{definition} %styl dla definicji
\newtheorem{zad}{} 
\title{Multizestaw zadań}
\author{Robert Fidytek}
%\date{\today}
\date{}
\newcounter{liczniksekcji}
\newcommand{\kategoria}[1]{\section{#1}} %olreślamy nazwę kateforii zadań
\newcommand{\zadStart}[1]{\begin{zad}#1\newline} %oznaczenie początku zadania
\newcommand{\zadStop}{\end{zad}}   %oznaczenie końca zadania
%Makra opcjonarne (nie muszą występować):
\newcommand{\rozwStart}[2]{\noindent \textbf{Rozwiązanie (autor #1 , recenzent #2): }\newline} %oznaczenie początku rozwiązania, opcjonarnie można wprowadzić informację o autorze rozwiązania zadania i recenzencie poprawności wykonania rozwiązania zadania
\newcommand{\rozwStop}{\newline}                                            %oznaczenie końca rozwiązania
\newcommand{\odpStart}{\noindent \textbf{Odpowiedź:}\newline}    %oznaczenie początku odpowiedzi końcowej (wypisanie wyniku)
\newcommand{\odpStop}{\newline}                                             %oznaczenie końca odpowiedzi końcowej (wypisanie wyniku)
\newcommand{\testStart}{\noindent \textbf{Test:}\newline} %ewentualne możliwe opcje odpowiedzi testowej: A. ? B. ? C. ? D. ? itd.
\newcommand{\testStop}{\newline} %koniec wprowadzania odpowiedzi testowych
\newcommand{\kluczStart}{\noindent \textbf{Test poprawna odpowiedź:}\newline} %klucz, poprawna odpowiedź pytania testowego (jedna literka): A lub B lub C lub D itd.
\newcommand{\kluczStop}{\newline} %koniec poprawnej odpowiedzi pytania testowego 
\newcommand{\wstawGrafike}[2]{\begin{figure}[h] \includegraphics[scale=#2] {#1} \end{figure}} %gdyby była potrzeba wstawienia obrazka, parametry: nazwa pliku, skala (jak nie wiesz co wpisać, to wpisz 1)

\begin{document}
\maketitle



\kategoria{Dymkowska,Beger/C2.4c}
\zadStart{Zadanie z Dymkowska,Beger C 2.4 c) moja wersja nr [nrWersji]}
%[a]:[2,3,4,5,6,7,8]
%[b]=[a]/2
Obliczyć całkę oznaczoną $\displaystyle \int_{0}^{\pi/4} [a]tg^3(x) \ dx $
\zadStop
\rozwStart{Mirella Narewska}{}
$$\int_{0}^{\pi/4} [a]tg^3(x) \ dx = [a] \int_{0}^{\pi/4} tg^3(x) \ dx= $$
$$\text{Korzystamy ze wzoru redukcyjnego, gdzie m=3:}\int \tg^m{x} \ dx=\frac{\tg^{m-1}{x}}{m-1} -\int \tg^{-2+m}{x} \ dx$$
$$=[a]\frac{\tg^2{x}}{2}|_{0}^{\pi/4}-[a]\int_{0}^{\pi/4} \tg{x} \ dx=[b]-[a]\int_{0}^{\pi/4} \frac{\sin{x}}{\cos{x}} \ dx$$
$$\text{Całkujemy przez podstawienie:}$$
$$u=\cos{x} \Rightarrow du=-\sin{x} \ dx$$
$$[b]-\int_{1}^{\frac{\sqrt{2}}{2}} -\frac{1}{u}=[b]+\ln{u}|_{1}^{\frac{\sqrt{2}}{2}}=[b]-\frac{\ln{2}}{2}$$
\odpStart
$$[b]-\frac{\ln{2}}{2}$$
\odpStop
\testStart
A.$[b]-\frac{\ln{2}}{2}$
\\
B.$\cos{x}$
\\
C.$0$
\\
D.$cos([a])$
\testStop
\kluczStart
A
\kluczStop


\end{document}