\documentclass[12pt, a4paper]{article}
\usepackage[utf8]{inputenc}
\usepackage{polski}
\usepackage{amsthm}  %pakiet do tworzenia twierdzeń itp.
\usepackage{amsmath} %pakiet do niektórych symboli matematycznych
\usepackage{amssymb} %pakiet do symboli mat., np. \nsubseteq
\usepackage{amsfonts}
\usepackage{graphicx} %obsługa plików graficznych z rozszerzeniem png, jpg
\theoremstyle{definition} %styl dla definicji
\newtheorem{zad}{} 
\title{Multizestaw zadań}
\author{Radosław Grzyb}
%\date{\today}
\date{}
\newcounter{liczniksekcji}
\newcommand{\kategoria}[1]{\section{#1}} %olreślamy nazwę kateforii zadań
\newcommand{\zadStart}[1]{\begin{zad}#1\newline} %oznaczenie początku zadania
\newcommand{\zadStop}{\end{zad}}   %oznaczenie końca zadania
%Makra opcjonarne (nie muszą występować):
\newcommand{\rozwStart}[2]{\noindent \textbf{Rozwiązanie (autor #1 , recenzent #2): }\newline} %oznaczenie początku rozwiązania, opcjonarnie można wprowadzić informację o autorze rozwiązania zadania i recenzencie poprawności wykonania rozwiązania zadania
\newcommand{\rozwStop}{\newline}                                            %oznaczenie końca rozwiązania
\newcommand{\odpStart}{\noindent \textbf{Odpowiedź:}\newline}    %oznaczenie początku odpowiedzi końcowej (wypisanie wyniku)
\newcommand{\odpStop}{\newline}                                             %oznaczenie końca odpowiedzi końcowej (wypisanie wyniku)
\newcommand{\testStart}{\noindent \textbf{Test:}\newline} %ewentualne możliwe opcje odpowiedzi testowej: A. ? B. ? C. ? D. ? itd.
\newcommand{\testStop}{\newline} %koniec wprowadzania odpowiedzi testowych
\newcommand{\kluczStart}{\noindent \textbf{Test poprawna odpowiedź:}\newline} %klucz, poprawna odpowiedź pytania testowego (jedna literka): A lub B lub C lub D itd.
\newcommand{\kluczStop}{\newline} %koniec poprawnej odpowiedzi pytania testowego 
\newcommand{\wstawGrafike}[2]{\begin{figure}[h] \includegraphics[scale=#2] {#1} \end{figure}} %gdyby była potrzeba wstawienia obrazka, parametry: nazwa pliku, skala (jak nie wiesz co wpisać, to wpisz 1)
\begin{document}
\maketitle
\kategoria{Beger/c1.5k}
\zadStart{Zadanie z Beger C 1.5k moja wersja nr [nrWersji]}
%[p1]:[2,3,4,5,6,7,8,9,10]
Obliczyć, całkując przez podstawienie.
$$\int \frac{[p1]}{\sin(x)\cos(x)\ln^2(\tan(x))} \,dx$$
\zadStop
\rozwStart{Radosław Grzyb}{}
Podstawiamy:
$$u=\tan(x)\implies du=\frac{1}{\cos^2(x)} dx \implies dx=\cos^2(x) du$$
A więc przekształcamy nieco naszą funkcję podcałkową i podstawiamy:
$$\int \frac{[p1]}{\sin(x)\cos(x)\ln^2(\tan(x))}\cdot\cos^2(x) \,du=[p1]\int \frac{\cos^2(x)}{\sin(x)\cos(x)\ln^2(\tan(x))}\,du=$$
$$=[p1]\int \frac{\cos(x)}{\sin(x)\ln^2(\tan(x))}\,du=[p1]\int \frac{\cot(x)}{\ln^2(\tan(x))}\,du=[p1]\int \frac{1}{\tan(x)\ln^2(\tan(x))}\,du$$
$$=[p1]\int \frac{1}{u\ln^2(u)}\,du$$
Powyższą całkę możemy obliczyć dokonując kolejnego podstawienia:
$$t=\ln(u)\implies dt=\frac{1}{u} du \implies du=u dt$$
A więc otrzymujemy:
$$[p1]\int \frac{1}{u\ln^2(u)}\cdot u\,dt=[p1]\int \frac{1}{t^2}\,dt=-\frac{[p1]}{t}+C$$
Zatem naszym końcowym wynikiem jest:
$$-\frac{[p1]}{t}+C=-\frac{[p1]}{\ln(u)}+C=-\frac{[p1]}{\ln(\tan(x))}+C$$
\rozwStop
\odpStart
$$-\frac{[p1]}{\ln(\tan(x))}+C$$
\odpStop
\testStart
A.$$-\frac{1}{[p1]\arctan^{[p1]}(x)}+C$$
B.$$\frac{[p1]}{11}\arctan\frac{x}{[p1]}+C$$
C.$$-\frac{[p1]}{\ln(\tan(x))}+C$$
D.$$\frac{1}{[p1]}\arcsin|\frac{x}{[p1]}|+C$$
\testStop
\kluczStart
C
\kluczStop
\end{document}