\documentclass[12pt, a4paper]{article}
\usepackage[utf8]{inputenc}
\usepackage{polski}

\usepackage{amsthm}  %pakiet do tworzenia twierdzeń itp.
\usepackage{amsmath} %pakiet do niektórych symboli matematycznych
\usepackage{amssymb} %pakiet do symboli mat., np. \nsubseteq
\usepackage{amsfonts}
\usepackage{graphicx} %obsługa plików graficznych z rozszerzeniem png, jpg
\theoremstyle{definition} %styl dla definicji
\newtheorem{zad}{} 
\title{Multizestaw zadań}
\author{Robert Fidytek}
%\date{\today}
\date{}
\newcounter{liczniksekcji}
\newcommand{\kategoria}[1]{\section{#1}} %olreślamy nazwę kateforii zadań
\newcommand{\zadStart}[1]{\begin{zad}#1\newline} %oznaczenie początku zadania
\newcommand{\zadStop}{\end{zad}}   %oznaczenie końca zadania
%Makra opcjonarne (nie muszą występować):
\newcommand{\rozwStart}[2]{\noindent \textbf{Rozwiązanie (autor #1 , recenzent #2): }\newline} %oznaczenie początku rozwiązania, opcjonarnie można wprowadzić informację o autorze rozwiązania zadania i recenzencie poprawności wykonania rozwiązania zadania
\newcommand{\rozwStop}{\newline}                                            %oznaczenie końca rozwiązania
\newcommand{\odpStart}{\noindent \textbf{Odpowiedź:}\newline}    %oznaczenie początku odpowiedzi końcowej (wypisanie wyniku)
\newcommand{\odpStop}{\newline}                                             %oznaczenie końca odpowiedzi końcowej (wypisanie wyniku)
\newcommand{\testStart}{\noindent \textbf{Test:}\newline} %ewentualne możliwe opcje odpowiedzi testowej: A. ? B. ? C. ? D. ? itd.
\newcommand{\testStop}{\newline} %koniec wprowadzania odpowiedzi testowych
\newcommand{\kluczStart}{\noindent \textbf{Test poprawna odpowiedź:}\newline} %klucz, poprawna odpowiedź pytania testowego (jedna literka): A lub B lub C lub D itd.
\newcommand{\kluczStop}{\newline} %koniec poprawnej odpowiedzi pytania testowego 
\newcommand{\wstawGrafike}[2]{\begin{figure}[h] \includegraphics[scale=#2] {#1} \end{figure}} %gdyby była potrzeba wstawienia obrazka, parametry: nazwa pliku, skala (jak nie wiesz co wpisać, to wpisz 1)

\begin{document}
\maketitle



\kategoria{Dymkowska,Beger/C1.15c}
\zadStart{Zadanie z Dymkowska,Beger C 1.15 c) moja wersja nr [nrWersji]}
%[a]:[2,4,6,8,10,12]
%[b]=[a]/2
Stosując odpowiednie metody całkowania, obliczyć całkę $\displaystyle \int [a]xe^{-x^2}(1-x^2) \ dx$
\zadStop
\rozwStart{Mirella Narewska}{}
$$\int [a]xe^{-x^2}(1-x^2) \ dx =[a]\int xe^{-x^2}(1-x^2) \ dx=$$
$$[a]\int \left( e^{-x^2}x-e^{-x^2}x^3\right) \ dx=[a]\int e^{-x^2}x \ dx +[a]\int -e^{-x^2}x^3 \ dx$$
$$\text{Całkujemy pierwszą całkę przez podstawienie:} u=-x^2\Rightarrow du=-2x  \ dx$$
$$=-[b]\int e^u \ du=[b] e^u +C=[b]e^{-x^2} +C$$
$$\text{Całkujemy drugą całkę przez podstawienie:} t=x^2 \Rightarrow dt=2x \ dx$$
$$[b]\int e^{-t}t \ dt$$
$$\text{Następnie całkujemy drugą całkę przez części:}$$
$$p=t \ v'=e{-t}$$
$$p'=1 \ v=-e^{-t}$$
$$=[b]\left( -te^{-t}+\int e^{-t}\ dt \right)=-[b]te^{-t}-[b]e^{-t}+C=[b]x^{2}e^-{x^2}-[b]e^{-x^2}+C$$
$$\text{Końcowy wynik:} [b]e^{-x^2} +[b]x^{2}e^{-x^2}-[b]e^{-x^2}+C=[b]x^{2}e^{-x^2}+C$$
\odpStart
$$[b]x^{2}e^{-x^2}+C$$
\odpStop
\testStart
A.$[b]x^{2}e^{-x^2}+C$
\\
B.$[be^{-x^2}+C$
\\
C.$[b]x^{2}$
\\
D.$[a]e^{x}$
\testStop
\kluczStart
A
\kluczStop

\end{document}