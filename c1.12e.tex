\documentclass[12pt, a4paper]{article}
\usepackage[utf8]{inputenc}
\usepackage{polski}
\usepackage{amsthm}  %pakiet do tworzenia twierdzeń itp.
\usepackage{amsmath} %pakiet do niektórych symboli matematycznych
\usepackage{amssymb} %pakiet do symboli mat., np. \nsubseteq
\usepackage{amsfonts}
\usepackage{graphicx} %obsługa plików graficznych z rozszerzeniem png, jpg
\theoremstyle{definition} %styl dla definicji
\newtheorem{zad}{} 
\title{Multizestaw zadań}
\author{Radosław Grzyb}
%\date{\today}
\date{}
\newcounter{liczniksekcji}
\newcommand{\kategoria}[1]{\section{#1}} %olreślamy nazwę kateforii zadań
\newcommand{\zadStart}[1]{\begin{zad}#1\newline} %oznaczenie początku zadania
\newcommand{\zadStop}{\end{zad}}   %oznaczenie końca zadania
%Makra opcjonarne (nie muszą występować):
\newcommand{\rozwStart}[2]{\noindent \textbf{Rozwiązanie (autor #1 , recenzent #2): }\newline} %oznaczenie początku rozwiązania, opcjonarnie można wprowadzić informację o autorze rozwiązania zadania i recenzencie poprawności wykonania rozwiązania zadania
\newcommand{\rozwStop}{\newline}                                            %oznaczenie końca rozwiązania
\newcommand{\odpStart}{\noindent \textbf{Odpowiedź:}\newline}    %oznaczenie początku odpowiedzi końcowej (wypisanie wyniku)
\newcommand{\odpStop}{\newline}                                             %oznaczenie końca odpowiedzi końcowej (wypisanie wyniku)
\newcommand{\testStart}{\noindent \textbf{Test:}\newline} %ewentualne możliwe opcje odpowiedzi testowej: A. ? B. ? C. ? D. ? itd.
\newcommand{\testStop}{\newline} %koniec wprowadzania odpowiedzi testowych
\newcommand{\kluczStart}{\noindent \textbf{Test poprawna odpowiedź:}\newline} %klucz, poprawna odpowiedź pytania testowego (jedna literka): A lub B lub C lub D itd.
\newcommand{\kluczStop}{\newline} %koniec poprawnej odpowiedzi pytania testowego 
\newcommand{\wstawGrafike}[2]{\begin{figure}[h] \includegraphics[scale=#2] {#1} \end{figure}} %gdyby była potrzeba wstawienia obrazka, parametry: nazwa pliku, skala (jak nie wiesz co wpisać, to wpisz 1)
\begin{document}
\maketitle
\kategoria{Beger/c1.12e}
\zadStart{Zadanie z Beger C 1.12e moja wersja nr [nrWersji]}
%[p1]:[2,3,5,6,7,9,10,11,13]
%[s1]=2*[p1]
Obliczyć całkę funkcji niewymiernej:
$$\int \frac{\sqrt{x}}{x+[p1]}\,dx$$
\zadStop
\rozwStart{Radosław Grzyb}{}
Dokonujemy podstawienia:
$$u=\sqrt{x}\implies du=\frac{1}{2\sqrt{x}}dx\implies dx=2\sqrt{x} du$$
$$u^2=x$$
A więc otrzymujemy:
$$\int \frac{\sqrt{x}}{x+[p1]}\cdot2\sqrt{x} \,du=2\int \frac{u^2}{u^2+[p1]}\,du$$
Przekształcamy nieco naszą funkcję podcałkową:
$$2\int \frac{u^2}{u^2+[p1]}\,du=2\int \frac{u^2+[p1]-[p1]}{u^2+[p1]}\,du=2\int1-\frac{[p1]}{u^2+[p1]}\,du=$$
$$=2\int1\,du-2\int\frac{[p1]}{u^2+[p1]}\,du$$
Mamy do policzenia dwie proste całki:
$$=2\int1\,du=2u+C=2\sqrt{x}+C$$
Do policzenia drugiej wykorzystamy gotowy wzór $\int\frac{dx}{x^2+a^2}=\frac{1}{a}\arctan\frac{x}{a}+C$
$$-2\int\frac{[p1]}{u^2+[p1]}\,du=-[s1]\int\frac{1}{u^2+[p1]}\,du=-[s1]\cdot\frac{1}{\sqrt{[p1]}}\arctan\frac{u}{\sqrt{[p1]}}+C$$
Otrzymujemy finalny wynik:
$$-\frac{[s1]}{\sqrt{[p1]}}\arctan\frac{\sqrt{x}}{\sqrt{[p1]}}+C$$
\rozwStop
\odpStart
$$-\frac{[s1]}{\sqrt{[p1]}}\arctan\frac{\sqrt{x}}{\sqrt{[p1]}}+C$$
\odpStop
\testStart
A.$$-\frac{1}{[p1]\arctan^{[s1]}(x)}+C$$
B.$$-\frac{[s1]}{\sqrt{[p1]}}\arctan\frac{x}{\sqrt{[p1]}}+C$$
C.$$\frac{[s1]}{\sqrt{[p1]}}\arctan\frac{\sqrt{x}}{\sqrt{[p1]}}+C$$
D.$$-\frac{[s1]}{\sqrt{[p1]}}\arctan\frac{\sqrt{x}}{\sqrt{[p1]}}+C$$
\testStop
\kluczStart
D
\kluczStop
\end{document}