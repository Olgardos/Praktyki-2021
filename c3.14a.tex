\documentclass[12pt, a4paper]{article}
\usepackage[utf8]{inputenc}
\usepackage{polski}

\usepackage{amsthm}  %pakiet do tworzenia twierdzeń itp.
\usepackage{amsmath} %pakiet do niektórych symboli matematycznych
\usepackage{amssymb} %pakiet do symboli mat., np. \nsubseteq
\usepackage{amsfonts}
\usepackage{graphicx} %obsługa plików graficznych z rozszerzeniem png, jpg
\theoremstyle{definition} %styl dla definicji
\newtheorem{zad}{} 
\title{Multizestaw zadań}
\author{Robert Fidytek}
%\date{\today}
\date{}
\newcounter{liczniksekcji}
\newcommand{\kategoria}[1]{\section{#1}} %olreślamy nazwę kateforii zadań
\newcommand{\zadStart}[1]{\begin{zad}#1\newline} %oznaczenie początku zadania
\newcommand{\zadStop}{\end{zad}}   %oznaczenie końca zadania
%Makra opcjonarne (nie muszą występować):
\newcommand{\rozwStart}[2]{\noindent \textbf{Rozwiązanie (autor #1 , recenzent #2): }\newline} %oznaczenie początku rozwiązania, opcjonarnie można wprowadzić informację o autorze rozwiązania zadania i recenzencie poprawności wykonania rozwiązania zadania
\newcommand{\rozwStop}{\newline}                                            %oznaczenie końca rozwiązania
\newcommand{\odpStart}{\noindent \textbf{Odpowiedź:}\newline}    %oznaczenie początku odpowiedzi końcowej (wypisanie wyniku)
\newcommand{\odpStop}{\newline}                                             %oznaczenie końca odpowiedzi końcowej (wypisanie wyniku)
\newcommand{\testStart}{\noindent \textbf{Test:}\newline} %ewentualne możliwe opcje odpowiedzi testowej: A. ? B. ? C. ? D. ? itd.
\newcommand{\testStop}{\newline} %koniec wprowadzania odpowiedzi testowych
\newcommand{\kluczStart}{\noindent \textbf{Test poprawna odpowiedź:}\newline} %klucz, poprawna odpowiedź pytania testowego (jedna literka): A lub B lub C lub D itd.
\newcommand{\kluczStop}{\newline} %koniec poprawnej odpowiedzi pytania testowego 
\newcommand{\wstawGrafike}[2]{\begin{figure}[h] \includegraphics[scale=#2] {#1} \end{figure}} %gdyby była potrzeba wstawienia obrazka, parametry: nazwa pliku, skala (jak nie wiesz co wpisać, to wpisz 1)

\begin{document}
\maketitle


\kategoria{Dymkowska, Beger/c3.14a}
\zadStart{Zadanie z Dymkowskiej, Beger C 3.14a) moja wersja nr [nrWersji]}
%[p1]:[2,3,4,6,7,8,9]
%[p2]=[p1]*[p1]
%[p3]=[p2]*[p2]
%[wyn]=[p3]*[p1]*10-2*pow([p1],5)
Za pomocą całki podwójnej obliczyć objętość bryły ograniczonej powierzchniami $$y=x^2, y=[p2], z=0, z=2y$$
\zadStop
\rozwStart{Jakub Janik}{}
Objętość V możemy zapisać jako
$$V=\{(x,y,z)\in\mathbb{R}^3\colon(x,y)\in D, 0 \leq z \leq 2y\}$$
Przy czym obszar D wyraża się jako
$$D\colon-[p1]\leq x \leq [p1], x^2 \leq y \leq [p2]$$
Przechodzimy do obliczenia objętości
$$\iint_D 2y\ dxdy=\int_{-[p1]}^{[p1]}dx\int_{x^2}^{[p2]}2y\ dy=\int_{-[p1]}^{[p1]}([p3]-x^4)dx=$$
$$=[p3]x-\frac{1}{5}x^5\Big|_{-[p1]}^{[p1]}=\frac{[wyn]}{5}$$
\rozwStop
\odpStart
$\frac{[wyn]}{5}$
\odpStop
\testStart
A.$\frac{[wyn]}{5}$
B.$0$
C.$-\frac{[wyn]}{5}$
D.$\infty$
\testStop
\kluczStart
A
\kluczStop



\end{document}