\documentclass[12pt, a4paper]{article}
\usepackage[utf8]{inputenc}
\usepackage{polski}

\usepackage{amsthm}  %pakiet do tworzenia twierdzeń itp.
\usepackage{amsmath} %pakiet do niektórych symboli matematycznych
\usepackage{amssymb} %pakiet do symboli mat., np. \nsubseteq
\usepackage{amsfonts}
\usepackage{graphicx} %obsługa plików graficznych z rozszerzeniem png, jpg
\theoremstyle{definition} %styl dla definicji
\newtheorem{zad}{} 
\title{Multizestaw zadań}
\author{Robert Fidytek}
%\date{\today}
\date{}
\newcounter{liczniksekcji}
\newcommand{\kategoria}[1]{\section{#1}} %olreślamy nazwę kateforii zadań
\newcommand{\zadStart}[1]{\begin{zad}#1\newline} %oznaczenie początku zadania
\newcommand{\zadStop}{\end{zad}}   %oznaczenie końca zadania
%Makra opcjonarne (nie muszą występować):
\newcommand{\rozwStart}[2]{\noindent \textbf{Rozwiązanie (autor #1 , recenzent #2): }\newline} %oznaczenie początku rozwiązania, opcjonarnie można wprowadzić informację o autorze rozwiązania zadania i recenzencie poprawności wykonania rozwiązania zadania
\newcommand{\rozwStop}{\newline}                                            %oznaczenie końca rozwiązania
\newcommand{\odpStart}{\noindent \textbf{Odpowiedź:}\newline}    %oznaczenie początku odpowiedzi końcowej (wypisanie wyniku)
\newcommand{\odpStop}{\newline}                                             %oznaczenie końca odpowiedzi końcowej (wypisanie wyniku)
\newcommand{\testStart}{\noindent \textbf{Test:}\newline} %ewentualne możliwe opcje odpowiedzi testowej: A. ? B. ? C. ? D. ? itd.
\newcommand{\testStop}{\newline} %koniec wprowadzania odpowiedzi testowych
\newcommand{\kluczStart}{\noindent \textbf{Test poprawna odpowiedź:}\newline} %klucz, poprawna odpowiedź pytania testowego (jedna literka): A lub B lub C lub D itd.
\newcommand{\kluczStop}{\newline} %koniec poprawnej odpowiedzi pytania testowego 
\newcommand{\wstawGrafike}[2]{\begin{figure}[h] \includegraphics[scale=#2] {#1} \end{figure}} %gdyby była potrzeba wstawienia obrazka, parametry: nazwa pliku, skala (jak nie wiesz co wpisać, to wpisz 1)

\begin{document}
\maketitle


\kategoria{Wikieł/Z1.9c}
\zadStart{Zadanie z Wikieł Z 1.9 c) moja wersja nr [nrWersji]}
%[a]:[2,3,5,7,8,11,13,14,15,17,18]
%[b]:[2,3,4,5,6,7,8]
%[c]=3*[a]
%[d]=[a]*[a]
%[e]=[b]*[d]
%[f]=2*[c]-[e]
%[g]=2+[b]
%[f]!=0
Uprościć wyrażenie: $
(x+\sqrt{[a]})^3+[b]x(x+[a])(x-[a])+(x-\sqrt{[a]})^3$.
\zadStop
\rozwStart{Klaudia Klejdysz}{}
W poniższym zadaniu korzystać będziemy z dwumianu Newtona:
\begin{equation*}
(x+y)^n=\sum_{k=0}^n {{n}\choose {k} }x^{n-k}y^k   
\end{equation*}
\noindent Rozłóżmy wyrażenie na poszczególne składniki sumy:
\begin{equation*}
    (x+\sqrt{[a]})^3=(\sqrt{[a]}+x)^3=\sum_{k=0}^3 {{3}\choose {k} }x^{k}(\sqrt{[a]})^{3-k}=
    \end{equation*}
    \begin{equation*}
    ={{3}\choose {0} }(\sqrt{[a]})^3+{{3}\choose {1} }x(\sqrt{[a]})^2+{{3}\choose {2} }x^2(\sqrt{[a]})+{{3}\choose {3} }x^3=
\end{equation*}
\begin{equation*}
=1\cdot[a]\sqrt{[a]}+3\cdot[a]x+3\cdot\sqrt{[a]}x^2+1\cdot x^3=
\end{equation*}
\begin{equation*}
=[a]\sqrt{[a]}+[c]x+3\sqrt{[a]}x^2+x^3
\end{equation*}
\\
\begin{equation*}
[b]x(x+[a])(x-[a])=[b]x(x^2-[d])=[b]x^3-[e]x    
\end{equation*}
\\
\begin{equation*}
 (x-\sqrt{[a]})^3=\sum_{k=0}^3 {{3}\choose {k} }x^{3-k}(-\sqrt{[a]})^{k}=   
\end{equation*}
\begin{equation*}
 ={{3}\choose {0} }x^3(-\sqrt{[a]})^0+{{3}\choose {1} }x^2(-\sqrt{[a]})^1+{{3}\choose {2} }x^1(-\sqrt{[a]})^2+{{3}\choose {3} }x^0(-\sqrt{[a]})^3=
\end{equation*}
\begin{equation*}
x^3-3\sqrt{[a]}x^2+[c]x-[a]\sqrt{[a]}
\end{equation*}
Podsumowując otrzymujemy:
\begin{equation*}
(x+\sqrt{[a]})^3+[b]x(x+[a])(x-[a])+(x-\sqrt{[a]})^3=
\end{equation*}
\begin{equation*}
=[a]\sqrt{[a]}+[c]x+3\sqrt{[a]}x^2+x^3+ [b]x^3-[e]x+x^3-3\sqrt{[a]}x^2+[c]x-[a]\sqrt{[a]}=
\end{equation*}
\begin{equation*}
    =[g]x^3+([f]x)
\end{equation*}
\rozwStop
\odpStart
$[g]x^3+([f])x$.
\odpStop
\testStart
A.$[g]x^3+([f]x)$\\
B.$[f]x^3$\\
C.$0$\\
D.$[a]\sqrt{[a]}$
\testStop
\kluczStart
A
\kluczStop



\end{document}