\documentclass[12pt, a4paper]{article}
\usepackage[utf8]{inputenc}
\usepackage{polski}

\usepackage{amsthm}  %pakiet do tworzenia twierdzeń itp.
\usepackage{amsmath} %pakiet do niektórych symboli matematycznych
\usepackage{amssymb} %pakiet do symboli mat., np. \nsubseteq
\usepackage{amsfonts}
\usepackage{graphicx} %obsługa plików graficznych z rozszerzeniem png, jpg
\theoremstyle{definition} %styl dla definicji
\newtheorem{zad}{} 
\title{Multizestaw zadań}
\author{Robert Fidytek}
%\date{\today}
\date{}
\newcounter{liczniksekcji}
\newcommand{\kategoria}[1]{\section{#1}} %olreślamy nazwę kateforii zadań
\newcommand{\zadStart}[1]{\begin{zad}#1\newline} %oznaczenie początku zadania
\newcommand{\zadStop}{\end{zad}}   %oznaczenie końca zadania
%Makra opcjonarne (nie muszą występować):
\newcommand{\rozwStart}[2]{\noindent \textbf{Rozwiązanie (autor #1 , recenzent #2): }\newline} %oznaczenie początku rozwiązania, opcjonarnie można wprowadzić informację o autorze rozwiązania zadania i recenzencie poprawności wykonania rozwiązania zadania
\newcommand{\rozwStop}{\newline}                                            %oznaczenie końca rozwiązania
\newcommand{\odpStart}{\noindent \textbf{Odpowiedź:}\newline}    %oznaczenie początku odpowiedzi końcowej (wypisanie wyniku)
\newcommand{\odpStop}{\newline}                                             %oznaczenie końca odpowiedzi końcowej (wypisanie wyniku)
\newcommand{\testStart}{\noindent \textbf{Test:}\newline} %ewentualne możliwe opcje odpowiedzi testowej: A. ? B. ? C. ? D. ? itd.
\newcommand{\testStop}{\newline} %koniec wprowadzania odpowiedzi testowych
\newcommand{\kluczStart}{\noindent \textbf{Test poprawna odpowiedź:}\newline} %klucz, poprawna odpowiedź pytania testowego (jedna literka): A lub B lub C lub D itd.
\newcommand{\kluczStop}{\newline} %koniec poprawnej odpowiedzi pytania testowego 
\newcommand{\wstawGrafike}[2]{\begin{figure}[h] \includegraphics[scale=#2] {#1} \end{figure}} %gdyby była potrzeba wstawienia obrazka, parametry: nazwa pliku, skala (jak nie wiesz co wpisać, to wpisz 1)

\begin{document}
\maketitle



\kategoria{Dymkowska,Beger/C2.2i}
\zadStart{Zadanie z Dymkowska,Beger C 2.2 i) moja wersja nr [nrWersji]}
%[a]:[2,4,6,8,10,12]
%[b1]=[a]/2
%[b]=int([b1])
Obliczyć całkę oznaczoną $\displaystyle \int_{-\pi}^{\pi} [a]\cos^2{x}  \ dx $
\zadStop
\rozwStart{Mirella Narewska}{}
$$\int_{-\pi}^{\pi} [a]\cos^2{x}  \ dx= [a] \int_{-\pi}^{\pi} \cos^2{x}  \ dx =[a] \int_{-\pi}^{\pi} \left(\frac{1}{2}\cos{2x} + \frac{1}{2} \right) \ dx$$
$$=[b]\int_{-\pi}^{\pi} \cos{2x}  \ dx + [b]\int_{-\pi}^{\pi} 1  \ dx$$
$$\text{Całkujemy pierwszą całkę przez podstawienie: }$$
$$u=2x \Rightarrow du=2 dx $$
$$\text{Zmieniają się także granice całkowania: }[b] \int_{-2\pi}^{2\pi} \cos{u}  \ dx=[b]\sin{u}|_{-2\pi}^{2\pi}=0$$
$$\text{Z drugiej całki otrzymamy: } [b]\int_{-\pi}^{\pi} 1  \ dx=[b]x|_{-\pi}^{\pi}=[a]\pi$$
\odpStart
$$[a]\pi$$
\odpStop
\testStart
A.$[a]\pi$
\\
B.$\pi$
\\
C.$0$
\\
D.$cos([a])$
\testStop
\kluczStart
A
\kluczStop


\end{document}