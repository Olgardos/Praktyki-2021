\documentclass[12pt, a4paper]{article}
\usepackage[utf8]{inputenc}
\usepackage{polski}

\usepackage{amsthm}  %pakiet do tworzenia twierdzeń itp.
\usepackage{amsmath} %pakiet do niektórych symboli matematycznych
\usepackage{amssymb} %pakiet do symboli mat., np. \nsubseteq
\usepackage{amsfonts}
\usepackage{graphicx} %obsługa plików graficznych z rozszerzeniem png, jpg
\theoremstyle{definition} %styl dla definicji
\newtheorem{zad}{} 
\title{Multizestaw zadań}
\author{Robert Fidytek}
%\date{\today}
\date{}
\newcounter{liczniksekcji}
\newcommand{\kategoria}[1]{\section{#1}} %olreślamy nazwę kateforii zadań
\newcommand{\zadStart}[1]{\begin{zad}#1\newline} %oznaczenie początku zadania
\newcommand{\zadStop}{\end{zad}}   %oznaczenie końca zadania
%Makra opcjonarne (nie muszą występować):
\newcommand{\rozwStart}[2]{\noindent \textbf{Rozwiązanie (autor #1 , recenzent #2): }\newline} %oznaczenie początku rozwiązania, opcjonarnie można wprowadzić informację o autorze rozwiązania zadania i recenzencie poprawności wykonania rozwiązania zadania
\newcommand{\rozwStop}{\newline}                                            %oznaczenie końca rozwiązania
\newcommand{\odpStart}{\noindent \textbf{Odpowiedź:}\newline}    %oznaczenie początku odpowiedzi końcowej (wypisanie wyniku)
\newcommand{\odpStop}{\newline}                                             %oznaczenie końca odpowiedzi końcowej (wypisanie wyniku)
\newcommand{\testStart}{\noindent \textbf{Test:}\newline} %ewentualne możliwe opcje odpowiedzi testowej: A. ? B. ? C. ? D. ? itd.
\newcommand{\testStop}{\newline} %koniec wprowadzania odpowiedzi testowych
\newcommand{\kluczStart}{\noindent \textbf{Test poprawna odpowiedź:}\newline} %klucz, poprawna odpowiedź pytania testowego (jedna literka): A lub B lub C lub D itd.
\newcommand{\kluczStop}{\newline} %koniec poprawnej odpowiedzi pytania testowego 
\newcommand{\wstawGrafike}[2]{\begin{figure}[h] \includegraphics[scale=#2] {#1} \end{figure}} %gdyby była potrzeba wstawienia obrazka, parametry: nazwa pliku, skala (jak nie wiesz co wpisać, to wpisz 1)

\begin{document}
\maketitle

\kategoria{Wikieł/Z5.26n}

\zadStart{Zadanie z Wikieł Z 5.26 n) moja wersja nr [nrWersji]}
%[a]:[1,2,3,4,5,6,7,8,9,10,11]
%[b]:[1,2,3,4,5,6,7,8,9,10,11]
%[c]=[b]+1
%[d]=[b]*[c]
%[e]=[a]*[c] + [a]*[b]
%[f]=[b]+2
%[g]=[c]*[f]
%[h]=[a]*[f] + [a]*[c]
%[i]=[e]/[d]
%[j]=round([i],2)
%[k]=[h]/[g]
%[l]=round([k],2)
%[m]=math.gcd([e],[d])
%[n]=int([e]/[m])
%[o]=int([d]/[m])
%[p]=math.gcd([h],[g])
%[r]=int([h]/[p])
%[s]=int([g]/[p])
%math.gcd([b],[c])==1 and math.gcd([c],[f])==1 and [j]>[l]
Wyznaczyć wartość największą oraz wartość najmniejszą funkcji w przedziale. 
$$y = \frac{[a]}{[b] + \mid x \mid} + \frac{[a]}{[b] + \mid x - 1 \mid}, \quad \langle -1,2 \rangle$$
\zadStop

\rozwStart{Natalia Danieluk}{}
Wartość bezwzględna nie jest różniczkowalna w zerze. Wartość największą $M$ i najmniejszą $m$ znajdziemy więc w $0$ i $1$ oraz na końcach przedziału, tj. $f(-1)$ i $f(2)$. \\
Sprawdzamy wartości funkcji w tych punktach oraz na końcach naszego przedziału: \\
$$ f(0) = \frac{[a]}{[b]} + \frac{[a]}{[c]} = \frac{[n]}{[o]} \approx [j], \quad f(1) = \frac{[a]}{[c]} + \frac{[a]}{[b]} = \frac{[n]}{[o]}, $$
$$ f(-1) = \frac{[a]}{[c]} + \frac{[a]}{[f]} = \frac{[r]}{[s]} \approx [l], \quad f(2) = \frac{[a]}{[f]} + \frac{[a]}{[c]} = \frac{[r]}{[s]} $$
\rozwStop

\odpStart
Wartość największa $M$ funkcji $f$ w przedziale $\langle -1,2 \rangle$ to $\frac{[n]}{[o]}$, natomiast wartość najmniejsza $m$ to $\frac{[r]}{[s]}$.
\odpStop

\testStart
A. $M=[a], m=-[a]$
B. $M=\frac{[r]}{[s]}, m=\frac{[n]}{[o]}$
C. $M=\frac{[n]}{[o]}, m=\frac{[r]}{[s]}$
D. $M=-1, m=2$
\testStop

\kluczStart
C
\kluczStop

\end{document}