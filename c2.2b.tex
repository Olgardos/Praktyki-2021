\documentclass[12pt, a4paper]{article}
\usepackage[utf8]{inputenc}
\usepackage{polski}
\usepackage{amsthm}  %pakiet do tworzenia twierdzeń itp.
\usepackage{amsmath} %pakiet do niektórych symboli matematycznych
\usepackage{amssymb} %pakiet do symboli mat., np. \nsubseteq
\usepackage{amsfonts}
\usepackage{graphicx} %obsługa plików graficznych z rozszerzeniem png, jpg
\theoremstyle{definition} %styl dla definicji
\newtheorem{zad}{} 
\title{Multizestaw zadań}
\author{Radosław Grzyb}
%\date{\today}
\date{}
\newcounter{liczniksekcji}
\newcommand{\kategoria}[1]{\section{#1}} %olreślamy nazwę kateforii zadań
\newcommand{\zadStart}[1]{\begin{zad}#1\newline} %oznaczenie początku zadania
\newcommand{\zadStop}{\end{zad}}   %oznaczenie końca zadania
%Makra opcjonarne (nie muszą występować):
\newcommand{\rozwStart}[2]{\noindent \textbf{Rozwiązanie (autor #1 , recenzent #2): }\newline} %oznaczenie początku rozwiązania, opcjonarnie można wprowadzić informację o autorze rozwiązania zadania i recenzencie poprawności wykonania rozwiązania zadania
\newcommand{\rozwStop}{\newline}                                            %oznaczenie końca rozwiązania
\newcommand{\odpStart}{\noindent \textbf{Odpowiedź:}\newline}    %oznaczenie początku odpowiedzi końcowej (wypisanie wyniku)
\newcommand{\odpStop}{\newline}                                             %oznaczenie końca odpowiedzi końcowej (wypisanie wyniku)
\newcommand{\testStart}{\noindent \textbf{Test:}\newline} %ewentualne możliwe opcje odpowiedzi testowej: A. ? B. ? C. ? D. ? itd.
\newcommand{\testStop}{\newline} %koniec wprowadzania odpowiedzi testowych
\newcommand{\kluczStart}{\noindent \textbf{Test poprawna odpowiedź:}\newline} %klucz, poprawna odpowiedź pytania testowego (jedna literka): A lub B lub C lub D itd.
\newcommand{\kluczStop}{\newline} %koniec poprawnej odpowiedzi pytania testowego 
\newcommand{\wstawGrafike}[2]{\begin{figure}[h] \includegraphics[scale=#2] {#1} \end{figure}} %gdyby była potrzeba wstawienia obrazka, parametry: nazwa pliku, skala (jak nie wiesz co wpisać, to wpisz 1)
\begin{document}
\maketitle
\kategoria{Beger/c2.2b}
\zadStart{Zadanie z Beger C 2.2b moja wersja nr [nrWersji]}
%[p1]:[1,2,3,4,5,6,7,8,9]
%[p2]:[0,1,2,3,4,5,6,7,8]
%[w1]=2/3*[p1]**(3/2)+2*[p1]**(1/2)-(2/3*[p2]**(3/2)+2*[p2]**(1/2))
%[zw1]=1000*[w1]
%[zw1]=10*[w1]
%[zw2]=-[w1]
%[zw3]=7+[w1]
%[p1]>[p2] and ([zw1]).is_integer() is True
Obliczyć całkę oznaczoną
$$\int_{[p2]}^{[p1]} \frac{x+1}{\sqrt{x}} \,dx$$
\zadStop
\rozwStart{Radosław Grzyb}{}
$$\int_{[p2]}^{[p1]} \frac{x+1}{\sqrt{x}} \,dx = \int_{[p2]}^{[p1]} \sqrt{x}+\frac{1}{\sqrt{x}} \,dx = \int_{[p2]}^{[p1]} x^{\frac{1}{2}}+x^{-\frac{1}{2}} \,dx$$
Całka nieoznaczona z naszej funkcji wynosi: $\frac{2}{3}x^{\frac{3}{2}}+2x^{\frac{1}{2}}$\\
Podstawiając $[p1]$ oraz $[p2]$ w miejsce $x$ otrzymujemy:
$$\frac{2}{3}[p1]^{\frac{3}{2}}+2\cdot[p1]^{\frac{1}{2}}-(\frac{2}{3}[p2]^{\frac{3}{2}}+2\cdot[p2]^{\frac{1}{2}})=[w]$$
\rozwStop
\odpStart
$[w]$
\odpStop
\testStart
A.$[zw1]$
B.$[w]$
C.$[zw3]$
D.$[zw2]$
\testStop
\kluczStart
B
\kluczStop
\end{document}
