\documentclass[12pt, a4paper]{article}
\usepackage[utf8]{inputenc}
\usepackage{polski}

\usepackage{amsthm}  %pakiet do tworzenia twierdzeń itp.
\usepackage{amsmath} %pakiet do niektórych symboli matematycznych
\usepackage{amssymb} %pakiet do symboli mat., np. \nsubseteq
\usepackage{amsfonts}
\usepackage{graphicx} %obsługa plików graficznych z rozszerzeniem png, jpg
\theoremstyle{definition} %styl dla definicji
\newtheorem{zad}{} 
\title{Multizestaw zadań}
\author{Robert Fidytek}
%\date{\today}
\date{}
\newcounter{liczniksekcji}
\newcommand{\kategoria}[1]{\section{#1}} %olreślamy nazwę kateforii zadań
\newcommand{\zadStart}[1]{\begin{zad}#1\newline} %oznaczenie początku zadania
\newcommand{\zadStop}{\end{zad}}   %oznaczenie końca zadania
%Makra opcjonarne (nie muszą występować):
\newcommand{\rozwStart}[2]{\noindent \textbf{Rozwiązanie (autor #1 , recenzent #2): }\newline} %oznaczenie początku rozwiązania, opcjonarnie można wprowadzić informację o autorze rozwiązania zadania i recenzencie poprawności wykonania rozwiązania zadania
\newcommand{\rozwStop}{\newline}                                            %oznaczenie końca rozwiązania
\newcommand{\odpStart}{\noindent \textbf{Odpowiedź:}\newline}    %oznaczenie początku odpowiedzi końcowej (wypisanie wyniku)
\newcommand{\odpStop}{\newline}                                             %oznaczenie końca odpowiedzi końcowej (wypisanie wyniku)
\newcommand{\testStart}{\noindent \textbf{Test:}\newline} %ewentualne możliwe opcje odpowiedzi testowej: A. ? B. ? C. ? D. ? itd.
\newcommand{\testStop}{\newline} %koniec wprowadzania odpowiedzi testowych
\newcommand{\kluczStart}{\noindent \textbf{Test poprawna odpowiedź:}\newline} %klucz, poprawna odpowiedź pytania testowego (jedna literka): A lub B lub C lub D itd.
\newcommand{\kluczStop}{\newline} %koniec poprawnej odpowiedzi pytania testowego 
\newcommand{\wstawGrafike}[2]{\begin{figure}[h] \includegraphics[scale=#2] {#1} \end{figure}} %gdyby była potrzeba wstawienia obrazka, parametry: nazwa pliku, skala (jak nie wiesz co wpisać, to wpisz 1)

\begin{document}
\maketitle


\kategoria{Wikieł/Z1.36r}
\zadStart{Zadanie z Wikieł Z 1.36 r) moja wersja nr [nrWersji]}
%[a]:[2,3,4,8,9,10,11,12,13]
%[b]:[2,3,4,5,6,7,8,9]
%[c]:[6,7,8,9,10,11,12,13,14,15,16,17]
%[d]:[3,4,5,6,7,8,9]
%[e]=[b]-4*[c]
%[e]<0
%[f]=[a]-[b]+1
%[g]=[c]+[d]
%[h]=[b]-[a]-1
%[i]=[h]^2-4*2*[g]
%[j]=[a]-[b]+1
%[k]=[a]-[b]-1
%[l]=[c]-[d]
%math.gcd([f],[g])==1 and math.gcd([j],[g])==1 and math.gcd([l],[k])==1 and [g]/[j]>[d] and [k]!=0 and [l]/[k]>[d]
Rozwiązać równanie: $|x^2+[a]x|=|x^2+[b]x+[c]|+|x-[d]|$.
\zadStop
\rozwStart{Klaudia Klejdysz}{}
Szukamy miejsc zerowych wyrażeń ujętych w wartościach bezwględnych:
$$|x^2+[a]x|=0$$
$$x^2+[a]x=0$$
$$x(x+[a])=0$$
$$x=0\text{ }\lor\text{ }x=-[a]$$\\
$$|x^2+[b]x+[c]|=0$$
$$x^2+[b]x+[c]=0$$
$$\Delta=[b]^2-4*1*[c]=[e]<0\text{ (brak miejsc zerowych)}$$\\
$$|x-[d]|=0$$
$$x=[d]$$\\
\indent Przypadek 1: $x\in(-\infty,-[a])$
$$x^2+[a]x=x^2+[b]x+[c]-(x-[d])$$
$$[a]x=[b]x+[c]-x+[d]$$
$$([a]-[b]+1)x=[c]+[d]$$
$$[f]x=[g]$$
$$x=\frac{[g]}{[f]}$$
Ostatecznie otrzymujemy:
$$x\in(-\infty,-[a])\qquad\land\qquad x=\frac{[g]}{[f]}$$
Czyli z przypadku 1 otrzymujemy sprzeczność, a więc: $x\in\emptyset$.\\


Przypadek 2: $x\in[-[a],0]$ $$-(x^2+[a]x)=x^2+[b]x+[c]-(x-[d])$$
$$-x^2+[a]x=x^2+[b]x+[c]-x+[d]$$
$$2x^2+([b]-[a]-1)x+[g]=0$$
$$2x^2+[h]x+[g]=0$$
$$\Delta=[h]^2-4*2*[g]=[i]<0\text{ (brak miejsc zerowych)}$$\\
Czyli z przypadku 2 otrzymujemy sprzeczność, a więc: $x\in\emptyset$.\\


Przypadek 3: $x\in(0,[d])$ $$x^2+[a]x=x^2+[b]x+[c]-(x-[d])$$
$$[a]x=[b]x+[c]-x+[d]$$
$$([a]-[b]+1)x=[g]$$
$$[j]x=[g]$$
$$x=\frac{[g]}{[j]}$$
Ostatecznie otrzymujemy:
$$x\in(0,[d])\qquad\land\qquad x=\frac{[g]}{[j]}$$
Czyli z przypadku 3 otrzymujemy sprzeczność, a więc: $x\in\emptyset$.\\


Przypadek 4: $x\in[[d],\infty)$ $$x^2+[a]x=x^2+[b]x+[c]+(x-[d])$$
$$([a]-[b]-1)x=[c]-[d]$$
$$[k]x=[l]$$
$$x=\frac{[l]}{[k]}$$
Ostatecznie otrzymujemy:
$$x\in[[d],\infty)\qquad\land\qquad x=\frac{[l]}{[k]}$$
Czyli z przypadku 4 mamy: $x=\frac{[l]}{[k]}$.\\
Ostatecznym rozwiązaniem wyjściowego równania jest:
$$x\in\bigg\{\frac{[l]}{[k]}\bigg\}$$
\rozwStop
\odpStart
$x\in\bigg\{\frac{[l]}{[k]}\bigg\}$
\odpStop
\testStart
A. $x\in\bigg\{\frac{[l]}{[k]}\bigg\}$\\
B. $x\in\mathbb{R}$\\
C. $x\in\emptyset$\\
D. $x\in\bigg\{\frac{[k]}{[j]}\bigg\}$
\testStop
\kluczStart
A
\kluczStop



\end{document}

