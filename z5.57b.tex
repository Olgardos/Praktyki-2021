\documentclass[12pt, a4paper]{article}
\usepackage[utf8]{inputenc}
\usepackage{polski}

\usepackage{amsthm}  %pakiet do tworzenia twierdzeń itp.
\usepackage{amsmath} %pakiet do niektórych symboli matematycznych
\usepackage{amssymb} %pakiet do symboli mat., np. \nsubseteq
\usepackage{amsfonts}
\usepackage{graphicx} %obsługa plików graficznych z rozszerzeniem png, jpg
\theoremstyle{definition} %styl dla definicji
\newtheorem{zad}{} 
\title{Multizestaw zadań}
\author{Robert Fidytek}
%\date{\today}
\date{}
\newcounter{liczniksekcji}
\newcommand{\kategoria}[1]{\section{#1}} %olreślamy nazwę kateforii zadań
\newcommand{\zadStart}[1]{\begin{zad}#1\newline} %oznaczenie początku zadania
\newcommand{\zadStop}{\end{zad}}   %oznaczenie końca zadania
%Makra opcjonarne (nie muszą występować):
\newcommand{\rozwStart}[2]{\noindent \textbf{Rozwiązanie (autor #1 , recenzent #2): }\newline} %oznaczenie początku rozwiązania, opcjonarnie można wprowadzić informację o autorze rozwiązania zadania i recenzencie poprawności wykonania rozwiązania zadania
\newcommand{\rozwStop}{\newline}                                            %oznaczenie końca rozwiązania
\newcommand{\odpStart}{\noindent \textbf{Odpowiedź:}\newline}    %oznaczenie początku odpowiedzi końcowej (wypisanie wyniku)
\newcommand{\odpStop}{\newline}                                             %oznaczenie końca odpowiedzi końcowej (wypisanie wyniku)
\newcommand{\testStart}{\noindent \textbf{Test:}\newline} %ewentualne możliwe opcje odpowiedzi testowej: A. ? B. ? C. ? D. ? itd.
\newcommand{\testStop}{\newline} %koniec wprowadzania odpowiedzi testowych
\newcommand{\kluczStart}{\noindent \textbf{Test poprawna odpowiedź:}\newline} %klucz, poprawna odpowiedź pytania testowego (jedna literka): A lub B lub C lub D itd.
\newcommand{\kluczStop}{\newline} %koniec poprawnej odpowiedzi pytania testowego 
\newcommand{\wstawGrafike}[2]{\begin{figure}[h] \centering \includegraphics[scale=#2] {#1} \end{figure}} %gdyby była potrzeba wstawienia obrazka, parametry: nazwa pliku, skala (jak nie wiesz co wpisać, to wpisz 1)

\begin{document}
\maketitle

\kategoria{Wikieł/Z5.57b}

\zadStart{Zadanie z Wikieł Z 5.57 b) moja wersja nr [nrWersji]}
%[a]:[2,3,4,5,6,7,8,9,10,11]
Wyznaczyć przedziały, w których funkcja:
$$f(x) = \frac{[a]x}{\ln x}$$
jest jednocześnie malejąca i wypukła w dół.
\zadStop

\rozwStart{Natalia Danieluk}{}
Postępujemy następująco:
\begin{enumerate}
\item Określamy dziedzinę funkcji: $\quad \mathcal{D}_f=\mathbb{R_+}\backslash\{1\}$. \\
\item Obliczamy pochodne: 
$$\quad f'(x) = \frac{[a](\ln x-1)}{\ln^2 x},\quad f''(x) = -\frac{[a](\ln x - 2)}{x \ln^3 x}$$
i określamy ich dziedziny: $\quad \mathcal{D}_{f'}=\mathcal{D}_{f''}=\mathbb{R_+}\backslash\{1\}$.\\
\item Badamy znak $f'$. \\
$$ f'(x) = 0 \Leftrightarrow \ln x = 1 \Leftrightarrow x = e $$ 
	\begin{enumerate}
	\item $f'(x) > 0 \Leftrightarrow x \in (e,\infty)$ i w tym przedziale funkcja $f$ jest rosnąca \\
	\item $f'(x) < 0 \Leftrightarrow x \in (0,1)\cup(1,e)$ i w tym przedziale funkcja $f$ jest malejąca 
	\end{enumerate}
\item Badamy znak $f''$. \\
$$ f''(x) = 0 \Leftrightarrow \ln x = 2 \Leftrightarrow x = e^2 $$ 
	\begin{enumerate}
	\item $f''(x) > 0 \Leftrightarrow x \in (1,e^2)$ i w tym przedziale wykres funkcji $f$ jest wypukły (wypukły w dół) $ \smile $ \\
	\item $f''(x) < 0 \Leftrightarrow x \in (0,1)\cup(e^2,\infty)$ i w tym przedziale wykres funkcji $f$ jest wklęsły (wypukły w górę) $ \frown $
	\end{enumerate}
\end{enumerate}
Bierzemy część wspólną odpowiednich przedziałów i otrzymujemy: $$(1,e)$$
\rozwStop

\odpStart
Funkcja jest jednocześnie malejąca i wypukła w dół w przedziale $(1,e)$.
\odpStop

\testStart
A. $(-\infty,e)$
B. $(e,\infty)$
C. $(1,e^2)$
D. $(1,e)$
\testStop

\kluczStart
D
\kluczStop

\end{document}