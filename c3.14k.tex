\documentclass[12pt, a4paper]{article}
\usepackage[utf8]{inputenc}
\usepackage{polski}

\usepackage{amsthm}  %pakiet do tworzenia twierdzeń itp.
\usepackage{amsmath} %pakiet do niektórych symboli matematycznych
\usepackage{amssymb} %pakiet do symboli mat., np. \nsubseteq
\usepackage{amsfonts}
\usepackage{graphicx} %obsługa plików graficznych z rozszerzeniem png, jpg
\theoremstyle{definition} %styl dla definicji
\newtheorem{zad}{} 
\title{Multizestaw zadań}
\author{Robert Fidytek}
%\date{\today}
\date{}
\newcounter{liczniksekcji}
\newcommand{\kategoria}[1]{\section{#1}} %olreślamy nazwę kateforii zadań
\newcommand{\zadStart}[1]{\begin{zad}#1\newline} %oznaczenie początku zadania
\newcommand{\zadStop}{\end{zad}}   %oznaczenie końca zadania
%Makra opcjonarne (nie muszą występować):
\newcommand{\rozwStart}[2]{\noindent \textbf{Rozwiązanie (autor #1 , recenzent #2): }\newline} %oznaczenie początku rozwiązania, opcjonarnie można wprowadzić informację o autorze rozwiązania zadania i recenzencie poprawności wykonania rozwiązania zadania
\newcommand{\rozwStop}{\newline}                                            %oznaczenie końca rozwiązania
\newcommand{\odpStart}{\noindent \textbf{Odpowiedź:}\newline}    %oznaczenie początku odpowiedzi końcowej (wypisanie wyniku)
\newcommand{\odpStop}{\newline}                                             %oznaczenie końca odpowiedzi końcowej (wypisanie wyniku)
\newcommand{\testStart}{\noindent \textbf{Test:}\newline} %ewentualne możliwe opcje odpowiedzi testowej: A. ? B. ? C. ? D. ? itd.
\newcommand{\testStop}{\newline} %koniec wprowadzania odpowiedzi testowych
\newcommand{\kluczStart}{\noindent \textbf{Test poprawna odpowiedź:}\newline} %klucz, poprawna odpowiedź pytania testowego (jedna literka): A lub B lub C lub D itd.
\newcommand{\kluczStop}{\newline} %koniec poprawnej odpowiedzi pytania testowego 
\newcommand{\wstawGrafike}[2]{\begin{figure}[h] \includegraphics[scale=#2] {#1} \end{figure}} %gdyby była potrzeba wstawienia obrazka, parametry: nazwa pliku, skala (jak nie wiesz co wpisać, to wpisz 1)

\begin{document}
\maketitle


\kategoria{Dymkowska, Beger/c3.14k}
\zadStart{Zadanie z Dymkowskiej, Beger C 3.14k) moja wersja nr [nrWersji]}
%[p1]:[2,3,4,5,6,7,8,9,10]
%[p2]:[2,3,4,5,6,7,8,9,10]
%[p3]=[p2]+1
%[a]=[p1]+1
%[b]=pow(2,[p3])
%[a1]=[a]+1
%[b1]=[p3]+1
%[c1]=[b1]*[p3]
%[up1]=[a1]*pow(2,[a1])*[p3]*2*[c1]*pow(2,[b1])-[a]*pow(2,[a])*2*[p3]*2*[c1]*pow(2,[b1])+[a]*pow(2,[a])*[a1]*pow(2,[a1])*[c1]*pow(2,[b1])-[a]*pow(2,[a])*[a1]*pow(2,[a1])*[p3]*2*[b]
%[down1]=[a]*pow(2,[a])*[a1]*pow(2,[a1])*[p3]*2*[c1]*pow(2,[b1])
%[g]=math.gcd([up1],[down1])
%[up]=int([up1]/[g])
%[down]=int([down1]/[g])
Za pomocą całki podwójnej obliczyć objętość bryły ograniczonej powierzchniami $$x=0, y=2x, y=1, z=0, z=-x^{[p1]}-y^{[p2]}$$
\zadStop
\rozwStart{Jakub Janik}{}
Objętość V możemy zapisać jako
$$V=\{(x,y,z)\in\mathbb{R}^3\colon(x,y)\in D, -x^2-y^2 \leq z \leq 0\}$$
Przy czym obszar D wyraża się jako
$$D\colon 0 \leq x \leq \frac{1}{2}, 2x \leq y \leq 1$$
Przechodzimy do obliczenia objętości
$$\iint_D x^{[p1]}+y^{[p2]}\ dxdy=\int_0^{\frac{1}{2}}dx\int_{2x}^1 x^{[p1]}+y^{[p2]}\ dy=$$
$$=\int_0^{\frac{1}{2}}x^{[p1]}y+\frac{1}{[p3]}y^{[p3]}\Big|_{2x}^1\ dx=\int_0^{\frac{1}{2}}x^{[p1]}-2x^{[a]}+\frac{1}{[p3]}-\frac{[b]}{[p3]}x^{[p3]}\ dx=$$
$$=\frac{1}{[a]}x^{[a]}-\frac{2}{[a1]}x^{[a1]}+\frac{1}{[p3]}x-\frac{[b]}{[c1]}x^{[b1]}\Big|_0^{\frac{1}{2}}=\frac{[up]}{[down]}$$
\rozwStop
\odpStart
$\frac{[up]}{[down]}$
\odpStop
\testStart
A.$\frac{[up]}{[down]}$
B.$0$
C.$-\frac{[up]}{[down]}$
D.$\infty$
\testStop
\kluczStart
A
\kluczStop



\end{document}