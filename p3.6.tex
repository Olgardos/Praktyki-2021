\documentclass[12pt, a4paper]{article}
\usepackage[utf8]{inputenc}
\usepackage{polski}

\usepackage{amsthm}  %pakiet do tworzenia twierdzeń itp.
\usepackage{amsmath} %pakiet do niektórych symboli matematycznych
\usepackage{amssymb} %pakiet do symboli mat., np. \nsubseteq
\usepackage{amsfonts}
\usepackage{graphicx} %obsługa plików graficznych z rozszerzeniem png, jpg
\theoremstyle{definition} %styl dla definicji
\newtheorem{zad}{} 
\title{Multizestaw zadań}
\author{Robert Fidytek}
%\date{\today}
\date{}
\newcounter{liczniksekcji}
\newcommand{\kategoria}[1]{\section{#1}} %olreślamy nazwę kateforii zadań
\newcommand{\zadStart}[1]{\begin{zad}#1\newline} %oznaczenie początku zadania
\newcommand{\zadStop}{\end{zad}}   %oznaczenie końca zadania
%Makra opcjonarne (nie muszą występować):
\newcommand{\rozwStart}[2]{\noindent \textbf{Rozwiązanie (autor #1 , recenzent #2): }\newline} %oznaczenie początku rozwiązania, opcjonarnie można wprowadzić informację o autorze rozwiązania zadania i recenzencie poprawności wykonania rozwiązania zadania
\newcommand{\rozwStop}{\newline}                                            %oznaczenie końca rozwiązania
\newcommand{\odpStart}{\noindent \textbf{Odpowiedź:}\newline}    %oznaczenie początku odpowiedzi końcowej (wypisanie wyniku)
\newcommand{\odpStop}{\newline}                                             %oznaczenie końca odpowiedzi końcowej (wypisanie wyniku)
\newcommand{\testStart}{\noindent \textbf{Test:}\newline} %ewentualne możliwe opcje odpowiedzi testowej: A. ? B. ? C. ? D. ? itd.
\newcommand{\testStop}{\newline} %koniec wprowadzania odpowiedzi testowych
\newcommand{\kluczStart}{\noindent \textbf{Test poprawna odpowiedź:}\newline} %klucz, poprawna odpowiedź pytania testowego (jedna literka): A lub B lub C lub D itd.
\newcommand{\kluczStop}{\newline} %koniec poprawnej odpowiedzi pytania testowego 
\newcommand{\wstawGrafike}[2]{\begin{figure}[h] \includegraphics[scale=#2] {#1} \end{figure}} %gdyby była potrzeba wstawienia obrazka, parametry: nazwa pliku, skala (jak nie wiesz co wpisać, to wpisz 1)

\begin{document}
\maketitle


\kategoria{Wikieł/p3.6}
\zadStart{Zadanie z Wikieł P 3.6 moja wersja nr [nrWersji]}
%[z]:[1,2,3,4,5,6,7,8,9,10,11,12]
%[y]:[1,2,3,4,5,6,7,8]
%[r]=random.randint(2,9)
%[a1]=random.randint([r]+1,50)
%[n]=random.randint(10,160)
%[an]=[a1]+([n]-1)*[r]
%[a2]=[a1]+[r]
%[a3]=[a2]+[r]
%[m]=[a1]-[r]
%[w2]=int((([a1]+[an])*[n])/2)
%[m]>0
Wyznaczyć liczbę składników w sumie $[a1]+[a2]+[a3]+...+[an]$ i obliczyć tę sumę.
\zadStop
\rozwStart{Katarzyna Filipowicz}{}
Zauważmy, że liczby $[a1],[a2],[a3],...$ są kolejnymi wyrazami ciągu arytmetycznego o wyrazie pierwszym $a_1=[a1]$ i różnicy $r=[r]$. Czyli wyraz ogólny wyraża się wzorem
$$
a_n=a_1+(n-1)r=[a1]+(n-1)\cdot [r]=[r] n+[m]
$$
Skoro $a_n=[r]n+[m]$ oraz $a_n=[an]$, to aby wyznaczyć liczbę składników w sumie, musimy wyznaczyć takie $n \in N$, dla którego $[r] n+[m]=[an]$. Stąd otrzymujemy $n=[n]$. Czyli $a_1=[a1]$, a $a_{[n]}=[an]$, dlatego suma
$$
S_{[n]}=\frac{(a_1+a_n)[n]}{2}=\frac{([a1]+[an])[n]}{2}=[w2]
$$
\rozwStop
\odpStart
Liczba składników w sumie: $[n]$, suma: $[w2]$
\odpStop
\testStart
A.Liczba składników w sumie: $[n]$, suma: $[w2]$
B.Liczba składników w sumie: $-[n]$, suma: $[w2]$
C.Liczba składników w sumie: $4$, suma: $0$
D.Liczba składników w sumie: $[n]$, suma: $-[w2]$
E.Liczba składników w sumie: $[r]$, suma: $[n]$
F.Liczba składników w sumie: $[n]$, suma: $[n]$
G.Liczba składników w sumie: $0$, suma: $0$
H.Liczba składników w sumie: $[an]$, suma: $[an]$
I.Liczba składników w sumie: $\infty$, suma: $\infty$
\testStop
\kluczStart
A
\kluczStop



\end{document}