\documentclass[12pt, a4paper]{article}
\usepackage[utf8]{inputenc}
\usepackage{polski}

\usepackage{amsthm}  %pakiet do tworzenia twierdzeń itp.
\usepackage{amsmath} %pakiet do niektórych symboli matematycznych
\usepackage{amssymb} %pakiet do symboli mat., np. \nsubseteq
\usepackage{amsfonts}
\usepackage{graphicx} %obsługa plików graficznych z rozszerzeniem png, jpg
\theoremstyle{definition} %styl dla definicji
\newtheorem{zad}{} 
\title{Multizestaw zadań}
\author{Robert Fidytek}
%\date{\today}
\date{}
\newcounter{liczniksekcji}
\newcommand{\kategoria}[1]{\section{#1}} %olreślamy nazwę kateforii zadań
\newcommand{\zadStart}[1]{\begin{zad}#1\newline} %oznaczenie początku zadania
\newcommand{\zadStop}{\end{zad}}   %oznaczenie końca zadania
%Makra opcjonarne (nie muszą występować):
\newcommand{\rozwStart}[2]{\noindent \textbf{Rozwiązanie (autor #1 , recenzent #2): }\newline} %oznaczenie początku rozwiązania, opcjonarnie można wprowadzić informację o autorze rozwiązania zadania i recenzencie poprawności wykonania rozwiązania zadania
\newcommand{\rozwStop}{\newline}                                            %oznaczenie końca rozwiązania
\newcommand{\odpStart}{\noindent \textbf{Odpowiedź:}\newline}    %oznaczenie początku odpowiedzi końcowej (wypisanie wyniku)
\newcommand{\odpStop}{\newline}                                             %oznaczenie końca odpowiedzi końcowej (wypisanie wyniku)
\newcommand{\testStart}{\noindent \textbf{Test:}\newline} %ewentualne możliwe opcje odpowiedzi testowej: A. ? B. ? C. ? D. ? itd.
\newcommand{\testStop}{\newline} %koniec wprowadzania odpowiedzi testowych
\newcommand{\kluczStart}{\noindent \textbf{Test poprawna odpowiedź:}\newline} %klucz, poprawna odpowiedź pytania testowego (jedna literka): A lub B lub C lub D itd.
\newcommand{\kluczStop}{\newline} %koniec poprawnej odpowiedzi pytania testowego 
\newcommand{\wstawGrafike}[2]{\begin{figure}[h] \includegraphics[scale=#2] {#1} \end{figure}} %gdyby była potrzeba wstawienia obrazka, parametry: nazwa pliku, skala (jak nie wiesz co wpisać, to wpisz 1)

\begin{document}
\maketitle


\kategoria{Wikieł/Z5.56a}
\zadStart{Zadanie z Wikieł Z 5.56 a) moja wersja nr [nrWersji]}
%[a]:[1,2,3,4,5,6,7,8,9,10,11,12,13,14]
Wyznaczyć równania asymptot wykresu funkcji:\\
a) $f(x)=\frac{x^2+[a]}{|x|}$
\zadStop
\rozwStart{Wojciech Przybylski}{Pascal Nawrocki}
1. Wyznaczamy dziedzinę funkcji.
$$|x|\neq0 \Rightarrow x\in\mathbb{R}\backslash\{0\}$$
$$D_{f}=(-\infty,0)\cup(0,\infty)$$
$$
f(x)=\frac{x^2+[a]}{|x|} = \left\{ \begin{array}{ll}
\frac{x^2+[a]}{-x} & \textrm{gdy $x\in(-\infty,0)$}\\
\frac{x^2+[a]}{x} & \textrm{gdy $x\in(0,\infty)$}
\end{array} \right.
$$
2. Wyznaczamy granicę. 
$$\lim_{x\to\infty}\frac{x^2+[a]}{x}=\lim_{x\to\infty}(x+\frac{[a]}{x})=\infty$$
$$\lim_{x\to-\infty}\frac{x^2+[a]}{-x}=\lim_{x\to-\infty}(-x+\frac{[a]}{-x})=\infty$$
$$\lim_{x\to0^{-}}\frac{x^2+[a]}{-x}=\frac{0^{+}+[a]}{0^{+}}=\infty$$
$$\lim_{x\to0^{+}}\frac{x^2+[a]}{x}=\frac{0^{+}+[a]}{0^{+}}=\infty$$
Istnieje asymptota pionowa w $x=0$, nie ma asymptot poziomych.\\
3.Sprawdzamy, czy istnieje asymtota ukośna.
$$y=ax+b,\hspace{3mm}a=\lim_{x\to\pm\infty}\frac{f(x)}{x},\hspace{3mm}b=\lim_{x\to\pm\infty}[f(x)-ax]$$
$$a=\lim_{x\to-\infty}\frac{\frac{x^2+[a]}{-x}}{x}=\lim_{x\to-\infty}\frac{x^2+[a]}{-x^{2}}=-1$$
$$b=\lim_{x\to-\infty}\frac{x^2+[a]}{-x}+x=\lim_{x\to-\infty}\frac{[a]}{-x}=0$$
$$a=\lim_{x\to\infty}\frac{\frac{x^2+[a]}{x}}{x}=\lim_{x\to\infty}\frac{x^2+[a]}{x^{2}}=1$$
$$b=\lim_{x\to\infty}\frac{x^2+[a]}{x}-x=\lim_{x\to\infty}\frac{[a]}{x}=0$$
Istnieją asymptoty poziome, lewostronna $y=-x$ oraz prawostronna $y=x$.
\rozwStop
\odpStart
pionowa $x=0$; ukośne: $y=-x$(lewostronna) i $y=x$(prawostronna).
\odpStop
\testStart
A. pionowa $x=0$; ukośne: $y=-x$(lewostronna) i $y=x$(prawostronna).\\
B. pionowa $x=[a]$; ukośne: $y=-x$(lewostronna) i $y=x$(prawostronna).\\
C. pionowa $x=[a]$; ukośne: $y=-2x$(lewostronna) i $y=2x$(prawostronna).\\
D. pionowa $x=0$; pozioma: $y=0$.\\
E. pionowa $x=[a]$; pozioma: $y=0$.\\
F. Funkcja $f(x)$ nie posiada asymptot.
\testStop
\kluczStart
A
\kluczStop



\end{document}