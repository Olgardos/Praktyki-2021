\documentclass[12pt, a4paper]{article}
\usepackage[utf8]{inputenc}
\usepackage{polski}

\usepackage{amsthm}  %pakiet do tworzenia twierdzeń itp.
\usepackage{amsmath} %pakiet do niektórych symboli matematycznych
\usepackage{amssymb} %pakiet do symboli mat., np. \nsubseteq
\usepackage{amsfonts}
\usepackage{graphicx} %obsługa plików graficznych z rozszerzeniem png, jpg
\theoremstyle{definition} %styl dla definicji
\newtheorem{zad}{} 
\title{Multizestaw zadań}
\author{Robert Fidytek}
%\date{\today}
\date{}
\newcounter{liczniksekcji}
\newcommand{\kategoria}[1]{\section{#1}} %olreślamy nazwę kateforii zadań
\newcommand{\zadStart}[1]{\begin{zad}#1\newline} %oznaczenie początku zadania
\newcommand{\zadStop}{\end{zad}}   %oznaczenie końca zadania
%Makra opcjonarne (nie muszą występować):
\newcommand{\rozwStart}[2]{\noindent \textbf{Rozwiązanie (autor #1 , recenzent #2): }\newline} %oznaczenie początku rozwiązania, opcjonarnie można wprowadzić informację o autorze rozwiązania zadania i recenzencie poprawności wykonania rozwiązania zadania
\newcommand{\rozwStop}{\newline}                                            %oznaczenie końca rozwiązania
\newcommand{\odpStart}{\noindent \textbf{Odpowiedź:}\newline}    %oznaczenie początku odpowiedzi końcowej (wypisanie wyniku)
\newcommand{\odpStop}{\newline}                                             %oznaczenie końca odpowiedzi końcowej (wypisanie wyniku)
\newcommand{\testStart}{\noindent \textbf{Test:}\newline} %ewentualne możliwe opcje odpowiedzi testowej: A. ? B. ? C. ? D. ? itd.
\newcommand{\testStop}{\newline} %koniec wprowadzania odpowiedzi testowych
\newcommand{\kluczStart}{\noindent \textbf{Test poprawna odpowiedź:}\newline} %klucz, poprawna odpowiedź pytania testowego (jedna literka): A lub B lub C lub D itd.
\newcommand{\kluczStop}{\newline} %koniec poprawnej odpowiedzi pytania testowego 
\newcommand{\wstawGrafike}[2]{\begin{figure}[h] \includegraphics[scale=#2] {#1} \end{figure}} %gdyby była potrzeba wstawienia obrazka, parametry: nazwa pliku, skala (jak nie wiesz co wpisać, to wpisz 1)

\begin{document}
\maketitle


\kategoria{Wikieł/P3.29a}
\zadStart{Zadanie z Wikieł P 3.29 a) moja wersja nr [nrWersji]}
%[a1]:[2,3,4,5,6,7,8,9]
%[r]:[2,3,4,5,6,7,8,9]
%[a]:[2,3,4,5,6,7,8,9]
%[b]:[2,3,4,5,6,7,8,9]
%[a2]=[a1]+[r]
%[a3]=[a2]+[r]
%[ra1]=[r]-[a1]
%[a1ra1m]=[a1]-[ra1]
%[a2r]=[a]*2
%[ra1]>1 and [a1ra1m]>1 and [a]!=[b] and math.gcd([r],[a2r])==1
Obliczyć granicę ciągu $a_{n}=\frac{[a1]+[a2]+[a3]+\cdots+([r]n-[ra1])}{[a]n^2+[b]}$.
\zadStop
\rozwStart{Robert Fidytek}{}
$$\lim\limits_{n\to\infty}\left(\frac{[a1]+[a2]+[a3]+\cdots+([r]n-[ra1])}{[a]n^2+[b]}\right)=$$ 
$$=\lim\limits_{n\to\infty}\left(\frac{\frac{([a1]+([r]n-[ra1]))n}{2}}{[a]n^2+[b]}\right)=$$ 
$$=\lim\limits_{n\to\infty}\frac{([r]n+[a1ra1m])n}{2\left([a]n^2+[b]\right)}=$$ 
$$=\lim\limits_{n\to\infty}\frac{n^2\left([r]+\frac{[a1ra1m]}{n}\right)}{2n^2\left([a]+\frac{[b]}{n^2}\right)}=$$ 
$$=\lim\limits_{n\to\infty}\frac{[r]+\frac{[a1ra1m]}{n}}{2\left([a]+\frac{[b]}{n^2}\right)}=\frac{[r]}{2\cdot[a]}=\frac{[r]}{[a2r]}$$ 
\rozwStop
\odpStart
$\frac{[r]}{[a2r]}$
\odpStop
\testStart
A.$\frac{[r]}{[a2r]}$
B.$-\infty$
C.$\infty$
D.$0$
E.$1$
F.$\frac{[a2r]}{[r]}$
G.$[r]$
H.$\frac{[r]}{[a]}$
I.$[a2r]$
\testStop
\kluczStart
A
\kluczStop



\end{document}