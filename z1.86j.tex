\documentclass[12pt, a4paper]{article}
\usepackage[utf8]{inputenc}
\usepackage{polski}
\usepackage{amsthm}  %pakiet do tworzenia twierdzeń itp.
\usepackage{amsmath} %pakiet do niektórych symboli matematycznych
\usepackage{amssymb} %pakiet do symboli mat., np. \nsubseteq
\usepackage{amsfonts}
\usepackage{graphicx} %obsługa plików graficznych z rozszerzeniem png, jpg
\theoremstyle{definition} %styl dla definicji
\newtheorem{zad}{} 
\title{Multizestaw zadań}
\author{Radosław Grzyb}
%\date{\today}
\date{}
\newcounter{liczniksekcji}
\newcommand{\kategoria}[1]{\section{#1}} %olreślamy nazwę kateforii zadań
\newcommand{\zadStart}[1]{\begin{zad}#1\newline} %oznaczenie początku zadania
\newcommand{\zadStop}{\end{zad}}   %oznaczenie końca zadania
%Makra opcjonarne (nie muszą występować):
\newcommand{\rozwStart}[2]{\noindent \textbf{Rozwiązanie (autor #1 , recenzent #2): }\newline} %oznaczenie początku rozwiązania, opcjonarnie można wprowadzić informację o autorze rozwiązania zadania i recenzencie poprawności wykonania rozwiązania zadania
\newcommand{\rozwStop}{\newline}                                            %oznaczenie końca rozwiązania
\newcommand{\odpStart}{\noindent \textbf{Odpowiedź:}\newline}    %oznaczenie początku odpowiedzi końcowej (wypisanie wyniku)
\newcommand{\odpStop}{\newline}                                             %oznaczenie końca odpowiedzi końcowej (wypisanie wyniku)
\newcommand{\testStart}{\noindent \textbf{Test:}\newline} %ewentualne możliwe opcje odpowiedzi testowej: A. ? B. ? C. ? D. ? itd.
\newcommand{\testStop}{\newline} %koniec wprowadzania odpowiedzi testowych
\newcommand{\kluczStart}{\noindent \textbf{Test poprawna odpowiedź:}\newline} %klucz, poprawna odpowiedź pytania testowego (jedna literka): A lub B lub C lub D itd.
\newcommand{\kluczStop}{\newline} %koniec poprawnej odpowiedzi pytania testowego 
\newcommand{\wstawGrafike}[2]{\begin{figure}[h] \includegraphics[scale=#2] {#1} \end{figure}} %gdyby była potrzeba wstawienia obrazka, parametry: nazwa pliku, skala (jak nie wiesz co wpisać, to wpisz 1)
\begin{document}
\maketitle
\kategoria{Wikieł/Z1.86j}
\zadStart{Zadanie z Wikieł Z 1.86j moja wersja nr [nrWersji]}
%[p1]:[3,4,5,6,7,8,9,10,11,12,13,14]
%[p2]:[2,3,4,5,6,7,8,9,11,12,13]
%[p3]:[1,2,3]
%[b]=[p2]**2
%[bb]=[p1]**2
%[a]=[p1]**[p3]
%[c]=[p1]*[p2]
%[Delta]=[a]**2+4*[b]
%[sDelta]=[Delta]**(1/2)
%[tDelta]=int([sDelta])
%[t1]=int((-[a]-[tDelta])/2)
%[t2]=int((-[a]+[tDelta])/2)
%[gcd]=math.gcd([p1],[p2])
%[gcd1]=int([p1]/[gcd])
%[gcd2]=int([p2]/[gcd])
%[Delta]>0 and ([sDelta]).is_integer() is True and [p1]>[p2]
Rozwiązać nierówność:
$$[p1]^{2x}+\left(\frac{1}{[p2]}\right)^{-x}\cdot[p1]^{x+[p3]}-[p2]^{2x+2}\leq0$$
\zadStop
\rozwStart{Radosław Grzyb}{}
Przekształcamy naszą nierówność:
$$[bb]^{x}+[p2]^{x}\cdot[a]\cdot[p1]^{x}-[b]^{x+1}\leq0$$
$$[bb]^{x}+[a]\cdot[c]^{x}-[b]\cdot[b]^{x}\leq0$$
Dzielimy obie strony przez $[c]^{x}$ otrzymując:
$$\left(\frac{[gcd1]}{[gcd2]}\right)^{x}+[a]-[b]\cdot\left(\frac{[gcd2]}{[gcd1]}\right)^{x}\leq0$$
Następnie mnożymy obie strony przez $\left(\frac{[gcd1]}{[gcd2]}\right)^{x}$ dostając:
$$\left(\frac{[gcd1]}{[gcd2]}\right)^{2x}+[a]\left(\frac{[gcd1]}{[gcd2]}\right)^{x}-[b]\leq0$$
Podstawiając $t=\left(\frac{[gcd1]}{[gcd2]}\right)^{x}$ otrzymamy do rozwiązania nierówność kwadratową:
$$t^{2}+[a]t-[b]\leq0$$
$$\Delta_{t}=[a]^2-4\cdot1\cdot(-[b])=[Delta]\implies \sqrt{\Delta_{t}}=[tDelta]$$\\
Czas znaleźć miejsca zerowe:
$$t_{1}=\frac{-[a]-[tDelta]}{2}=[t1]$$
$$t_{2}=\frac{-[a]+[tDelta]}{2}=[t2]$$
Współczynnik $a$ naszej funkcji kwadratowej jest dodatni oraz odrzucamy ujemne rozwiązania ponieważ $\left(\frac{[gcd1]}{[gcd2]}\right)^{x}\geq0$. Zatem naszym rozwiązaniem jest:
$$t\geq[t2]$$
$$\left(\frac{[gcd1]}{[gcd2]}\right)^{x}\geq[t2]\implies x\geq\frac{\ln[t2]}{\ln\left(\frac{[gcd1]}{[gcd2]}\right)}$$
\rozwStop
\odpStart
$$x>\frac{\ln[t2]}{\ln\left(\frac{[p1]}{[p2]}\right)}$$
\odpStop
\testStart
A.$$x>\frac{1}{\ln[p2]}$$
B.$$x>\ln[t2]\cdot\ln[p1]$$
C.$$x\geq\frac{\ln[t2]}{\ln\left(\frac{[gcd1]}{[gcd2]}\right)}$$
D.$$x<\frac{\ln[t2]}{\ln[p3]}$$
\testStop
\kluczStart
C
\kluczStop
\end{document}