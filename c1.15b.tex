\documentclass[12pt, a4paper]{article}
\usepackage[utf8]{inputenc}
\usepackage{polski}

\usepackage{amsthm}  %pakiet do tworzenia twierdzeń itp.
\usepackage{amsmath} %pakiet do niektórych symboli matematycznych
\usepackage{amssymb} %pakiet do symboli mat., np. \nsubseteq
\usepackage{amsfonts}
\usepackage{graphicx} %obsługa plików graficznych z rozszerzeniem png, jpg
\theoremstyle{definition} %styl dla definicji
\newtheorem{zad}{} 
\title{Multizestaw zadań}
\author{Robert Fidytek}
%\date{\today}
\date{}
\newcounter{liczniksekcji}
\newcommand{\kategoria}[1]{\section{#1}} %olreślamy nazwę kateforii zadań
\newcommand{\zadStart}[1]{\begin{zad}#1\newline} %oznaczenie początku zadania
\newcommand{\zadStop}{\end{zad}}   %oznaczenie końca zadania
%Makra opcjonarne (nie muszą występować):
\newcommand{\rozwStart}[2]{\noindent \textbf{Rozwiązanie (autor #1 , recenzent #2): }\newline} %oznaczenie początku rozwiązania, opcjonarnie można wprowadzić informację o autorze rozwiązania zadania i recenzencie poprawności wykonania rozwiązania zadania
\newcommand{\rozwStop}{\newline}                                            %oznaczenie końca rozwiązania
\newcommand{\odpStart}{\noindent \textbf{Odpowiedź:}\newline}    %oznaczenie początku odpowiedzi końcowej (wypisanie wyniku)
\newcommand{\odpStop}{\newline}                                             %oznaczenie końca odpowiedzi końcowej (wypisanie wyniku)
\newcommand{\testStart}{\noindent \textbf{Test:}\newline} %ewentualne możliwe opcje odpowiedzi testowej: A. ? B. ? C. ? D. ? itd.
\newcommand{\testStop}{\newline} %koniec wprowadzania odpowiedzi testowych
\newcommand{\kluczStart}{\noindent \textbf{Test poprawna odpowiedź:}\newline} %klucz, poprawna odpowiedź pytania testowego (jedna literka): A lub B lub C lub D itd.
\newcommand{\kluczStop}{\newline} %koniec poprawnej odpowiedzi pytania testowego 
\newcommand{\wstawGrafike}[2]{\begin{figure}[h] \includegraphics[scale=#2] {#1} \end{figure}} %gdyby była potrzeba wstawienia obrazka, parametry: nazwa pliku, skala (jak nie wiesz co wpisać, to wpisz 1)

\begin{document}
\maketitle



\kategoria{Dymkowska,Beger/C1.15a}
\zadStart{Zadanie z Dymkowska,Beger C 1.15 b) moja wersja nr [nrWersji]}
%[a]:[2,3,4,5,6,7,8,9]
%[b]=[a]/4
%[c]=[b]/2
%[d]=[b]/4
%[c].is_integer()==False
Stosując odpowiednie metody całkowania, obliczyć całkę $\displaystyle \int [a]e^{2x^4}x^7 \ dx$
\zadStop
\rozwStart{Mirella Narewska}{}
$$\int [a]e^{2x^4}x^7 \ dx=[a]\int e^{2x^4}x^7 \ dx$$
$$\text{Całkujemy przez podstawienie:} u=x^4\Rightarrow du=4x^3 \ dx$$
$$=[b]\int e^{2u}u \ du=$$
$$\text{Całkujemy przez części:}$$
$$t=u \ v'=e^{2u}$$
$$u'=1 \ v=\frac{1}{2}e^{2u}$$
$$[b]\left( \frac{1}{2}e^{2u}u -\frac{1}{2}\int e^{2u} \ du\right)= [b]\left( \frac{1}{2}e^{2u}u  -\frac{1}{4}e^{2u}  \ du\right)+C=[c]e^{2u}u-[d]e^{2u} +C= [c]e^{2x^4}x^4-[d]e^{2x^4}+C$$
\odpStart
$$[c]e^{2u}u-[d]e^{2u} +C= [c]e^{2x^4}x^4-[d]e^{2x^4}+C$$
\odpStop
\testStart
A.$[c]e^{2u}u-[d]e^{2u} +C= [c]e^{2x^4}x^4-[d]e^{2x^4}+C$
\\
B.$\frac{1}{[a]}\cdot e^{[a]x+[b]}+C$
\\
C.$e^{\sqrt{x}}(\sqrt{x}-1)+C$
\\
D.$e^{[a]x+[b]}+C$
\testStop
\kluczStart
A
\kluczStop


\end{document}