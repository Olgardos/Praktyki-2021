\documentclass[12pt, a4paper]{article}
\usepackage[utf8]{inputenc}
\usepackage{polski}
\usepackage{amsthm}  %pakiet do tworzenia twierdzeń itp.
\usepackage{amsmath} %pakiet do niektórych symboli matematycznych
\usepackage{amssymb} %pakiet do symboli mat., np. \nsubseteq
\usepackage{amsfonts}
\usepackage{graphicx} %obsługa plików graficznych z rozszerzeniem png, jpg
\theoremstyle{definition} %styl dla definicji
\newtheorem{zad}{} 
\title{Multizestaw zadań}
\author{Radosław Grzyb}
%\date{\today}
\date{}
\newcounter{liczniksekcji}
\newcommand{\kategoria}[1]{\section{#1}} %olreślamy nazwę kateforii zadań
\newcommand{\zadStart}[1]{\begin{zad}#1\newline} %oznaczenie początku zadania
\newcommand{\zadStop}{\end{zad}}   %oznaczenie końca zadania
%Makra opcjonarne (nie muszą występować):
\newcommand{\rozwStart}[2]{\noindent \textbf{Rozwiązanie (autor #1 , recenzent #2): }\newline} %oznaczenie początku rozwiązania, opcjonarnie można wprowadzić informację o autorze rozwiązania zadania i recenzencie poprawności wykonania rozwiązania zadania
\newcommand{\rozwStop}{\newline}                                            %oznaczenie końca rozwiązania
\newcommand{\odpStart}{\noindent \textbf{Odpowiedź:}\newline}    %oznaczenie początku odpowiedzi końcowej (wypisanie wyniku)
\newcommand{\odpStop}{\newline}                                             %oznaczenie końca odpowiedzi końcowej (wypisanie wyniku)
\newcommand{\testStart}{\noindent \textbf{Test:}\newline} %ewentualne możliwe opcje odpowiedzi testowej: A. ? B. ? C. ? D. ? itd.
\newcommand{\testStop}{\newline} %koniec wprowadzania odpowiedzi testowych
\newcommand{\kluczStart}{\noindent \textbf{Test poprawna odpowiedź:}\newline} %klucz, poprawna odpowiedź pytania testowego (jedna literka): A lub B lub C lub D itd.
\newcommand{\kluczStop}{\newline} %koniec poprawnej odpowiedzi pytania testowego 
\newcommand{\wstawGrafike}[2]{\begin{figure}[h] \includegraphics[scale=#2] {#1} \end{figure}} %gdyby była potrzeba wstawienia obrazka, parametry: nazwa pliku, skala (jak nie wiesz co wpisać, to wpisz 1)
\begin{document}
\maketitle
\kategoria{Beger/c1.10x}
\zadStart{Zadanie z Beger C 1.10x moja wersja nr [nrWersji]}
%[p1]:[2,3,4,5,6,7,8,9]
%[p2]:[2,4,6,8,10]
%[c]=2*[p2]
%[s]=int(math.sqrt([p1]))
%[Delta]=2*2-4*[p1]*[p1]
%[qDelta]=-int([Delta]/4)
%[h]=[p1]*[p2]
%[hh]=[p1]**2
Obliczyć całkę funkcji trygonometrycznej:
$$\int \frac{1}{[p1]\cos^2(x)+\sin([p2]x)\cos([p2]x)+[p1]\sin^2(x)},dx$$
\zadStop
\rozwStart{Radosław Grzyb}{}
Przekształcamy naszą funkcję podcałkową:
$$\int \frac{1}{[p1]\cos^2(x)+2\sin([p2]x)\cos([p2]x)+[p1]\sin^2(x)},dx = \int \frac{1}{\sin([c]x)+[p1]} \,dx$$
Podstawiamy teraz:
$$u=[c]x \implies dx=\frac{1}{[c]}du$$
Otrzymujemy:
$$\int \frac{1}{\sin([c]x)+[p1]}\cdot\frac{1}{[c]} \,du=\frac{1}{[c]}\int \frac{1}{\sin(u)+[p1]}\,du$$
Teraz dokonamy podstawienia Weierstrass'a:
$$\sin(x)=\frac{2t}{1+t^2}$$\
$$t=\tan(\frac{x}{2})$$
Po podstawieniu otrzymujemy:
$$\frac{1}{[c]}\int \frac{1}{\sin(u)+[p1]}\,du=\frac{1}{[c]}\int \frac{1}{\frac{2\tan(\frac{u}{2})}{\tan^2(\frac{u}{2})+1}+[p1]}\,du=\frac{1}{[c]}\int\frac{\tan^2(\frac{u}{2})+1}{2\tan(\frac{u}{2})+[p1]\tan^2(\frac{u}{2})+[p1]}\,du$$
Dokonujemy kolejnego podstawienia:
$$v=\tan(\frac{u}{2}) \implies dv=\frac{1}{2\cos^2(\frac{u}{2})}du\implies du=2\cos^2(\frac{u}{2})dv\implies du=\frac{2}{v^2+1}dv$$
Ostatnie przejście wynika z podstawienia Weierstrass'a:
$$\cos^2(x)=\frac{1}{t^2+1}$$
Otrzymujemy:
$$\frac{1}{[c]}\int\frac{v^2+1}{2v+[p1]v^2+[p1]}\cdot\frac{2}{v^2+1}\,dv=\frac{1}{[p2]}\int\frac{1}{[p1]v^2+2v+[p1]}\,dv$$
Otrzymany w mianowniku trójmian kwadratowy sprowadzamy do postaci kanonicznej i lekko przeszktałcamy:
$$\frac{1}{[p2]}\int\frac{1}{[p1](v+\frac{1}{[p1]})^2+\frac{[qDelta]}{[p1]}}\,dv=\frac{1}{[h]}\int\frac{1}{(v+\frac{1}{[p1]})^2+\frac{[qDelta]}{[hh]}}\,dv$$
Dokonujemy kolejnego, ale ostatniego już podstawienia:
$$p=v+\frac{1}{[p1]}\implies dp=dv$$
Otrzymujemy:
$$\frac{1}{[h]}\int\frac{1}{p^2+\frac{[qDelta]}{[hh]}}\,dp$$
Do policzenia tej całki wykorzystamy gotowy wzór: $\frac{dx}{x^2+a^2}=\frac{1}{a}\arctan\frac{x}{a}+C$
$$\frac{1}{[h]}\int\frac{1}{p^2+\left(\frac{\sqrt{[qDelta]}}{[p1]}\right)^2}\,dp=\frac{1}{[h]}\cdot\frac{[p1]}{\sqrt{[qDelta]}}\arctan\left(\frac{[p1]p}{\sqrt{[qDelta]}}\right)+C=$$
$$=\frac{1}{[p2]\sqrt{[qDelta]}}\arctan\left(\frac{[p1](v+\frac{1}{[p1]})}{\sqrt{[qDelta]}}\right)+C=\frac{1}{[p2]\sqrt{[qDelta]}}\arctan\left(\frac{[p1](\tan(\frac{u}{2})+\frac{1}{[p1]})}{\sqrt{[qDelta]}}\right)+C=$$
$$=\frac{1}{[p2]\sqrt{[qDelta]}}\arctan\left(\frac{[p1](\tan([p2]x)+\frac{1}{[p1]})}{\sqrt{[qDelta]}}\right)+C$$
\rozwStop
\odpStart
$$\frac{1}{[p2]\sqrt{[qDelta]}}\arctan\left(\frac{[p1](\tan([p2]x)+\frac{1}{[p1]})}{\sqrt{[qDelta]}}\right)+C$$
\odpStop
\testStart
A.$$\frac{1}{[p2]\sqrt{[qDelta]}}\arctan\left(\frac{[p1](\tan([p2]x)+\frac{1}{[p1]})}{\sqrt{[qDelta]}}\right)+C$$
B.$$\arctan\left(\frac{[p1](\tan([p2]x)+\frac{1}{[p1]})}{\sqrt{[qDelta]}}\right)+C$$
C.$$2\sqrt{[p1]-e^{[p2]x}}+[c]\ln\left|\frac{\sqrt{[p1]-e^{[p2]x}}-[s]}{\sqrt{[p1]-e^{[p2]x}}+1}\right|+C$$
D.$$\sqrt{[qDelta]x}+C$$
\testStop
\kluczStart
A
\kluczStop
\end{document}