\documentclass[12pt, a4paper]{article}
\usepackage[utf8]{inputenc}
\usepackage{polski}

\usepackage{amsthm}  %pakiet do tworzenia twierdzeń itp.
\usepackage{amsmath} %pakiet do niektórych symboli matematycznych
\usepackage{amssymb} %pakiet do symboli mat., np. \nsubseteq
\usepackage{amsfonts}
\usepackage{graphicx} %obsługa plików graficznych z rozszerzeniem png, jpg
\theoremstyle{definition} %styl dla definicji
\newtheorem{zad}{} 
\title{Multizestaw zadań}
\author{Robert Fidytek}
%\date{\today}
\date{}
\newcounter{liczniksekcji}
\newcommand{\kategoria}[1]{\section{#1}} %olreślamy nazwę kateforii zadań
\newcommand{\zadStart}[1]{\begin{zad}#1\newline} %oznaczenie początku zadania
\newcommand{\zadStop}{\end{zad}}   %oznaczenie końca zadania
%Makra opcjonarne (nie muszą występować):
\newcommand{\rozwStart}[2]{\noindent \textbf{Rozwiązanie (autor #1 , recenzent #2): }\newline} %oznaczenie początku rozwiązania, opcjonarnie można wprowadzić informację o autorze rozwiązania zadania i recenzencie poprawności wykonania rozwiązania zadania
\newcommand{\rozwStop}{\newline}                                            %oznaczenie końca rozwiązania
\newcommand{\odpStart}{\noindent \textbf{Odpowiedź:}\newline}    %oznaczenie początku odpowiedzi końcowej (wypisanie wyniku)
\newcommand{\odpStop}{\newline}                                             %oznaczenie końca odpowiedzi końcowej (wypisanie wyniku)
\newcommand{\testStart}{\noindent \textbf{Test:}\newline} %ewentualne możliwe opcje odpowiedzi testowej: A. ? B. ? C. ? D. ? itd.
\newcommand{\testStop}{\newline} %koniec wprowadzania odpowiedzi testowych
\newcommand{\kluczStart}{\noindent \textbf{Test poprawna odpowiedź:}\newline} %klucz, poprawna odpowiedź pytania testowego (jedna literka): A lub B lub C lub D itd.
\newcommand{\kluczStop}{\newline} %koniec poprawnej odpowiedzi pytania testowego 
\newcommand{\wstawGrafike}[2]{\begin{figure}[h] \includegraphics[scale=#2] {#1} \end{figure}} %gdyby była potrzeba wstawienia obrazka, parametry: nazwa pliku, skala (jak nie wiesz co wpisać, to wpisz 1)

\begin{document}
\maketitle



\kategoria{Dymkowska,Beger/C2.5c}
\zadStart{Zadanie z Dymkowska,Beger C 2.5 c) moja wersja nr [nrWersji]}
%[a]:[2,3,4,5,6,7,8]
%[b]=[a]*3
%[c]=3**[b]
%[d]=2*[a]
Obliczyć całkę oznaczoną $\displaystyle \int_{-2}^{0} [a]\ln{(1-x)} \ dx $
\zadStop
\rozwStart{Mirella Narewska}{}
$$ \int_{-2}^{0} [a]\ln{(1-x)} \ dx  = [a]  \int_{-2}^{0} \ln{(1-x)} \ dx = $$
$$\text{Całkujemy przez podstawienie:}$$
$$u=1-x \Rightarrow du=-dx$$
$$\text{Zmieniają się także granice:}$$
$$-[a]\int_{3}^{1} \ln{u} \ du= [a]\int_{1}^{3} \ln{u} \ du$$
$$\text{Całkujemy przez części}$$
$$f=\ln{u} \ \ g'=du$$
$$f'=\frac{1}{u} \ \ g=u$$
$$=[a]u\ln{u}|_{1}^{3} -[a]\int_{1}^{3} 1 \ du= [b]\ln{3} -[a]]u|_{1}^{3}=\ln{[c]} -[d]$$

\odpStart
$$\ln{[c]} -[d]$$
\odpStop
\testStart
A.$\ln{[c]} -[d]$
\\
B.$\cos{[a]}$
\\
C.$0$
\\
D.$\ln{[a]}$
\testStop
\kluczStart
A
\kluczStop


\end{document}