\documentclass[12pt, a4paper]{article}
\usepackage[utf8]{inputenc}
\usepackage{polski}
\usepackage{amsthm}  %pakiet do tworzenia twierdzeń itp.
\usepackage{amsmath} %pakiet do niektórych symboli matematycznych
\usepackage{amssymb} %pakiet do symboli mat., np. \nsubseteq
\usepackage{amsfonts}
\usepackage{graphicx} %obsługa plików graficznych z rozszerzeniem png, jpg
\theoremstyle{definition} %styl dla definicji
\newtheorem{zad}{} 
\title{Multizestaw zadań}
\author{Radosław Grzyb}
%\date{\today}
\date{}
\newcounter{liczniksekcji}
\newcommand{\kategoria}[1]{\section{#1}} %olreślamy nazwę kateforii zadań
\newcommand{\zadStart}[1]{\begin{zad}#1\newline} %oznaczenie początku zadania
\newcommand{\zadStop}{\end{zad}}   %oznaczenie końca zadania
%Makra opcjonarne (nie muszą występować):
\newcommand{\rozwStart}[2]{\noindent \textbf{Rozwiązanie (autor #1 , recenzent #2): }\newline} %oznaczenie początku rozwiązania, opcjonarnie można wprowadzić informację o autorze rozwiązania zadania i recenzencie poprawności wykonania rozwiązania zadania
\newcommand{\rozwStop}{\newline}                                            %oznaczenie końca rozwiązania
\newcommand{\odpStart}{\noindent \textbf{Odpowiedź:}\newline}    %oznaczenie początku odpowiedzi końcowej (wypisanie wyniku)
\newcommand{\odpStop}{\newline}                                             %oznaczenie końca odpowiedzi końcowej (wypisanie wyniku)
\newcommand{\testStart}{\noindent \textbf{Test:}\newline} %ewentualne możliwe opcje odpowiedzi testowej: A. ? B. ? C. ? D. ? itd.
\newcommand{\testStop}{\newline} %koniec wprowadzania odpowiedzi testowych
\newcommand{\kluczStart}{\noindent \textbf{Test poprawna odpowiedź:}\newline} %klucz, poprawna odpowiedź pytania testowego (jedna literka): A lub B lub C lub D itd.
\newcommand{\kluczStop}{\newline} %koniec poprawnej odpowiedzi pytania testowego 
\newcommand{\wstawGrafike}[2]{\begin{figure}[h] \includegraphics[scale=#2] {#1} \end{figure}} %gdyby była potrzeba wstawienia obrazka, parametry: nazwa pliku, skala (jak nie wiesz co wpisać, to wpisz 1)
\begin{document}
\maketitle
\kategoria{Beger/c1.12f}
\zadStart{Zadanie z Beger C 1.12f moja wersja nr [nrWersji]}
%[p1]:[1,2,3,4,5,6,7,8,9,10,11,12,13,14,15]
%[p2]:[1,2,3,4,5,6,7,8,9,10,11,12,13,14,15]
%[c]=[p1]+[p2]
%[ss]=math.sqrt([c])
%[a]=int([ss])
%[aa]=2*[a]
%([ss]).is_integer() is True
Obliczyć całkę funkcji niewymiernej:
$$\int \frac{1}{(x+[p1])\sqrt{[p2]-x}}\,dx$$
\zadStop
\rozwStart{Radosław Grzyb}{}
Dokonujemy podstawienia:
$$u=\sqrt{[p2]-x}\implies du=-\frac{1}{2\sqrt{[p2]-x}}dx\implies dx=-2\sqrt{[p2]-x}du$$
Robiąc to podstawienie warto zauważyć, że:
$$u=\sqrt{[p2]-x}\implies u^2=[p2]-x \implies x=[p2]-u^2$$
Otrzymujemy wówczas:
$$\int \frac{1}{(x+[p1])\sqrt{[p2]-x}}\cdot(-2\sqrt{[p2]-x})\,du=-2\int \frac{1}{x+[p1]}\,du=-2\int \frac{1}{[p2]-u^2+[p1]}\,du=$$
$$=-2\int \frac{1}{-u^2+[c]}\,du=2\int \frac{1}{u^2-[c]}\,du$$
Do policzenia otrzymanej całki wykorzystamy gotowy wzór: $$\int\frac{dx}{x^2-a^2}=\frac{1}{2a}\ln\left|\frac{x-a}{x+a}\right|+C$$
Podstawiając otrzymujemy finalny wynik:
$$2\int \frac{1}{u^2-[a]^2}\,du=\frac{1}{[a]}\ln\left|\frac{u-[a]}{u+[a]}\right|+C=\frac{1}{[a]}\ln\left|\frac{\sqrt{[p2]-x}-[a]}{\sqrt{[p2]-x}+[a]}\right|+C$$
A więc dostajemy finalny wynik:
$$\frac{1}{[a]}\ln\left|\frac{\sqrt{[p2]-x}-[a]}{\sqrt{[p2]-x}+[a]}\right|+C$$
\rozwStop
\odpStart
$$\frac{1}{[a]}\ln\left|\frac{\sqrt{[p2]-x}-[a]}{\sqrt{[p2]-x}+[a]}\right|+C$$
\odpStop
\testStart
A.$$-\frac{1}{[a]}\ln\left|\frac{\sqrt{[p2]-x}-[a]}{\sqrt{[p2]-x}+[a]}\right|+C$$
B.$$\frac{1}{[a]}\ln\left|\frac{\sqrt{x}-[a]}{\sqrt{x}+[a]}\right|+C$$
C.$$\frac{1}{[a]}\ln\left|\frac{\sqrt{[p2]-x}-[a]}{\sqrt{[p2]-x}+[a]}\right|+C$$
D.$$\frac{1}{[a]}\ln\left|\frac{[p2]-x}{[a]}{\sqrt{[p2]-x}+[a]}\right|+C$$
\testStop
\kluczStart
C
\kluczStop
\end{document}