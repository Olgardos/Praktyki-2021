\documentclass[12pt, a4paper]{article}
\usepackage[utf8]{inputenc}
\usepackage{polski}

\usepackage{amsthm}  %pakiet do tworzenia twierdzeń itp.
\usepackage{amsmath} %pakiet do niektórych symboli matematycznych
\usepackage{amssymb} %pakiet do symboli mat., np. \nsubseteq
\usepackage{amsfonts}
\usepackage{graphicx} %obsługa plików graficznych z rozszerzeniem png, jpg
\theoremstyle{definition} %styl dla definicji
\newtheorem{zad}{} 
\title{Multizestaw zadań}
\author{Robert Fidytek}
%\date{\today}
\date{}
\newcounter{liczniksekcji}
\newcommand{\kategoria}[1]{\section{#1}} %olreślamy nazwę kateforii zadań
\newcommand{\zadStart}[1]{\begin{zad}#1\newline} %oznaczenie początku zadania
\newcommand{\zadStop}{\end{zad}}   %oznaczenie końca zadania
%Makra opcjonarne (nie muszą występować):
\newcommand{\rozwStart}[2]{\noindent \textbf{Rozwiązanie (autor #1 , recenzent #2): }\newline} %oznaczenie początku rozwiązania, opcjonarnie można wprowadzić informację o autorze rozwiązania zadania i recenzencie poprawności wykonania rozwiązania zadania
\newcommand{\rozwStop}{\newline}                                            %oznaczenie końca rozwiązania
\newcommand{\odpStart}{\noindent \textbf{Odpowiedź:}\newline}    %oznaczenie początku odpowiedzi końcowej (wypisanie wyniku)
\newcommand{\odpStop}{\newline}                                             %oznaczenie końca odpowiedzi końcowej (wypisanie wyniku)
\newcommand{\testStart}{\noindent \textbf{Test:}\newline} %ewentualne możliwe opcje odpowiedzi testowej: A. ? B. ? C. ? D. ? itd.
\newcommand{\testStop}{\newline} %koniec wprowadzania odpowiedzi testowych
\newcommand{\kluczStart}{\noindent \textbf{Test poprawna odpowiedź:}\newline} %klucz, poprawna odpowiedź pytania testowego (jedna literka): A lub B lub C lub D itd.
\newcommand{\kluczStop}{\newline} %koniec poprawnej odpowiedzi pytania testowego 
\newcommand{\wstawGrafike}[2]{\begin{figure}[h] \includegraphics[scale=#2] {#1} \end{figure}} %gdyby była potrzeba wstawienia obrazka, parametry: nazwa pliku, skala (jak nie wiesz co wpisać, to wpisz 1)

\begin{document}
\maketitle


\kategoria{Wikieł/Z1.134a}
\zadStart{Zadanie z Wikieł Z 1.134a) moja wersja nr [nrWersji]}
%[a]:[2,3,5,6,7,8,9,10]
%[c]:[1,2,3,5,6,7,8,9,10]
%[d]=random.randrange(2,10,2)
Wyznaczyć dziedzinę naturalną oraz zbadać parzystość (nieparzystość) funkcji:\\
$a)f(x)=[a]^{x}+[a]^{-x}+\frac{\cos x}{x^{[d]}+[c]}$
\zadStop
\rozwStart{Wojciech Przybylski}{Maja Szabłowska}
I Wyznaczanie dziedziny naturalnej:\\
$$x^{[d]}+[c]\neq0 \Rightarrow x^{[d]}\neq-[c]\hspace{3mm}x\in\mathbb{R} $$
$$\mbox{Dziedzina funckji }f(x)\in \mathbb{R}$$
II Badanie parzystości (nieparzystości) funkcji:
$$f(-x)=[a]^{(-x)}+[a]^{-(-x)}+\frac{\cos (-x)}{(-x)^{[d]}+[c]}=$$
$$=[a]^{-x}+[a]^{x}+\frac{\cos x}{x^{[d]}+[c]}=f(x)$$
$$\mbox{Funkcja }f(x)\mbox{ jest parzysta}$$
\rozwStop
\odpStart
$\mbox{Dziedzina funckji }f(x)\in \mathbb{R} \mbox{ oraz } f(x)\mbox{ jest parzysta.}$ 
\odpStop
\testStart
A. $\mbox{Dziedzina funckji }f(x)\in \mathbb{R} \mbox{ oraz } f(x)\mbox{ jest parzysta.}$ \\
B. $\mbox{Dziedzina funckji }f(x)\in \mathbb{R} \mbox{ oraz } f(x)\mbox{ jest nieparzysta.}$ \\
C. $\mbox{Dziedzina funckji }f(x)\in \mathbb{R} \mbox{ oraz } f(x)\mbox{ nie jest parzysta.}$ \\
D. $\mbox{Dziedzina funckji }f(x)\in \mathbb{R}\backslash\{0\} \mbox{ oraz } f(x)\mbox{ nie jest parzysta.}$ \\
E. $\mbox{Dziedzina funckji }f(x)\in \mathbb{R}{\backslash}\{-1,1\} \mbox{ oraz } f(x)\mbox{ jest parzysta.}$ \\
F. $\mbox{Dziedzina funckji }f(x)\in \mathbb{N} \mbox{ oraz } f(x)\mbox{ jest parzysta.}$ \\
\testStop
\kluczStart
A
\kluczStop



\end{document}