\documentclass[12pt, a4paper]{article}
\usepackage[utf8]{inputenc}
\usepackage{polski}
\usepackage{amsthm}  %pakiet do tworzenia twierdzeń itp.
\usepackage{amsmath} %pakiet do niektórych symboli matematycznych
\usepackage{amssymb} %pakiet do symboli mat., np. \nsubseteq
\usepackage{amsfonts}
\usepackage{graphicx} %obsługa plików graficznych z rozszerzeniem png, jpg
\theoremstyle{definition} %styl dla definicji
\newtheorem{zad}{} 
\title{Multizestaw zadań}
\author{Radosław Grzyb}
%\date{\today}
\date{}
\newcounter{liczniksekcji}
\newcommand{\kategoria}[1]{\section{#1}} %olreślamy nazwę kateforii zadań
\newcommand{\zadStart}[1]{\begin{zad}#1\newline} %oznaczenie początku zadania
\newcommand{\zadStop}{\end{zad}}   %oznaczenie końca zadania
%Makra opcjonarne (nie muszą występować):
\newcommand{\rozwStart}[2]{\noindent \textbf{Rozwiązanie (autor #1 , recenzent #2): }\newline} %oznaczenie początku rozwiązania, opcjonarnie można wprowadzić informację o autorze rozwiązania zadania i recenzencie poprawności wykonania rozwiązania zadania
\newcommand{\rozwStop}{\newline}                                            %oznaczenie końca rozwiązania
\newcommand{\odpStart}{\noindent \textbf{Odpowiedź:}\newline}    %oznaczenie początku odpowiedzi końcowej (wypisanie wyniku)
\newcommand{\odpStop}{\newline}                                             %oznaczenie końca odpowiedzi końcowej (wypisanie wyniku)
\newcommand{\testStart}{\noindent \textbf{Test:}\newline} %ewentualne możliwe opcje odpowiedzi testowej: A. ? B. ? C. ? D. ? itd.
\newcommand{\testStop}{\newline} %koniec wprowadzania odpowiedzi testowych
\newcommand{\kluczStart}{\noindent \textbf{Test poprawna odpowiedź:}\newline} %klucz, poprawna odpowiedź pytania testowego (jedna literka): A lub B lub C lub D itd.
\newcommand{\kluczStop}{\newline} %koniec poprawnej odpowiedzi pytania testowego 
\newcommand{\wstawGrafike}[2]{\begin{figure}[h] \includegraphics[scale=#2] {#1} \end{figure}} %gdyby była potrzeba wstawienia obrazka, parametry: nazwa pliku, skala (jak nie wiesz co wpisać, to wpisz 1)
\begin{document}
\maketitle
\kategoria{Wikieł/P1.16}
\zadStart{Zadanie z Wikieł P 1.16 moja wersja nr [nrWersji]}
%[p1]:[1,3,5,7,9]
%[p2]:[3,5,7,9]
%[p3]:[1,2,3,4,5,6,7,8,9]
%[p4]:[1,2,3,4,5,6,7,8,9]
%[p4]>[p3]
%[max]=([p2]/2)*([p1]+[p3])
%[min]=([p2]/2)*([p1]-[p4])
%[w1]=[max]+3
%[w2]=[min]-3
Wyznaczyć najmniejszą i największą wartość funkcji $f(x)=\frac{[p2]}{2}([p1]-x)$, gdy $x\in\langle-[p3],[p4]\rangle$
\zadStop
\rozwStart{Radosław Grzyb}{}
Zauważmy, że jest to funkcja liniowa o ujemnym współczynniku $a$. Zatem dla większych argumentów osiągać będzie mniejsze wartości.\\
Minimum:
$$f([p4])=\frac{[p2]}{2}([p1]-[p4])=[min]$$
Maksimum:
$$f(-[p3])=\frac{[p2]}{2}([p1]+[p3])=[max]$$
\rozwStop
\odpStart
$$[min] \wedge [max]$$
\odpStop
\testStart
A.$$[min] \wedge [w1]$$
B.$$[min] \wedge [max]$$
C.$$[w2] \wedge [max]$$
D.$$[w2] \wedge [w1]$$
\testStop
\kluczStart
B
\kluczStop
\end{document}
