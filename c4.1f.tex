\documentclass[12pt, a4paper]{article}
\usepackage[utf8]{inputenc}
\usepackage{polski}

\usepackage{amsthm}  %pakiet do tworzenia twierdzeń itp.
\usepackage{amsmath} %pakiet do niektórych symboli matematycznych
\usepackage{amssymb} %pakiet do symboli mat., np. \nsubseteq
\usepackage{amsfonts}
\usepackage{graphicx} %obsługa plików graficznych z rozszerzeniem png, jpg
\theoremstyle{definition} %styl dla definicji
\newtheorem{zad}{} 
\title{Multizestaw zadań}
\author{Robert Fidytek}
%\date{\today}
\date{}
\newcounter{liczniksekcji}
\newcommand{\kategoria}[1]{\section{#1}} %olreślamy nazwę kateforii zadań
\newcommand{\zadStart}[1]{\begin{zad}#1\newline} %oznaczenie początku zadania
\newcommand{\zadStop}{\end{zad}}   %oznaczenie końca zadania
%Makra opcjonarne (nie muszą występować):
\newcommand{\rozwStart}[2]{\noindent \textbf{Rozwiązanie (autor #1 , recenzent #2): }\newline} %oznaczenie początku rozwiązania, opcjonarnie można wprowadzić informację o autorze rozwiązania zadania i recenzencie poprawności wykonania rozwiązania zadania
\newcommand{\rozwStop}{\newline}                                            %oznaczenie końca rozwiązania
\newcommand{\odpStart}{\noindent \textbf{Odpowiedź:}\newline}    %oznaczenie początku odpowiedzi końcowej (wypisanie wyniku)
\newcommand{\odpStop}{\newline}                                             %oznaczenie końca odpowiedzi końcowej (wypisanie wyniku)
\newcommand{\testStart}{\noindent \textbf{Test:}\newline} %ewentualne możliwe opcje odpowiedzi testowej: A. ? B. ? C. ? D. ? itd.
\newcommand{\testStop}{\newline} %koniec wprowadzania odpowiedzi testowych
\newcommand{\kluczStart}{\noindent \textbf{Test poprawna odpowiedź:}\newline} %klucz, poprawna odpowiedź pytania testowego (jedna literka): A lub B lub C lub D itd.
\newcommand{\kluczStop}{\newline} %koniec poprawnej odpowiedzi pytania testowego 
\newcommand{\wstawGrafike}[2]{\begin{figure}[h] \includegraphics[scale=#2] {#1} \end{figure}} %gdyby była potrzeba wstawienia obrazka, parametry: nazwa pliku, skala (jak nie wiesz co wpisać, to wpisz 1)

\begin{document}
\maketitle


\kategoria{Dymkowska, Beger/c4.1f}
\zadStart{Zadanie z Dymkowskiej, Beger C 4.1f) moja wersja nr [nrWersji]}
%[p1]:[2,3,4,5,6,7,8,9,10]
%[p2]:[2,3,4,5,6,7,8,9,10]
%[p3]:[2,3,4,5,6,7,8,9,10]
%[a]=([p2]-1)*[p3]
%[b]=[p1]*[p3]
%[c]=[p1]*([p2]-1)
%[up]=[a]*pow([p1],3)+[b]*(pow([p2],3)-1)+[c]*pow([p3],3)
%[wyn]=int([up]/3)
%math.gcd([wyn],3)==3
Obliczyć całkę potrójną po prostopadłościanie P $$\iiint_P x^2+y^2+z^2\ dxdydz, P: 0 \leq x \leq [p1], 1 \leq y \leq [p2], 0 \leq z \leq [p3]$$
\zadStop
\rozwStart{Jakub Janik}{}
$$\iiint_P x^2+y^2+z^2\ dxdydz=$$
$$=\int_{0}^{[p1]}x^2\ dx\cdot\int_{1}^{[p2]}dy\cdot\int_{0}^{[p3]}dz+\int_{0}^{[p1]}dx\cdot\int_{1}^{[p2]}y^2\ dy\cdot\int_{0}^{[p3]}dz+\int_{0}^{[p1]}dx\cdot\int_{1}^{[p2]}dy\cdot\int_{0}^{[p3]}z^2\ dz=$$
$$=\frac{[a]}{3}x^3\Big|_0^{[p1]}+\frac{[b]}{3}y^3\Big|_1^{[p2]}+\frac{[c]}{3}z^3\Big|_0^{[p3]}=[wyn]$$
\rozwStop
\odpStart
$[wyn]$
\odpStop
\testStart
A.$[wyn]$
B.$0$
C.$-[wyn]$
D.$\infty$
\testStop
\kluczStart
A
\kluczStop



\end{document}