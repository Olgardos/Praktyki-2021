\documentclass[12pt, a4paper]{article}
\usepackage[utf8]{inputenc}
\usepackage{polski}
\usepackage{amsthm}  %pakiet do tworzenia twierdzeń itp.
\usepackage{amsmath} %pakiet do niektórych symboli matematycznych
\usepackage{amssymb} %pakiet do symboli mat., np. \nsubseteq
\usepackage{amsfonts}
\usepackage{graphicx} %obsługa plików graficznych z rozszerzeniem png, jpg
\theoremstyle{definition} %styl dla definicji
\newtheorem{zad}{} 
\title{Multizestaw zadań}
\author{Radosław Grzyb}
%\date{\today}
\date{}
\newcounter{liczniksekcji}
\newcommand{\kategoria}[1]{\section{#1}} %olreślamy nazwę kateforii zadań
\newcommand{\zadStart}[1]{\begin{zad}#1\newline} %oznaczenie początku zadania
\newcommand{\zadStop}{\end{zad}}   %oznaczenie końca zadania
%Makra opcjonarne (nie muszą występować):
\newcommand{\rozwStart}[2]{\noindent \textbf{Rozwiązanie (autor #1 , recenzent #2): }\newline} %oznaczenie początku rozwiązania, opcjonarnie można wprowadzić informację o autorze rozwiązania zadania i recenzencie poprawności wykonania rozwiązania zadania
\newcommand{\rozwStop}{\newline}                                            %oznaczenie końca rozwiązania
\newcommand{\odpStart}{\noindent \textbf{Odpowiedź:}\newline}    %oznaczenie początku odpowiedzi końcowej (wypisanie wyniku)
\newcommand{\odpStop}{\newline}                                             %oznaczenie końca odpowiedzi końcowej (wypisanie wyniku)
\newcommand{\testStart}{\noindent \textbf{Test:}\newline} %ewentualne możliwe opcje odpowiedzi testowej: A. ? B. ? C. ? D. ? itd.
\newcommand{\testStop}{\newline} %koniec wprowadzania odpowiedzi testowych
\newcommand{\kluczStart}{\noindent \textbf{Test poprawna odpowiedź:}\newline} %klucz, poprawna odpowiedź pytania testowego (jedna literka): A lub B lub C lub D itd.
\newcommand{\kluczStop}{\newline} %koniec poprawnej odpowiedzi pytania testowego 
\newcommand{\wstawGrafike}[2]{\begin{figure}[h] \includegraphics[scale=#2] {#1} \end{figure}} %gdyby była potrzeba wstawienia obrazka, parametry: nazwa pliku, skala (jak nie wiesz co wpisać, to wpisz 1)
\begin{document}
\maketitle
\kategoria{Beger/c1.12i}
\zadStart{Zadanie z Beger C 1.12i moja wersja nr [nrWersji]}
%[p1]:[1,2,3,4,5,6,7,8,9]
%[p2]:[1,2,3,4,5,6,7,8,9]
%[c]=[p1]+[p2]
%[d]=[c]*(-2)
%[e]=[d]*[p2]
Obliczyć całkę funkcji niewymiernej:
$$\int \frac{\sqrt{x+1}+[p1]}{\sqrt{x+1}-[p2]}\,dx$$
\zadStop
\rozwStart{Radosław Grzyb}{}
Dokonujemy podstawienia:
$$u=x+1\implies du=dx$$
Po podstawieniu przekształcamy nieco naszą funkcję podcałkową:
$$\int \frac{\sqrt{u}+[p1]}{\sqrt{u}-[p2]}\,du=\int \frac{\sqrt{u}-[p2]+[c]}{\sqrt{u}-[p2]}\,du=\int 1 \,du+[c]\int\frac{1}{\sqrt{u}-[p2]}\,du$$
Otrzymaliśmy dwie całki do policzenia. Pierwsza jest łatwa i wynosi:
$$\int 1 \,du=u=x+1$$
Druga zaś jest nieco trudniejsza. Policzymy ją dokonując kolejnego podstawienia:
$$t=\sqrt{u}-[p2]\implies dt=-\frac{1}{2\sqrt{u}}du\implies du=-2\sqrt{u}dt$$
$$\sqrt{u}=t+[p2]$$
Otrzymujemy wówczas: 
$$[c]\int\frac{1}{\sqrt{u}-[p2]}\cdot(-2)\sqrt{u}\,dt=[d]\int\frac{t+[p2]}{t}\,dt=[d]\int1\,dt[e]\int\frac{1}{t}\,dt=$$
$$=[d]t[e]\ln|t|+C=[d](\sqrt{u}-[p2])[e]\ln|\sqrt{u}-[p2]|+C=[d](\sqrt{x+1}-[p2])[e]\ln|\sqrt{x+1}-[p2]|+C$$\\
Łączac nasze dwie całki otrzymujemy finalny wynik:
$$x+1[d](\sqrt{x+1}-[p2])[e]\ln|\sqrt{x+1}-[p2]|+C$$
\rozwStop
\odpStart
$$x+1[d](\sqrt{x+1}-[p2])[e]\ln|\sqrt{x+1}-[p2]|+C$$
\odpStop
\testStart
A.$$[d](\sqrt{x+1}-[p2])[e]\ln|\sqrt{x+1}|+C$$
B.$$x^{2}[d](\sqrt{x+1}-[p2])[e]\ln|\sqrt{x+1}-[p2]|+C$$
C.$$x+1[d](\sqrt{x+1}-[p2])[e]\ln|\sqrt{x+1}-[p2]|+C$$
D.$$2(\sqrt[4]{x-[p1]}+[p2])^{3}+C$$
\testStop
\kluczStart
C
\kluczStop
\end{document}