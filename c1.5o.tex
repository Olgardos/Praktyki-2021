\documentclass[12pt, a4paper]{article}
\usepackage[utf8]{inputenc}
\usepackage{polski}
\usepackage{amsthm}  %pakiet do tworzenia twierdzeń itp.
\usepackage{amsmath} %pakiet do niektórych symboli matematycznych
\usepackage{amssymb} %pakiet do symboli mat., np. \nsubseteq
\usepackage{amsfonts}
\usepackage{graphicx} %obsługa plików graficznych z rozszerzeniem png, jpg
\theoremstyle{definition} %styl dla definicji
\newtheorem{zad}{} 
\title{Multizestaw zadań}
\author{Radosław Grzyb}
%\date{\today}
\date{}
\newcounter{liczniksekcji}
\newcommand{\kategoria}[1]{\section{#1}} %olreślamy nazwę kateforii zadań
\newcommand{\zadStart}[1]{\begin{zad}#1\newline} %oznaczenie początku zadania
\newcommand{\zadStop}{\end{zad}}   %oznaczenie końca zadania
%Makra opcjonarne (nie muszą występować):
\newcommand{\rozwStart}[2]{\noindent \textbf{Rozwiązanie (autor #1 , recenzent #2): }\newline} %oznaczenie początku rozwiązania, opcjonarnie można wprowadzić informację o autorze rozwiązania zadania i recenzencie poprawności wykonania rozwiązania zadania
\newcommand{\rozwStop}{\newline}                                            %oznaczenie końca rozwiązania
\newcommand{\odpStart}{\noindent \textbf{Odpowiedź:}\newline}    %oznaczenie początku odpowiedzi końcowej (wypisanie wyniku)
\newcommand{\odpStop}{\newline}                                             %oznaczenie końca odpowiedzi końcowej (wypisanie wyniku)
\newcommand{\testStart}{\noindent \textbf{Test:}\newline} %ewentualne możliwe opcje odpowiedzi testowej: A. ? B. ? C. ? D. ? itd.
\newcommand{\testStop}{\newline} %koniec wprowadzania odpowiedzi testowych
\newcommand{\kluczStart}{\noindent \textbf{Test poprawna odpowiedź:}\newline} %klucz, poprawna odpowiedź pytania testowego (jedna literka): A lub B lub C lub D itd.
\newcommand{\kluczStop}{\newline} %koniec poprawnej odpowiedzi pytania testowego 
\newcommand{\wstawGrafike}[2]{\begin{figure}[h] \includegraphics[scale=#2] {#1} \end{figure}} %gdyby była potrzeba wstawienia obrazka, parametry: nazwa pliku, skala (jak nie wiesz co wpisać, to wpisz 1)
\begin{document}
\maketitle
\kategoria{Beger/c1.5o}
\zadStart{Zadanie z Beger C 1.5o moja wersja nr [nrWersji]}
%[p1]:[2,3,5,6,7,10]
%[p2]:[4,9,16,25,49,64]
%[p11]=2*[p1]
%[p22]=int(math.sqrt([p2]))
%[s1]=[p1]*[p22]
Obliczyć, całkując przez podstawienie.
$$\int \frac{1}{e^{[p1]x}+[p2]e^{-[p1]x}} \,dx$$
\zadStop
\rozwStart{Radosław Grzyb}{}
Mnożymy licznik i mianownik naszego ułamka przez $e^{[p1]x}$:
$$\int \frac{e^{[p1]x}}{e^{[p11]x}+[p2]} \,dx$$
Podstawiamy: 
$$u=e^{[p1]x}\implies du=[p1]e^{[p1]x}dx \implies dx=\frac{e^{-[p1]x}}{[p1]}du$$
Po podstawieniu otrzymujemy:
$$\int \frac{e^{[p1]x}}{(e^{[p1]x})^{2}+[p2]}\cdot\frac{e^{-[p1]x}}{[p1]} \,du=\frac{1}{[p1]}\int \frac{1}{u^{2}+[p2]} \,du$$
Wykorzystajmy teraz gotowy wzór na całkę $\int\frac{1}{x^2+a^2}dx=\frac{1}{a}\arctan\frac{x}{a}+C$ i otrzymajmy finalny wynik:
$$\frac{1}{[p1]}\cdot\frac{1}{[p22]}\arctan\frac{u}{[p22]}+C=\frac{1}{[s1]}\arctan\frac{e^{[p1]x}}{[p22]}+C$$
\rozwStop
\odpStart
$$\frac{1}{[s1]}\arctan\frac{e^{[p1]x}}{[p22]}+C$$
\odpStop
\testStart
A.$$\frac{3}{[p2]}\arcsin(\frac{3x}{[p2]})+C$$
B.$$\arcsin(\frac{x}{\sqrt{[p2]}})+C$$
C.$$\frac{1}{[s1]}\arctan\frac{e^{[p1]x}}{[p22]}+C$$
D.$$\frac{[p1]}{3}\ln(|\frac{3x}{[p2]}|)+C$$
\testStop
\kluczStart
C
\kluczStop
\end{document}