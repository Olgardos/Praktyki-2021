\documentclass[12pt, a4paper]{article}
\usepackage[utf8]{inputenc}
\usepackage{polski}

\usepackage{amsthm}  %pakiet do tworzenia twierdzeń itp.
\usepackage{amsmath} %pakiet do niektórych symboli matematycznych
\usepackage{amssymb} %pakiet do symboli mat., np. \nsubseteq
\usepackage{amsfonts}
\usepackage{graphicx} %obsługa plików graficznych z rozszerzeniem png, jpg
\theoremstyle{definition} %styl dla definicji
\newtheorem{zad}{} 
\title{Multizestaw zadań}
\author{Robert Fidytek}
%\date{\today}
\date{}
\newcounter{liczniksekcji}
\newcommand{\kategoria}[1]{\section{#1}} %olreślamy nazwę kateforii zadań
\newcommand{\zadStart}[1]{\begin{zad}#1\newline} %oznaczenie początku zadania
\newcommand{\zadStop}{\end{zad}}   %oznaczenie końca zadania
%Makra opcjonarne (nie muszą występować):
\newcommand{\rozwStart}[2]{\noindent \textbf{Rozwiązanie (autor #1 , recenzent #2): }\newline} %oznaczenie początku rozwiązania, opcjonarnie można wprowadzić informację o autorze rozwiązania zadania i recenzencie poprawności wykonania rozwiązania zadania
\newcommand{\rozwStop}{\newline}                                            %oznaczenie końca rozwiązania
\newcommand{\odpStart}{\noindent \textbf{Odpowiedź:}\newline}    %oznaczenie początku odpowiedzi końcowej (wypisanie wyniku)
\newcommand{\odpStop}{\newline}                                             %oznaczenie końca odpowiedzi końcowej (wypisanie wyniku)
\newcommand{\testStart}{\noindent \textbf{Test:}\newline} %ewentualne możliwe opcje odpowiedzi testowej: A. ? B. ? C. ? D. ? itd.
\newcommand{\testStop}{\newline} %koniec wprowadzania odpowiedzi testowych
\newcommand{\kluczStart}{\noindent \textbf{Test poprawna odpowiedź:}\newline} %klucz, poprawna odpowiedź pytania testowego (jedna literka): A lub B lub C lub D itd.
\newcommand{\kluczStop}{\newline} %koniec poprawnej odpowiedzi pytania testowego 
\newcommand{\wstawGrafike}[2]{\begin{figure}[h] \includegraphics[scale=#2] {#1} \end{figure}} %gdyby była potrzeba wstawienia obrazka, parametry: nazwa pliku, skala (jak nie wiesz co wpisać, to wpisz 1)

\begin{document}
\maketitle



\kategoria{Dymkowska,Beger/C2.2n}
\zadStart{Zadanie z Dymkowska,Beger C 2.2 n) moja wersja nr [nrWersji]}
%[a]:[2,4,6,8,10,12]
%[b1]=[a]/2
%[b]=int([b1])
%[c]=[b]/6
Obliczyć całkę oznaczoną $\displaystyle \int_{0}^{\frac{1}{\sqrt{2}}} \frac{[a]x}{\sqrt{(1-x^4)}} \ dx $
\zadStop
\rozwStart{Mirella Narewska}{}
$$\int_{0}^{\frac{1}{\sqrt{2}}} \frac{[a]x}{\sqrt{(1-x^4)}} \ dx = [a]\int_{0}^{\frac{1}{\sqrt{2}}} \frac{x}{\sqrt{(1-x^4)}} \ dx$$
$$\text{Całkujemy  całkę przez podstawienie: }$$
$$u=x^2 \Rightarrow du=2x dx $$
$$\text{Zmieniają się także granice całkowania: }$$
$$[b] \int_{0}^{\frac{1}{2}}  \frac{1}{\sqrt{1-u^2}} \ dx=[b]arcsin(u)|_{0}^{\frac{1}{2}}=$$
$$[b]\left(arcsin(\frac{1}{2})-arcsin(0)\right)=[c]\pi$$
\odpStart
$$[c]\pi$$
\odpStop
\testStart
A.$[c]\pi$
\\
B.$\pi$
\\
C.$0$
\\
D.$arccos([a])$
\testStop
\kluczStart
A
\kluczStop


\end{document}