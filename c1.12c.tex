\documentclass[12pt, a4paper]{article}
\usepackage[utf8]{inputenc}
\usepackage{polski}
\usepackage{amsthm}  %pakiet do tworzenia twierdzeń itp.
\usepackage{amsmath} %pakiet do niektórych symboli matematycznych
\usepackage{amssymb} %pakiet do symboli mat., np. \nsubseteq
\usepackage{amsfonts}
\usepackage{graphicx} %obsługa plików graficznych z rozszerzeniem png, jpg
\theoremstyle{definition} %styl dla definicji
\newtheorem{zad}{} 
\title{Multizestaw zadań}
\author{Radosław Grzyb}
%\date{\today}
\date{}
\newcounter{liczniksekcji}
\newcommand{\kategoria}[1]{\section{#1}} %olreślamy nazwę kateforii zadań
\newcommand{\zadStart}[1]{\begin{zad}#1\newline} %oznaczenie początku zadania
\newcommand{\zadStop}{\end{zad}}   %oznaczenie końca zadania
%Makra opcjonarne (nie muszą występować):
\newcommand{\rozwStart}[2]{\noindent \textbf{Rozwiązanie (autor #1 , recenzent #2): }\newline} %oznaczenie początku rozwiązania, opcjonarnie można wprowadzić informację o autorze rozwiązania zadania i recenzencie poprawności wykonania rozwiązania zadania
\newcommand{\rozwStop}{\newline}                                            %oznaczenie końca rozwiązania
\newcommand{\odpStart}{\noindent \textbf{Odpowiedź:}\newline}    %oznaczenie początku odpowiedzi końcowej (wypisanie wyniku)
\newcommand{\odpStop}{\newline}                                             %oznaczenie końca odpowiedzi końcowej (wypisanie wyniku)
\newcommand{\testStart}{\noindent \textbf{Test:}\newline} %ewentualne możliwe opcje odpowiedzi testowej: A. ? B. ? C. ? D. ? itd.
\newcommand{\testStop}{\newline} %koniec wprowadzania odpowiedzi testowych
\newcommand{\kluczStart}{\noindent \textbf{Test poprawna odpowiedź:}\newline} %klucz, poprawna odpowiedź pytania testowego (jedna literka): A lub B lub C lub D itd.
\newcommand{\kluczStop}{\newline} %koniec poprawnej odpowiedzi pytania testowego 
\newcommand{\wstawGrafike}[2]{\begin{figure}[h] \includegraphics[scale=#2] {#1} \end{figure}} %gdyby była potrzeba wstawienia obrazka, parametry: nazwa pliku, skala (jak nie wiesz co wpisać, to wpisz 1)
\begin{document}
\maketitle
\kategoria{Beger/c1.12c}
\zadStart{Zadanie z Beger C 1.12c moja wersja nr [nrWersji]}
%[p1]:[1,2,4,5,6,7,8,10,11]
%[c]=[p1]*4
Obliczyć całkę funkcji niewymiernej:
$$\int \frac{\sqrt{x}}{[p1]-\sqrt[4]{x^3}}\,dx$$
\zadStop
\rozwStart{Radosław Grzyb}{}
Przekształćmy lekko naszą funkcję podcałkową:
$$\int \frac{\sqrt{x}}{[p1]-\sqrt[4]{x^3}}\,dx=-\int \frac{x^{\frac{1}{2}}}{x^{\frac{3}{4}}-[p1]}\,dx$$
Podstawmy teraz:
$$u=x^{\frac{3}{4}}-[p1]\implies du=\frac{3}{4x^{\frac{1}{4}}}dx\implies dx=\frac{4x^{\frac{1}{4}}}{3}du$$
Otrzymujemy wówczas:
$$-\int \frac{x^{\frac{1}{2}}}{x^{\frac{3}{4}}-[p1]}\cdot\frac{4x^{\frac{1}{4}}}{3}\,du=-\frac{4}{3}\int \frac{x^{\frac{1}{2}+\frac{1}{4}}}{x^{\frac{3}{4}}-[p1]}\,du=-\frac{4}{3}\int \frac{x^{\frac{3}{4}}}{x^{\frac{3}{4}}-[p1]}\,du=-\frac{4}{3}\int \frac{u+[p1]}{u}\,du$$
Otrzymaliśmy w miarę prostą do policzenia całkę:
$$-\frac{4}{3}\int \frac{u+[p1]}{u}\,du=-\frac{4}{3}\int1\,du-\frac{[c]}{3}\int\frac{1}{u}\,du=-\frac{4}{3}u-\frac{[c]}{3}\ln|u|+C$$
A więc otrzymujemy finalny wynik:
$$-\frac{4}{3}x^{\frac{3}{4}}-\frac{[c]}{3}\ln|x^{\frac{3}{4}}|+C$$
Otrzymujemy finalny wynik:
$$-\frac{4}{3}x^{\frac{3}{4}}-\frac{[c]}{3}\ln|x^{\frac{3}{4}}|+C$$
\rozwStop
\odpStart
$$-\frac{4}{3}x^{\frac{3}{4}}-\frac{[c]}{3}\ln|x^{\frac{3}{4}}|+C$$
\odpStop
\testStart
A.$$\frac{4}{3}x^{\frac{3}{4}}-\frac{[c]}{3}\ln|x^{\frac{3}{4}}|+C$$
B.$$-x\frac{4}{3}x^{\frac{3}{4}}-\frac{[c]}{3}\ln|x^{\frac{3}{4}}|+C$$
C.$$\frac{[c]}{\sqrt{[p1]}}\arctan\frac{\sqrt{x}}{\sqrt{[p1]}}+C$$
D.$$-\frac{4}{3}x^{\frac{3}{4}}-\frac{[c]}{3}\ln|x^{\frac{3}{4}}|+C$$
\testStop
\kluczStart
D
\kluczStop
\end{document}