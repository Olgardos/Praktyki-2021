\documentclass[12pt, a4paper]{article}
\usepackage[utf8]{inputenc}
\usepackage{polski}

\usepackage{amsthm}  %pakiet do tworzenia twierdzeń itp.
\usepackage{amsmath} %pakiet do niektórych symboli matematycznych
\usepackage{amssymb} %pakiet do symboli mat., np. \nsubseteq
\usepackage{amsfonts}
\usepackage{graphicx} %obsługa plików graficznych z rozszerzeniem png, jpg
\theoremstyle{definition} %styl dla definicji
\newtheorem{zad}{} 
\title{Multizestaw zadań}
\author{Robert Fidytek}
%\date{\today}
\date{}
\newcounter{liczniksekcji}
\newcommand{\kategoria}[1]{\section{#1}} %olreślamy nazwę kateforii zadań
\newcommand{\zadStart}[1]{\begin{zad}#1\newline} %oznaczenie początku zadania
\newcommand{\zadStop}{\end{zad}}   %oznaczenie końca zadania
%Makra opcjonarne (nie muszą występować):
\newcommand{\rozwStart}[2]{\noindent \textbf{Rozwiązanie (autor #1 , recenzent #2): }\newline} %oznaczenie początku rozwiązania, opcjonarnie można wprowadzić informację o autorze rozwiązania zadania i recenzencie poprawności wykonania rozwiązania zadania
\newcommand{\rozwStop}{\newline}                                            %oznaczenie końca rozwiązania
\newcommand{\odpStart}{\noindent \textbf{Odpowiedź:}\newline}    %oznaczenie początku odpowiedzi końcowej (wypisanie wyniku)
\newcommand{\odpStop}{\newline}                                             %oznaczenie końca odpowiedzi końcowej (wypisanie wyniku)
\newcommand{\testStart}{\noindent \textbf{Test:}\newline} %ewentualne możliwe opcje odpowiedzi testowej: A. ? B. ? C. ? D. ? itd.
\newcommand{\testStop}{\newline} %koniec wprowadzania odpowiedzi testowych
\newcommand{\kluczStart}{\noindent \textbf{Test poprawna odpowiedź:}\newline} %klucz, poprawna odpowiedź pytania testowego (jedna literka): A lub B lub C lub D itd.
\newcommand{\kluczStop}{\newline} %koniec poprawnej odpowiedzi pytania testowego 
\newcommand{\wstawGrafike}[2]{\begin{figure}[h] \includegraphics[scale=#2] {#1} \end{figure}} %gdyby była potrzeba wstawienia obrazka, parametry: nazwa pliku, skala (jak nie wiesz co wpisać, to wpisz 1)

\begin{document}
\maketitle

\kategoria{Wikieł/Z5.40}

\zadStart{Zadanie z Wikieł Z 5.40 moja wersja nr [nrWersji]}
%[b]:[3,6,9,12,15,18,21,24,27]
%[c]:[2,3,4,5,6]
%[e]=48*[b]
%[f]=int([e]/36)
%math.gcd([b],2)==1
Sprawdzić, dla jakich wartości parametru $a$ krzywa:
$$y = x^4 + ax^3 + \frac{[b]}{2}x^2 + [c]a$$
jest wypukła dla dowolnej liczby rzeczywistej.
\zadStop

\rozwStart{Natalia Danieluk}{}
Dziedzina funkcji: $\quad \mathcal{D}_f=\mathbb{R}$. \\
Zaczynamy od obliczenia pochodnych: \\
$$f'(x) = 4x^3 + 3ax^2 + [b]x, \quad f''(x) = 12x^2 + 6ax + [b]$$
i określamy ich dziedziny: $\quad \mathcal{D}_{f'}=\mathcal{D}_{f''}=\mathbb{R}$. \\
Chcemy, aby krzywa była wypukła dla dowolnego $x \in \mathbb{R}$, zatem nie mogą istnieć miejsca zerowe. Aby tak się stalo delta nie może mieć rozwiązań ($\Delta < 0$).
$$\Delta = 36a^2 - [e] < 0 \quad\Leftrightarrow\quad a^2 < [f] \quad\Leftrightarrow\quad \mid a \mid < \sqrt{[f]}$$
\rozwStop

\odpStart
$\mid a \mid < \sqrt{[f]}$
\odpStop

\testStart
A. $\mid a \mid < \sqrt{[f]}$
B. $\mid a \mid > \sqrt{[f]}$
C. $\mid a \mid < [f]$
D. $a < [f]$
\testStop

\kluczStart
A
\kluczStop

\end{document}