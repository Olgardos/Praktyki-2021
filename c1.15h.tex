\documentclass[12pt, a4paper]{article}
\usepackage[utf8]{inputenc}
\usepackage{polski}
\usepackage{amsthm}  %pakiet do tworzenia twierdzeń itp.
\usepackage{amsmath} %pakiet do niektórych symboli matematycznych
\usepackage{amssymb} %pakiet do symboli mat., np. \nsubseteq
\usepackage{amsfonts}
\usepackage{graphicx} %obsługa plików graficznych z rozszerzeniem png, jpg
\theoremstyle{definition} %styl dla definicji
\newtheorem{zad}{} 
\title{Multizestaw zadań}
\author{Radosław Grzyb}
%\date{\today}
\date{}
\newcounter{liczniksekcji}
\newcommand{\kategoria}[1]{\section{#1}} %olreślamy nazwę kateforii zadań
\newcommand{\zadStart}[1]{\begin{zad}#1\newline} %oznaczenie początku zadania
\newcommand{\zadStop}{\end{zad}}   %oznaczenie końca zadania
%Makra opcjonarne (nie muszą występować):
\newcommand{\rozwStart}[2]{\noindent \textbf{Rozwiązanie (autor #1 , recenzent #2): }\newline} %oznaczenie początku rozwiązania, opcjonarnie można wprowadzić informację o autorze rozwiązania zadania i recenzencie poprawności wykonania rozwiązania zadania
\newcommand{\rozwStop}{\newline}                                            %oznaczenie końca rozwiązania
\newcommand{\odpStart}{\noindent \textbf{Odpowiedź:}\newline}    %oznaczenie początku odpowiedzi końcowej (wypisanie wyniku)
\newcommand{\odpStop}{\newline}                                             %oznaczenie końca odpowiedzi końcowej (wypisanie wyniku)
\newcommand{\testStart}{\noindent \textbf{Test:}\newline} %ewentualne możliwe opcje odpowiedzi testowej: A. ? B. ? C. ? D. ? itd.
\newcommand{\testStop}{\newline} %koniec wprowadzania odpowiedzi testowych
\newcommand{\kluczStart}{\noindent \textbf{Test poprawna odpowiedź:}\newline} %klucz, poprawna odpowiedź pytania testowego (jedna literka): A lub B lub C lub D itd.
\newcommand{\kluczStop}{\newline} %koniec poprawnej odpowiedzi pytania testowego 
\newcommand{\wstawGrafike}[2]{\begin{figure}[h] \includegraphics[scale=#2] {#1} \end{figure}} %gdyby była potrzeba wstawienia obrazka, parametry: nazwa pliku, skala (jak nie wiesz co wpisać, to wpisz 1)
\begin{document}
\maketitle
\kategoria{Beger/c1.15h}
\zadStart{Zadanie z Beger C 1.15h moja wersja nr [nrWersji]}
%[p1]:[1,4,9,16,25,36]
%[p2]:[2,3,4,5,6,7,8,9]
%[c]=2*[p1]
%[s]=int(math.sqrt([p1]))
Stosując odpowiednie metody całkowania obliczyć całkę:
$$\int \sqrt{[p1]-e^{[p2]x}}\,dx$$
\zadStop
\rozwStart{Radosław Grzyb}{}
Dokonujemy podstawienia:
$$u=[p1]-e^{[p2]x} \implies du=-[p2]e^{[p2]x} dx\implies dx=-\frac{1}{[p2]e^{[p2]x}}du$$
$$[p2]e^{[p2]x}=[p1]-u$$
Otrzymujemy:
$$\int \sqrt{[p1]-e^{[p2]x}}\cdot\frac{-1}{[p2]e^{[p2]x}}\,du=-\int \frac{\sqrt{u}}{[p1]-u}\,du=\int \frac{\sqrt{u}}{u-[p1]}\,du$$
Dokonujemy kolejnego podstawienia:
$$t=\sqrt{u} \implies dt=\frac{1}{2\sqrt{u}}du\implies du=2\sqrt{u}dt$$
$$t^2=u$$
Otrzymujemy:
$$\int \frac{\sqrt{u}}{u-[p1]}\cdot2\sqrt{u}\,dt=2\int \frac{t^2}{t^2-[p1]}\,dt$$
Przekształcamy naszą całkę otrzymując:
$$2\int \frac{t^2}{t^2-[p1]}\,dt=2\int\frac{t^2-[p1]+[p1]}{t^2-[p1]}\,dt=2\int1\,dt+2\int\frac{[p1]}{t^2-[p1]}\,dt=2t+[c]\int\frac{1}{t^2-[p1]}\,dt$$
Do policzenia otrzymanej całki użyjemy gotowego wzoru: $$\int\frac{dx}{x^2-a^2}=\frac{1}{2a}\ln\left|\frac{x-a}{x+a}\right|+C$$
Po paru prostych obliczeniach otrzymujemy finalny wynik:
$$2t+[c]\cdot\frac{1}{2\cdot[s]}\ln\left|\frac{t-[s]}{t+[s]}\right|+C=2t+\frac{[p1]}{[s]}\ln\left|\frac{t-[s]}{t+[s]}\right|+C=2\sqrt{u}+[s]\ln\left|\frac{\sqrt{u}-[s]}{\sqrt{u}+[s]}\right|+C=$$$$=2\sqrt{[p1]-e^{[p2]x}}+[c]\ln\left|\frac{\sqrt{[p1]-e^{[p2]x}}-[s]}{\sqrt{[p1]-e^{[p2]x}}+[s]}\right|+C$$
\rozwStop
\odpStart
$$2\sqrt{[p1]-e^{[p2]x}}+[c]\ln\left|\frac{\sqrt{[p1]-e^{[p2]x}}-[s]}{\sqrt{[p1]-e^{[p2]x}}+[s]}\right|+C$$
\odpStop
\testStart
A.$$\sqrt[[p1]]{x}e^{\sqrt[[p1]]{x}}-e^{\sqrt[[p1]]{x}}+C$$
B.$$\frac{1}{\sqrt{[p1]\pi}}+C$$
C.$$2\sqrt{[p1]-e^{[p2]x}}+[c]\ln\left|\frac{\sqrt{[p1]-e^{[p2]x}}-[s]}{\sqrt{[p1]-e^{[p2]x}}+[s]}\right|+C$$
D.$$\ln\left|\frac{\left(\frac{1}{[p1]}\right)^x-1}{\left(\frac{1}{[p1]}\right)^x+1}\right|+C$$
\testStop
\kluczStart
C
\kluczStop
\end{document}