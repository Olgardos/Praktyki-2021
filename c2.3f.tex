\documentclass[12pt, a4paper]{article}
\usepackage[utf8]{inputenc}
\usepackage{polski}

\usepackage{amsthm}  %pakiet do tworzenia twierdzeń itp.
\usepackage{amsmath} %pakiet do niektórych symboli matematycznych
\usepackage{amssymb} %pakiet do symboli mat., np. \nsubseteq
\usepackage{amsfonts}
\usepackage{graphicx} %obsługa plików graficznych z rozszerzeniem png, jpg
\theoremstyle{definition} %styl dla definicji
\newtheorem{zad}{} 
\title{Multizestaw zadań}
\author{Robert Fidytek}
%\date{\today}
\date{}
\newcounter{liczniksekcji}
\newcommand{\kategoria}[1]{\section{#1}} %olreślamy nazwę kateforii zadań
\newcommand{\zadStart}[1]{\begin{zad}#1\newline} %oznaczenie początku zadania
\newcommand{\zadStop}{\end{zad}}   %oznaczenie końca zadania
%Makra opcjonarne (nie muszą występować):
\newcommand{\rozwStart}[2]{\noindent \textbf{Rozwiązanie (autor #1 , recenzent #2): }\newline} %oznaczenie początku rozwiązania, opcjonarnie można wprowadzić informację o autorze rozwiązania zadania i recenzencie poprawności wykonania rozwiązania zadania
\newcommand{\rozwStop}{\newline}                                            %oznaczenie końca rozwiązania
\newcommand{\odpStart}{\noindent \textbf{Odpowiedź:}\newline}    %oznaczenie początku odpowiedzi końcowej (wypisanie wyniku)
\newcommand{\odpStop}{\newline}                                             %oznaczenie końca odpowiedzi końcowej (wypisanie wyniku)
\newcommand{\testStart}{\noindent \textbf{Test:}\newline} %ewentualne możliwe opcje odpowiedzi testowej: A. ? B. ? C. ? D. ? itd.
\newcommand{\testStop}{\newline} %koniec wprowadzania odpowiedzi testowych
\newcommand{\kluczStart}{\noindent \textbf{Test poprawna odpowiedź:}\newline} %klucz, poprawna odpowiedź pytania testowego (jedna literka): A lub B lub C lub D itd.
\newcommand{\kluczStop}{\newline} %koniec poprawnej odpowiedzi pytania testowego 
\newcommand{\wstawGrafike}[2]{\begin{figure}[h] \includegraphics[scale=#2] {#1} \end{figure}} %gdyby była potrzeba wstawienia obrazka, parametry: nazwa pliku, skala (jak nie wiesz co wpisać, to wpisz 1)

\begin{document}
\maketitle



\kategoria{Dymkowska,Beger/C2.3f}
\zadStart{Zadanie z Dymkowska,Beger C 2.3 f) moja wersja nr [nrWersji]}
%[a]:[2,3,4,5,6,7,8]
%[b]=[a]/2
Obliczyć całkę oznaczoną $\displaystyle \int_{-1}^{0} [a]arcctg(x) \ dx $
\zadStop
\rozwStart{Mirella Narewska}{}
$$\int_{-1}^{0} [a]arcctg{x}  \ dx = [a]\int_{-1}^{0} arcctg(x) \ dx $$
$$\text{Całkujemy całkę przez części: }$$
$$f=arcctg(x) \ \ g'=1$$
$$f'=-\frac{1}{x^{2}+1} \ \ g=x$$
$$=[a]xarcctg(x)|_{-1}^{0} -[a]\int_{-1}^{0} -\frac{x}{x^{2}+1}$$
$$\text{Następnie całkujemy przez podstawienie}$$
$$u=x^2+1 \Rightarrow du=2x \ dx$$
$$-[a]\cdot\frac{3\pi}{4} + [b]\int_{2}^{1} \frac{1}{u} \ dx=-[a]\cdot\frac{3\pi}{4}-[b]\int_{1}^{2} \frac{1}{u}$$
$$=-[a]\cdot\frac{3\pi}{4} -[b]\ln{u}|_{1}^{2}=-[a]\cdot\frac{\pi}{4}-[b]\ln{2}$$

\odpStart
$$-[a]\cdot\frac{\pi}{4}-[b]\ln{2}$$
\odpStop
\testStart
A.$-[a]\cdot\frac{\pi}{4}-[b]\ln{2}$
\\
B.$\pi$
\\
C.$0$
\\
D.$cos([a])$
\testStop
\kluczStart
A
\kluczStop


\end{document}