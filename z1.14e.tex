\documentclass[12pt, a4paper]{article}
\usepackage[utf8]{inputenc}
\usepackage{polski}

\usepackage{amsthm}  %pakiet do tworzenia twierdzeń itp.
\usepackage{amsmath} %pakiet do niektórych symboli matematycznych
\usepackage{amssymb} %pakiet do symboli mat., np. \nsubseteq
\usepackage{amsfonts}
\usepackage{graphicx} %obsługa plików graficznych z rozszerzeniem png, jpg
\theoremstyle{definition} %styl dla definicji
\newtheorem{zad}{} 
\title{Multizestaw zadań}
\author{Robert Fidytek}
%\date{\today}
\date{}
\newcounter{liczniksekcji}
\newcommand{\kategoria}[1]{\section{#1}} %olreślamy nazwę kateforii zadań
\newcommand{\zadStart}[1]{\begin{zad}#1\newline} %oznaczenie początku zadania
\newcommand{\zadStop}{\end{zad}}   %oznaczenie końca zadania
%Makra opcjonarne (nie muszą występować):
\newcommand{\rozwStart}[2]{\noindent \textbf{Rozwiązanie (autor #1 , recenzent #2): }\newline} %oznaczenie początku rozwiązania, opcjonarnie można wprowadzić informację o autorze rozwiązania zadania i recenzencie poprawności wykonania rozwiązania zadania
\newcommand{\rozwStop}{\newline}                                            %oznaczenie końca rozwiązania
\newcommand{\odpStart}{\noindent \textbf{Odpowiedź:}\newline}    %oznaczenie początku odpowiedzi końcowej (wypisanie wyniku)
\newcommand{\odpStop}{\newline}                                             %oznaczenie końca odpowiedzi końcowej (wypisanie wyniku)
\newcommand{\testStart}{\noindent \textbf{Test:}\newline} %ewentualne możliwe opcje odpowiedzi testowej: A. ? B. ? C. ? D. ? itd.
\newcommand{\testStop}{\newline} %koniec wprowadzania odpowiedzi testowych
\newcommand{\kluczStart}{\noindent \textbf{Test poprawna odpowiedź:}\newline} %klucz, poprawna odpowiedź pytania testowego (jedna literka): A lub B lub C lub D itd.
\newcommand{\kluczStop}{\newline} %koniec poprawnej odpowiedzi pytania testowego 
\newcommand{\wstawGrafike}[2]{\begin{figure}[h] \includegraphics[scale=#2] {#1} \end{figure}} %gdyby była potrzeba wstawienia obrazka, parametry: nazwa pliku, skala (jak nie wiesz co wpisać, to wpisz 1)

\begin{document}
\maketitle


\kategoria{Wikieł/Z1.14e}
\zadStart{Zadanie z Wikieł Z 1.14 e) moja wersja nr [nrWersji]}
%[a]:[2,3,4,5,6,7,8,9,10,11,12,13,14,15,16,17,18,19,20]
%[b]:[2,3,4,5,6,7,8,9,10,11,12,13,14,15,16,17,18,19,20]
%math.gcd([a],[b])==1
Rozwiązać równanie: $
|[a]x^2+[b]x|=x|[a]x+[b]|$.
\zadStop
\rozwStart{Klaudia Klejdysz}{}
W poniższym zadaniu korzystać będziemy z następującej własności wartości bezwględnej:
\begin{equation*}
|x\cdot y|=|x|\cdot|y|
\end{equation*}
Przejdziemy teraz do równania:
\begin{equation*}
|[a]x^2+[b]x|=x|[a]x+[b]|
\end{equation*}
\begin{equation*}
|x([a]x+[b])|=x|[a]x+[b]|
\end{equation*}
\begin{equation*}
|x||[a]x+[b]|=x|[a]x+[b]|
\end{equation*}
\begin{equation*}
|x||[a]x+[b]|-x|[a]x+[b]|=0
\end{equation*}
\begin{equation*}
(|[a]x+[b]|)(|x|-x)=0
\end{equation*}
\begin{equation*}
|[a]x+[b]|=0\text{ }\lor\text{ }|x|-x=0
\end{equation*}
\begin{equation*}
[a]x+[b]=0\text{ }\lor\text{ }|x|=x
\end{equation*}
\begin{equation*}
[a]x=-[b]\text{ }\lor\text{ }x\in<0,\infty)
\end{equation*}
\begin{equation*}
x=-\frac{[b]}{[a]}\text{ }\lor\text{ }x\in<0,\infty)
\end{equation*}
\begin{equation*}
x\in<0,\infty)\cup\frac{-[b]}{[a]}
\end{equation*}
\rozwStop
\odpStart
$x\in<0,\infty)\cup\frac{-[b]}{[a]}$.
\odpStop
\testStart
A. $x\in<0,\infty)\cup\frac{-[b]}{[a]}$\\
B. $x\in\mathbb{R}$\\
C. $x=\frac{-[b]}{[a]}$\\
D. $x=0$
\testStop
\kluczStart
A
\kluczStop



\end{document}

