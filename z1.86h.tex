\documentclass[12pt, a4paper]{article}
\usepackage[utf8]{inputenc}
\usepackage{polski}
\usepackage{amsthm}  %pakiet do tworzenia twierdzeń itp.
\usepackage{amsmath} %pakiet do niektórych symboli matematycznych
\usepackage{amssymb} %pakiet do symboli mat., np. \nsubseteq
\usepackage{amsfonts}
\usepackage{graphicx} %obsługa plików graficznych z rozszerzeniem png, jpg
\theoremstyle{definition} %styl dla definicji
\newtheorem{zad}{} 
\title{Multizestaw zadań}
\author{Radosław Grzyb}
%\date{\today}
\date{}
\newcounter{liczniksekcji}
\newcommand{\kategoria}[1]{\section{#1}} %olreślamy nazwę kateforii zadań
\newcommand{\zadStart}[1]{\begin{zad}#1\newline} %oznaczenie początku zadania
\newcommand{\zadStop}{\end{zad}}   %oznaczenie końca zadania
%Makra opcjonarne (nie muszą występować):
\newcommand{\rozwStart}[2]{\noindent \textbf{Rozwiązanie (autor #1 , recenzent #2): }\newline} %oznaczenie początku rozwiązania, opcjonarnie można wprowadzić informację o autorze rozwiązania zadania i recenzencie poprawności wykonania rozwiązania zadania
\newcommand{\rozwStop}{\newline}                                            %oznaczenie końca rozwiązania
\newcommand{\odpStart}{\noindent \textbf{Odpowiedź:}\newline}    %oznaczenie początku odpowiedzi końcowej (wypisanie wyniku)
\newcommand{\odpStop}{\newline}                                             %oznaczenie końca odpowiedzi końcowej (wypisanie wyniku)
\newcommand{\testStart}{\noindent \textbf{Test:}\newline} %ewentualne możliwe opcje odpowiedzi testowej: A. ? B. ? C. ? D. ? itd.
\newcommand{\testStop}{\newline} %koniec wprowadzania odpowiedzi testowych
\newcommand{\kluczStart}{\noindent \textbf{Test poprawna odpowiedź:}\newline} %klucz, poprawna odpowiedź pytania testowego (jedna literka): A lub B lub C lub D itd.
\newcommand{\kluczStop}{\newline} %koniec poprawnej odpowiedzi pytania testowego 
\newcommand{\wstawGrafike}[2]{\begin{figure}[h] \includegraphics[scale=#2] {#1} \end{figure}} %gdyby była potrzeba wstawienia obrazka, parametry: nazwa pliku, skala (jak nie wiesz co wpisać, to wpisz 1)
\begin{document}
\maketitle
\kategoria{Wikieł/Z1.86h}
\zadStart{Zadanie z Wikieł Z 1.86h moja wersja nr [nrWersji]}
%[p1]:[1,2,3,4,5,6,7,8,9,10,11,12,13,14,15,16,17,18,19,20,21,22,23,24,25]
%[p2]:[1,2,3]
%[p3]:[2,3,5,6,7,10]
%[a]=[p3]**[p2]
%[Delta]=[p1]**2+4*1*[a]
%[sDelta]=[Delta]**(1/2)
%[tDelta]=int([sDelta])
%[t1]=int(([p1]-[tDelta])/2)
%[t2]=int(([p1]+[tDelta])/2)
%[Delta]>0 and ([sDelta]).is_integer() is True and [p3]!=[t2]
Rozwiązać równanie:
$$[p3]^x-[p1]>[p3]^{[p2]-x}$$
\zadStop
\rozwStart{Radosław Grzyb}{}
Mnożymy obie strony równania przez $[p3]^x$ otrzymując:
$$[p3]^{2x}-[p1]\cdot[p3]^x>[p3]^{[p2]}$$
$$[p3]^{2x}-[p1]\cdot[p3]^x-[a]>0$$
Podstawiając $t=[p3]^x$ otrzymamy do rozwiązania nierówność kwadratową:
$$t^{2}-[p1]\cdot t-[a]>0$$
$$\Delta_{t}=(-[p1])^2-4\cdot1\cdot[a]=[Delta]\implies \sqrt{\Delta_{t}}=[tDelta]$$\\
Czas znaleźć miejsca zerowe:
$$t_{1}=\frac{[p1]-[tDelta]}{2}=[t1]$$
$$t_{2}=\frac{[p1]+[tDelta]}{2}=[t2]$$
Współczynnik $a$ naszej funkcji kwadratowej jest dodatni oraz odrzucamy ujemne rozwiązania ponieważ $[p3]^x>0$. Zatem naszym rozwiązaniem jest:
$$t>[t2]$$
$$[p3]^x>[t2]\implies x>\frac{\ln[t2]}{\ln[p3]}$$
\rozwStop
\odpStart
\odpStop
\testStart
A.$$x>\frac{1}{\ln[p3]}$$
B.$$x>\ln[t2]\cdot\ln[p3]$$
C.$$x>\frac{\ln[t2]}{\ln[p3]}$$
D.$$x<\frac{\ln[t2]}{\ln[p3]}$$
\testStop
\kluczStart
C
\kluczStop
\end{document}