\documentclass[12pt, a4paper]{article}
\usepackage[utf8]{inputenc}
\usepackage{polski}
\usepackage{amsthm}  %pakiet do tworzenia twierdzeń itp.
\usepackage{amsmath} %pakiet do niektórych symboli matematycznych
\usepackage{amssymb} %pakiet do symboli mat., np. \nsubseteq
\usepackage{amsfonts}
\usepackage{graphicx} %obsługa plików graficznych z rozszerzeniem png, jpg
\theoremstyle{definition} %styl dla definicji
\newtheorem{zad}{} 
\title{Multizestaw zadań}
\author{Radosław Grzyb}
%\date{\today}
\date{}
\newcounter{liczniksekcji}
\newcommand{\kategoria}[1]{\section{#1}} %olreślamy nazwę kateforii zadań
\newcommand{\zadStart}[1]{\begin{zad}#1\newline} %oznaczenie początku zadania
\newcommand{\zadStop}{\end{zad}}   %oznaczenie końca zadania
%Makra opcjonarne (nie muszą występować):
\newcommand{\rozwStart}[2]{\noindent \textbf{Rozwiązanie (autor #1 , recenzent #2): }\newline} %oznaczenie początku rozwiązania, opcjonarnie można wprowadzić informację o autorze rozwiązania zadania i recenzencie poprawności wykonania rozwiązania zadania
\newcommand{\rozwStop}{\newline}                                            %oznaczenie końca rozwiązania
\newcommand{\odpStart}{\noindent \textbf{Odpowiedź:}\newline}    %oznaczenie początku odpowiedzi końcowej (wypisanie wyniku)
\newcommand{\odpStop}{\newline}                                             %oznaczenie końca odpowiedzi końcowej (wypisanie wyniku)
\newcommand{\testStart}{\noindent \textbf{Test:}\newline} %ewentualne możliwe opcje odpowiedzi testowej: A. ? B. ? C. ? D. ? itd.
\newcommand{\testStop}{\newline} %koniec wprowadzania odpowiedzi testowych
\newcommand{\kluczStart}{\noindent \textbf{Test poprawna odpowiedź:}\newline} %klucz, poprawna odpowiedź pytania testowego (jedna literka): A lub B lub C lub D itd.
\newcommand{\kluczStop}{\newline} %koniec poprawnej odpowiedzi pytania testowego 
\newcommand{\wstawGrafike}[2]{\begin{figure}[h] \includegraphics[scale=#2] {#1} \end{figure}} %gdyby była potrzeba wstawienia obrazka, parametry: nazwa pliku, skala (jak nie wiesz co wpisać, to wpisz 1)
\begin{document}
\maketitle
\kategoria{Wikieł/Z1.88}
\zadStart{Zadanie z Wikieł Z 1.88 moja wersja nr [nrWersji]}
%[p1]:[5,7,11,13,15]
%[p3]:[5,23,77]
%[p2]:[7,55,247]
%[wykl]:[2,3,4]
%[Czw]=4+[p3]
%[Trz]=9+[p2]
%([p3]==5 and [p2]==7 and [wykl]==2) or ([p3]==23 and [p2]==55 and [wykl]==3) or ([p3]==77 and [p2]==247 and [wykl]==4)
Dane są funkcje $f(x)=4^{x+1}-[p2]\cdot3^{x}$ i $g(x)=3^{x+2}-[p3]\cdot4^{x}$.\\ Rozwiązać nierówność $f(x)\leq g(x)$.
\zadStop
\rozwStart{Radosław Grzyb}{}
Przekształcamy naszą nierówność:
$$4^{[p1]x+1}-[p2]\cdot3^{[p1]x}\leq3^{[p1]x+2}-[p3]\cdot4^{[p1]x}$$
$$4\cdot4^{[p1]x}-[p2]\cdot3^{[p1]x}\leq9\cdot3^{[p1]x}-[p3]\cdot4^{[p1]x}$$
$$4\cdot4^{[p1]x}+[p3]\cdot4^{[p1]x}\leq9\cdot3^{[p1]x}+[p2]\cdot3^{[p1]x}$$
$$[Czw]\cdot4^{[p1]x}\leq[Trz]\cdot3^{[p1]x}$$
Dzielimy obie stronyu równania przez $3^{[p1]x}$ oraz $[Czw]$ otrzymując:
$$\left(\frac{4}{3}\right)^{[p1]x}\leq\frac{[Trz]}{[Czw]}$$
$$\left(\frac{4}{3}\right)^{[p1]x}\leq\left(\frac{4}{3}\right)^{[wykl]}$$
Logarytmując otrzymujemy prostą nierówność i zarazem odpowiedź:
$$[p1]x\leq[wykl]\implies x\leq\frac{[wykl]}{[p1]}$$
\rozwStop
\odpStart
$$x\leq\frac{[wykl]}{[p1]}$$
\odpStop
\testStart
A.$$x\leq\frac{[wykl]}{[p1]}$$
B.$$x\geq\frac{[wykl]}{[p1]}$$
C.$$x\leq\-\frac{[wykl]}{[p1]}$$
D.$$x\leq[p1]$$
\testStop
\kluczStart
A
\kluczStop
\end{document}