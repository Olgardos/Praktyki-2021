\documentclass[12pt, a4paper]{article}
\usepackage[utf8]{inputenc}
\usepackage{polski}
\usepackage{amsthm}  %pakiet do tworzenia twierdzeń itp.
\usepackage{amsmath} %pakiet do niektórych symboli matematycznych
\usepackage{amssymb} %pakiet do symboli mat., np. \nsubseteq
\usepackage{amsfonts}
\usepackage{graphicx} %obsługa plików graficznych z rozszerzeniem png, jpg
\theoremstyle{definition} %styl dla definicji
\newtheorem{zad}{} 
\title{Multizestaw zadań}
\author{Radosław Grzyb}
%\date{\today}
\date{}
\newcounter{liczniksekcji}
\newcommand{\kategoria}[1]{\section{#1}} %olreślamy nazwę kateforii zadań
\newcommand{\zadStart}[1]{\begin{zad}#1\newline} %oznaczenie początku zadania
\newcommand{\zadStop}{\end{zad}}   %oznaczenie końca zadania
%Makra opcjonarne (nie muszą występować):
\newcommand{\rozwStart}[2]{\noindent \textbf{Rozwiązanie (autor #1 , recenzent #2): }\newline} %oznaczenie początku rozwiązania, opcjonarnie można wprowadzić informację o autorze rozwiązania zadania i recenzencie poprawności wykonania rozwiązania zadania
\newcommand{\rozwStop}{\newline}                                            %oznaczenie końca rozwiązania
\newcommand{\odpStart}{\noindent \textbf{Odpowiedź:}\newline}    %oznaczenie początku odpowiedzi końcowej (wypisanie wyniku)
\newcommand{\odpStop}{\newline}                                             %oznaczenie końca odpowiedzi końcowej (wypisanie wyniku)
\newcommand{\testStart}{\noindent \textbf{Test:}\newline} %ewentualne możliwe opcje odpowiedzi testowej: A. ? B. ? C. ? D. ? itd.
\newcommand{\testStop}{\newline} %koniec wprowadzania odpowiedzi testowych
\newcommand{\kluczStart}{\noindent \textbf{Test poprawna odpowiedź:}\newline} %klucz, poprawna odpowiedź pytania testowego (jedna literka): A lub B lub C lub D itd.
\newcommand{\kluczStop}{\newline} %koniec poprawnej odpowiedzi pytania testowego 
\newcommand{\wstawGrafike}[2]{\begin{figure}[h] \includegraphics[scale=#2] {#1} \end{figure}} %gdyby była potrzeba wstawienia obrazka, parametry: nazwa pliku, skala (jak nie wiesz co wpisać, to wpisz 1)
\begin{document}
\maketitle
\kategoria{Beger/c1.15g}
\zadStart{Zadanie z Beger C 1.15g moja wersja nr [nrWersji]}
%[p1]:[2,3,5,6,7,9,11,13,14,15,17,19,21]
Stosując odpowiednie metody całkowania obliczyć całkę:
$$\int \frac{dx}{\sqrt{[p1]\pi+e^x}}$$
\zadStop
\rozwStart{Radosław Grzyb}{}
Dokonujemy podstawienia:
$$u=[p1]\pi+e^x \implies du=e^x dx\implies dx=\frac{1}{e^x}du$$
$$e^x=u-[p1]\pi$$
Otrzymujemy:
$$\int \frac{1}{\sqrt{[p1]\pi+e^x}}\cdot\frac{1}{e^x}\,du=\int \frac{1}{\sqrt{u}(u-[p1]\pi)}\,du$$
Dokonujemy kolejnego podstawienia:
$$t=\sqrt{u} \implies dt=\frac{1}{2\sqrt{u}}du\implies du=2\sqrt{u}dt$$
$$t^2=u$$
Otrzymujemy:
$$\int \frac{1}{\sqrt{u}(u-[p1]\pi)}\cdot2\sqrt{u}\,dt=2\int\frac{1}{t^2-[p1]\pi}\,dt$$
Do policzenia otrzymanej całki użyjemy gotowego wzoru: $$\int\frac{dx}{x^2-a^2}=\frac{1}{2a}\ln\left|\frac{x-a}{x+a}\right|+C$$
Po paru prostych obliczeniach otrzymujemy finalny wynik:
$$2\int\frac{1}{t^2-\sqrt{[p1]\pi}^2}\,dt=\frac{1}{\sqrt{[p1]\pi}}\ln\left|\frac{t-\sqrt{[p1]\pi}}{t+\sqrt{[p1]\pi}}\right|+C=\frac{1}{\sqrt{[p1]\pi}}\ln\left|\frac{\sqrt{u}-\sqrt{[p1]\pi}}{\sqrt{u}+\sqrt{[p1]\pi}}\right|+C=$$
$$=\frac{1}{\sqrt{[p1]\pi}}\ln\left|\frac{\sqrt{[p1]\pi+e^x}-\sqrt{[p1]\pi}}{\sqrt{[p1]\pi+e^x}+\sqrt{[p1]\pi}}\right|+C$$
\rozwStop
\odpStart
$$\frac{1}{\sqrt{[p1]\pi}}\ln\left|\frac{\sqrt{[p1]\pi+e^x}-\sqrt{[p1]\pi}}{\sqrt{[p1]\pi+e^x}+\sqrt{[p1]\pi}}\right|+C$$
\odpStop
\testStart
A.$$\sqrt[[p1]]{x}e^{\sqrt[[p1]]{x}}-e^{\sqrt[[p1]]{x}}+C$$
B.$$\frac{1}{\sqrt{[p1]\pi}}\ln\left|\frac{\sqrt{[p1]\pi+e^x}-\sqrt{[p1]\pi}}{\sqrt{[p1]\pi+e^x}+\sqrt{[p1]\pi}}\right|+C$$
C.$$\frac{1}{\sqrt{[p1]\pi}}\ln\left|\frac{\sqrt{[p1]\pi}-\sqrt{[p1]\pi}}{\sqrt{[p1]\pi}+\sqrt{[p1]\pi}}\right|+C$$
D.$$\ln\left|\frac{\left(\frac{1}{[p1]}\right)^x-1}{\left(\frac{1}{[p1]}\right)^x+1}\right|+C$$
\testStop
\kluczStart
B
\kluczStop
\end{document}