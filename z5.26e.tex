\documentclass[12pt, a4paper]{article}
\usepackage[utf8]{inputenc}
\usepackage{polski}

\usepackage{amsthm}  %pakiet do tworzenia twierdzeń itp.
\usepackage{amsmath} %pakiet do niektórych symboli matematycznych
\usepackage{amssymb} %pakiet do symboli mat., np. \nsubseteq
\usepackage{amsfonts}
\usepackage{graphicx} %obsługa plików graficznych z rozszerzeniem png, jpg
\theoremstyle{definition} %styl dla definicji
\newtheorem{zad}{} 
\title{Multizestaw zadań}
\author{Robert Fidytek}
%\date{\today}
\date{}
\newcounter{liczniksekcji}
\newcommand{\kategoria}[1]{\section{#1}} %olreślamy nazwę kateforii zadań
\newcommand{\zadStart}[1]{\begin{zad}#1\newline} %oznaczenie początku zadania
\newcommand{\zadStop}{\end{zad}}   %oznaczenie końca zadania
%Makra opcjonarne (nie muszą występować):
\newcommand{\rozwStart}[2]{\noindent \textbf{Rozwiązanie (autor #1 , recenzent #2): }\newline} %oznaczenie początku rozwiązania, opcjonarnie można wprowadzić informację o autorze rozwiązania zadania i recenzencie poprawności wykonania rozwiązania zadania
\newcommand{\rozwStop}{\newline}                                            %oznaczenie końca rozwiązania
\newcommand{\odpStart}{\noindent \textbf{Odpowiedź:}\newline}    %oznaczenie początku odpowiedzi końcowej (wypisanie wyniku)
\newcommand{\odpStop}{\newline}                                             %oznaczenie końca odpowiedzi końcowej (wypisanie wyniku)
\newcommand{\testStart}{\noindent \textbf{Test:}\newline} %ewentualne możliwe opcje odpowiedzi testowej: A. ? B. ? C. ? D. ? itd.
\newcommand{\testStop}{\newline} %koniec wprowadzania odpowiedzi testowych
\newcommand{\kluczStart}{\noindent \textbf{Test poprawna odpowiedź:}\newline} %klucz, poprawna odpowiedź pytania testowego (jedna literka): A lub B lub C lub D itd.
\newcommand{\kluczStop}{\newline} %koniec poprawnej odpowiedzi pytania testowego 
\newcommand{\wstawGrafike}[2]{\begin{figure}[h] \includegraphics[scale=#2] {#1} \end{figure}} %gdyby była potrzeba wstawienia obrazka, parametry: nazwa pliku, skala (jak nie wiesz co wpisać, to wpisz 1)

\begin{document}
\maketitle

\kategoria{Wikieł/Z5.26c}

\zadStart{Zadanie z Wikieł Z 5.26 c) moja wersja nr [nrWersji]}
%[a]:[2,3,4,5,6,7,8,9,10,11,12,13]
%[b]=[a]*4 - 1
%[c]=-1/[a]
%[d]=round([c],2)
%[e]=-4/[b]
%[g]=round([e],2)
%[l]=max([d],[g])
%[m]=min([d],[g])
%[l]!=[m]
Wyznaczyć wartość największą oraz wartość najmniejszą funkcji w przedziale. 
$$y = \frac{1}{x^2-[a]}, \quad \big\langle-\frac{1}{2},\frac{1}{2}\big\rangle$$
\zadStop

\rozwStart{Natalia Danieluk}{}
Funkcja $f$ jest ciągła w przedziale $\big\langle-\frac{1}{2},\frac{1}{2}\big\rangle$. Wartość największą $M$ i najmniejszą $m$ znajdziemy więc wśród ekstremów tej funkcji w przedziale $\big\langle-\frac{1}{2},\frac{1}{2}\big\rangle$ oraz na końcach przedziału, tj. $f(-\frac{1}{2})$ i $f(\frac{1}{2})$. \\
A zatem obliczamy pochodną i wyznaczamy jej miejsca zerowe:
$$ f'(x) = -\frac{2 x}{(x^2-[a])^2} $$
$$ f'(x) = 0 \Leftrightarrow 2 x = 0 \Leftrightarrow x = 0 $$ 
Punkt $0$ należy do przedziału. Sprawdzamy wartości funkcji w tym punkcie oraz na końcach naszego przedziału: \\
$$ f(0) = -\frac{1}{[a]}\approx [d],\quad f\big(-\frac{1}{2}\big) = -\frac{4}{[b]} \approx [g],\quad f\big(\frac{1}{2}\big) = -\frac{4}{[b]} \approx [g] $$
\rozwStop

\odpStart
Wartość największa $M$ funkcji $f$ w przedziale $\big\langle-\frac{1}{2},\frac{1}{2}\big\rangle$ to $[l]$, natomiast wartość najmniejsza $m$ to $[m]$.
\odpStop

\testStart
A. $M=[m], m=[l]$
B. $M=5, m=0$
C. $M=[l], m=[m]$
D. $M=\frac{1}{2}, m=-\frac{1}{2}$
\testStop

\kluczStart
C
\kluczStop

\end{document}